%
% SECTION
%
\section{Theoretical Maximum Performance (Gflops)}

The Raspberry Pi 4 Model B is based on the Broadcom BCM2711 System on a Chip (SoC). The BCM2711 contains 4 Arm Cortex-A72 cores clocked at 1.5 GHz.

Each core implements the 64-bit Armv8-A Instruction Set Architecture (ISA). This instruction set includes Advanced SIMD instructions which operate on a single 128-bit SIMD pipeline. This 128-bit pipeline can conduct two 64-bit double precision floating point operations (Flops) per clock cycle.  

A \emph{fused multiply-add} (FMA) instruction implements a multiplication followed by an add in a single instruction. The main purpose of FMA instructions is to improve result accuracy by conducting a single rounding operation on completion of both the multiplication and the add operations. A single FMA instruction counts as two Flops. 

The theoretical maximum performance of a single Aerin Cluster node, $R_{peak}$, is therefore:

\begin{align}
R_{peak} &= 4 \textrm{ cores} \times 1.5 \textrm{ GHz} \times 2 \textrm{ doubles} \times 2 \textrm{ FMA}\\
&= 24 \textrm{ Gflops}
\end{align}

This is only achievable continuously if every instruction in a program is an FMA instruction, which obviously cannot be the case, since data has to be loaded from memory and stored back into memory. Nevertheless, this is the standard measure of theoretical maximum performance.

The theoretical maximum performance of the Aerin Cluster as a whole is therefore:

\begin{align}
R_{peak} &= 8 \textrm{ nodes} \times 24 \textrm{ Gflops}\\
&= 192 \textrm{ Gflops}
\end{align}

For the High Performance Linpack benchmark, to achieve 100\% performance requires a problem size that utilises 100\% of memory. Because the operating system requires memory, is it not possible to use 100\% for benchmarks.

The Linux \verb|dmesg| command prints out the kernel boot messages, which can be searched using \verb|grep| to determine how memory is utilised on the system:

\lstset{style=type}
\begin{lstlisting}
$ dmesg | grep Memory
\end{lstlisting}

\lstset{style=type}
\begin{lstlisting}
[    0.000000] Memory: 3783876K/4050944K available (11772K kernel code, 1236K rwdata, 4244K rodata, 6144K init, 1072K bss, 201532K reserved, 65536K cma-reserved)
\end{lstlisting}

As can be seen, 37838776k of memory is available, which equates to 90\% of the 4 GB (4194304k) on each node. It would be optimistic to expect to use every byte of this 90\%, and using any more than this would result in swap space being used which would negatively impact benchmark results.

So, for the HPL baseline benchmarks, 80\% of memory was chosen for the problem size. This is the amount suggested as an initial ``good guess'' in the HPL Frequently Asked Questions.

The above necessarily results in the baseline benchmarks only being able to achieve 80\% of $R_{peak}$ at best, 4.8 Gflops for a single core, 19.2 Gflops for a single node, and 153.6 Gflops for the 8 node cluster. These values are indicated on the HPL baseline result plots.


%
% SECTION
%
\section{HPL Baseline}

The HPL benchmark software was compiled locally from source. Detailed instructions on how to do this are in Part II Chapter??

Ubuntu 20.04 LTS 64-bit packages, without any tweaks...

80\% of memory

Methodology...

1 core... to investigate single core performance... caveats... use 1GB of memory...

1 node... to investigate inter-core performance...

2 nodes... to investigate inter-core and inter-node performance...

1..8 nodes ... to investigate over scaling of performance with node count... with optimal N, NB, P and Q parameters determined from 2 node investigation... caveats...


%
% SUB SECTION
%
\subsection{HPL 1 Core Baseline}

The purpose of this baseline is to determine the performance of a single core running a single \verb|xhpl| process, with the single core having exclusive access to the shared L2 cache. 

As discussed in the previous section, the HPL problem size is restricted to 80\% of available memory. In the case of this baseline, this is 80\% of a single node's 4 GB.

Using values of block size NB from 32 to 256, in increments of 8, and using formula ?? to ensure the problem size N is an integer multiple of NB, results in the table below of NB and N combinations.

\begin{table}[H]
\begin{center}
	\begin{tabular}{ |c|c|c|c|c|c|c|c|c|c| } 
		\hline
		NB & N & NB & N & NB & N & NB & N & NB & N \\ 
		\hline
		32 & 18528 &  80 & 18480 & 128 & 18432 & 176 & 18480 & 224 & 18368 \\ 
		40 & 18520 &  88 & 18480 & 136 & 18496 & 184 & 18400 & 232 & 18328 \\ 
 		48 & 18528 &  96 & 18528 & 144 & 18432 & 192 & 18432 & 240 & 18480 \\
		56 & 18536 & 104 & 18512 & 152 & 18392 & 200 & 18400 & 248 & 18352 \\ 
 		64 & 18496 & 112 & 18480 & 160 & 18400 & 208 & 18512 & 256 & 18432 \\
		72 & 18504 & 120 & 18480 & 168 & 18480 & 216 & 18360 &   - &     - \\ 
 		\hline
	\end{tabular}
\end{center}
\caption{\label{tab:table-name}1 Core NB and N Combinations using 80\% of 4 GB Memory.}
\end{table}


\begin{figure}
	\begin{subfigure}{1.0\textwidth}
		\centering
		\includegraphics[width=0.85\textwidth]{openmpi_gflops_vs_nb_1_core_80_percent_memory.pdf}
		\caption{Pure OpenMPI}
		\label{fig:subim1}
	\end{subfigure}
	\par\bigskip
	\begin{subfigure}{1.0\textwidth}
		\centering
		\includegraphics[width=0.85\textwidth]{openmp_gflops_vs_nb_1_core_80_percent_memory.pdf}
		\caption{Hybrid OpenMPI/OpenMP}
		\label{fig:subim2}
	\end{subfigure}
\caption{1 Core $R_{max}$ vs NB using 80\% memory.}
\label{fig:image2}
\end{figure}

The results are plotted in Figure ??.


%
% SUB SUB SECTION
%
\subsubsection{Observations}

As expected, there is no noticeable performance difference between a pure OpenMPI and hybrid OpenMPI/OpenMP topology for a single \verb|xhpl| process running on a single core.

Both topologies attain 80\% $R_{peak}$ for a single core.

Discussion about OpenBLAS and BLIS internal kernel blocking...


%
% SUB SECTION
%
\subsection{HPL 1 Node Baseline}

The purpose of this baseline is to determine the performance of the 4 cores of a single node. In this case each core has to share access to the L2 cache, which will result in more main memory accesses. It is therefore anticipated that this will result in a performance reduction, per core, compared to the single core case.

As per the single core benchmark, the HPL problem size is restricted to 80\% of available memory. Again, in the case, this is 80\% of a single node's 4 GB. This results in the same table of NB and N combinations as the single core benchmark.

\begin{table}[H]
\begin{center}
	\begin{tabular}{ |c|c|c|c|c|c|c|c|c|c| } 
		\hline
		NB & N & NB & N & NB & N & NB & N & NB & N \\ 
		\hline
		32 & 18528 &  80 & 18480 & 128 & 18432 & 176 & 18480 & 224 & 18368 \\ 
		40 & 18520 &  88 & 18480 & 136 & 18496 & 184 & 18400 & 232 & 18328 \\ 
 		48 & 18528 &  96 & 18528 & 144 & 18432 & 192 & 18432 & 240 & 18480 \\
		56 & 18536 & 104 & 18512 & 152 & 18392 & 200 & 18400 & 248 & 18352 \\ 
 		64 & 18496 & 112 & 18480 & 160 & 18400 & 208 & 18512 & 256 & 18432 \\
		72 & 18504 & 120 & 18480 & 168 & 18480 & 216 & 18360 &   - &     - \\ 
 		\hline
	\end{tabular}
\end{center}
\caption{\label{tab:table-name}1 Node NB and N Combinations using 80\% of 4 GB Memory.}
\end{table}


The results are plotted in Figure ??

\begin{figure}[H]
	\begin{subfigure}{1.0\textwidth}
		\centering
		\includegraphics[width=0.85\textwidth]{openmpi_gflops_vs_nb_1_node_80_percent_memory.pdf}
		\caption{Pure OpenMPI}
		\label{fig:subim1}
	\end{subfigure}
	\par\bigskip
	\begin{subfigure}{1.0\textwidth}
		\centering
		\includegraphics[width=0.85\textwidth]{openmp_gflops_vs_nb_1_node_80_percent_memory.pdf}
		\caption{Hybrid OpenMPI/OpenMP}
		\label{fig:subim2}
	\end{subfigure}
\caption{1 Node $R_{max}$ vs NB using 80\% memory.}
\label{fig:image2}
\end{figure}


%
% SUB SUB SECTION
%
\subsubsection{Observations}

As anticipated, there is indeed a reduction in performance per core, 80\% $R_{peak}$ in no longer attained.

Pure OpenMPI topology attains a $R_{max}$ of ?? with an NB of ??.

The hybrid OpenMPI/OpenMP topology attains a $R_{max}$ of ?? with an NB of ??.


%
% SUB SECTION
%
\subsection{HPL 2 Node Baseline}

The purpose of this baseline is to determine the performance of 2 nodes. Now, each core not only has to share access to the L2 cache, but the cache may be refreshed with data less frequently due to network delays and competition between the nodes for access to network. It is therefore anticipated that this will result in a performance reduction, per node, compared to the single node case.

For this baseline the HPL problem size is restricted to 80\% of 2 nodes combined memory, 80\% of 8 GB. This results in NB and N combinations as tabulated below:

\begin{table}[H]
\begin{center}
	\begin{tabular}{ |c|c|c|c|c|c|c|c|c|c| } 
		\hline
		NB & N & NB & N & NB & N & NB & N & NB & N \\ 
		\hline
		32 & 26208 &  80 & 26160 & 128 & 26112 & 176 & 26048 & 224 & 26208 \\ 
		40 & 26200 &  88 & 26136 & 136 & 26112 & 184 & 26128 & 232 & 25984 \\ 
 		48 & 26208 &  96 & 26208 & 144 & 26208 & 192 & 26112 & 240 & 26160 \\
		56 & 26208 & 104 & 26208 & 152 & 26144 & 200 & 26200 & 248 & 26040 \\ 
 		64 & 26176 & 112 & 26208 & 160 & 26080 & 208 & 26208 & 256 & 26112 \\
		72 & 26208 & 120 & 26160 & 168 & 26208 & 216 & 26136 &   - &     - \\ 
 		\hline
	\end{tabular}
\end{center}
\caption{\label{tab:table-name}2 Node NB and N Combinations using 80\% of 8 GB Memory.}
\end{table}


The results are plotted in Figure ??

\begin{figure}[H]
	\begin{subfigure}{1.0\textwidth}
		\centering
		\includegraphics[width=0.85\textwidth]{openmpi_gflops_vs_nb_2_node_80_percent_memory.pdf}
		\caption{Pure OpenMPI}
		\label{fig:subim1}
	\end{subfigure}
	\par\bigskip
	\begin{subfigure}{1.0\textwidth}
		\centering
		\includegraphics[width=0.85\textwidth]{openmp_gflops_vs_nb_2_node_80_percent_memory.pdf}
		\caption{Hybrid OpenMPI/OpenMP}
		\label{fig:subim2}
	\end{subfigure}
\caption{2 Node $R_{max}$ vs NB using 80\% memory.}
\label{fig:image2}
\end{figure}



%
% SUB SUB SECTION
%
\subsubsection{Observations}




%
% SUB SECTION
%
\subsection{HPL Cluster Baseline}

This cluster baseline uses the optimum values of NB from the 2 Node Baseline. For each of the 4 BLAS library combinations, OpenBLAS serial, OpenBLAS OpenMP, BLIS serial, and BLIS OpenMP, with the corresponding value of N for 80\% of memory for the particular node count is used, as tabulated below.

\begin{center}
	\begin{tabular}{ |c|c|c|c|c|c|c|c| } 
		\hline
		\multicolumn{2}{|c|}{} & \multicolumn{6}{c|}{Nodes} \\
		\hline
		BLAS & NB & 3 & 4 & 5 & 6 & 7 & 8 \\ 
		\hline
		OpenBLAS Serial & 000 & 32000 & 36000 & 41000 & 45000 & 48000 & 52000 \\ 
		OpenBLAS OpenMP & 000 & 32000 & 36000 & 41000 & 45000 & 48000 & 52000 \\ 
 		BLIS Serial     & 000 & 32000 & 36000 & 41000 & 45000 & 48000 & 52000 \\
		BLIS OpenMP     & 000 & 32000 & 36000 & 41000 & 45000 & 48000 & 52000 \\ 
 		\hline
	\end{tabular}
\end{center}


\begin{table}
\begin{center}
\begin{tabular}{ |l|c|c|c|c|c|c| } 
\hline
                & Nodes & N & NB & P & Q & $R_{max}$ (Gflops) \\ 
\hline
OpenBLAS Serial & 3 & 32040 & 120 & 1 & 12 & 3.3720e+01 \\ 
                & 3 & 32040 & 120 & 2 &  6 & 3.1946e+01 \\
                & 3 & 32040 & 120 & 3 &  4 & 3.3844e+01 \\
                \cline{2-7} 
                & 4 & 36960 & 120 & 1 & 16 & 4.7742e+01 \\ 
                & 4 & 36960 & 120 & 2 &  8 & 4.9390e+01 \\ 
                \cline{2-7} 
                & 5 & 41400 & 120 & 1 & 20 & 5.6513e+01 \\ 
                & 5 & 41400 & 120 & 2 & 10 & 5.6038e+01 \\ 
                & 5 & 41400 & 120 & 4 &  5 & 5.5649e+01 \\ 
                \cline{2-7} 
                & 6 & 45360 & 120 & 1 & 24 & 6.8392e+01 \\ 
                & 6 & 45360 & 120 & 2 & 12 & 7.3856e+01 \\ 
                & 6 & 45360 & 120 & 3 &  8 & 6.9952e+01 \\ 
                \cline{2-7} 
                & 7 & 48960 & 120 & 1 & 28 & 7.8248e+01 \\ 
                & 7 & 48960 & 120 & 2 & 14 & 8.1017e+01 \\ 
                & 7 & 48960 & 120 & 4 &  7 & 8.1433e+01 \\ 
                \cline{2-7} 
                & 8 & 52320 & 120 & 1 & 32 & 8.6787e+01 \\ 
                & 8 & 52320 & 120 & 2 & 16 & 9.5517e+01 \\ 
                & 8 & 52320 & 120 & 4 &  8 & 9.5525e+01 \\ 
\hline
OpenBLAS OpenMP & 3 & 32032 & 88 & 1 & 3 & 3.7842e+01 \\ 
                \cline{2-7} 
                & 4 & 37048 & 88 & 1 & 4 & 4.8657e+01 \\ 
                \cline{2-7} 
                & 5 & 41448 & 88 & 1 & 5 & 6.0428e+01 \\ 
                \cline{2-7} 
                & 6 & 45320 & 88 & 1 & 6 & 6.8713e+01 \\ 
                & 6 & 45320 & 88 & 2 & 3 & 7.3722e+01 \\ 
                \cline{2-7} 
                & 7 & 49016 & 88 & 1 & 7 & 7.8712e+01 \\ 
                \cline{2-7} 
                & 8 & 52360 & 88 & 1 & 8 & 9.4245e+01 \\ 
                & 8 & 52360 & 88 & 2 & 4 & 9.6630e+01 \\ 
\hline
\end{tabular}
\end{center}
\caption{\label{tab:table-name}HPL Cluster Baseline using OpenBLAS.}
\end{table}



\begin{table}
\begin{center}
\begin{tabular}{ |l|c|c|c|c|c|c| } 
\hline
                & Nodes & N & NB & P & Q & $R_{max}$ (Gflops) \\ 
\hline
BLIS Serial     & 3 & 32088 & 168 & 1 & 12 & 3.9005e+01 \\ 
                & 3 & 32088 & 168 & 2 &  6 & 3.9050e+01 \\ 
                & 3 & 32088 & 168 & 3 &  4 & 3.8958e+01 \\ 
                \cline{2-7} 
                & 4 & 36960 & 168 & 1 & 16 & 4.9694e+01 \\ 
                & 4 & 36960 & 168 & 2 &  8 & 5.4268e+01\\ 
                \cline{2-7} 
                & 5 & 41328 & 168 & 1 & 20 & 5.5398e+01 \\ 
                & 5 & 41328 & 168 & 2 & 10 & 6.5226e+01 \\ 
                & 5 & 41328 & 168 & 4 &  5 & 6.2356e+01 \\ 
                \cline{2-7} 
                & 6 & 45360 & 168 & 1 & 24 & 7.0278e+01 \\ 
                & 6 & 45360 & 168 & 2 & 12 & 7.9685e+01 \\ 
                & 6 & 45360 & 168 & 3 &  8 & 7.5475e+01 \\ 
                \cline{2-7} 
                & 7 & 48888 & 168 & 1 & 28 & 8.0168e+01 \\ 
                & 7 & 48888 & 168 & 2 & 14 & 8.7571e+01 \\ 
                & 7 & 48888 & 168 & 4 &  7 & 8.6035e+01 \\ 
                \cline{2-7} 
                & 8 & 52416 & 168 & 1 & 32 & 9.1148e+01 \\ 
                & 8 & 52416 & 168 & 2 & 16 & 1.0341e+02 \\ 
                & 8 & 52416 & 168 & 4 &  8 & 1.0190e+02 \\ 
\hline
BLIS OpenMP     & 3 & 32000 & 200 & 1 & 3 & 3.5132e+01 \\ 
                \cline{2-7} 
                & 4 & 37000 & 200 & 1 & 4 & 4.6953e+01 \\ 
                \cline{2-7} 
                & 5 & 41400 & 200 & 1 & 5 & 6.2550e+01 \\ 
                \cline{2-7} 
                & 6 & 45400 & 200 & 1 & 6 & 6.7204e+01 \\ 
                & 6 & 45400 & 200 & 2 & 3 & 7.2585e+01 \\ 
                \cline{2-7} 
                & 7 & 49000 & 200 & 1 & 7 & 8.1255e+01 \\ 
                \cline{2-7} 
                & 8 & 52400 & 200 & 1 & 8 & 9.1180e+01 \\ 
                & 8 & 52400 & 200 & 2 & 4 & 1.0041e+02 \\ 
\hline
\end{tabular}
\end{center}
\caption{\label{tab:table-name}HPL Cluster Baseline using BLIS.}
\end{table}



The baseline results are presented in Figure ??.

\begin{figure}[H]
	\begin{subfigure}{1.0\textwidth}
		\centering
		\includegraphics[width=0.85\textwidth]{openmpi_gflops_vs_nodes_80_percent_memory.pdf}
		\caption{Pure OpenMPI}
		\label{fig:subim1}
	\end{subfigure}
	\par\bigskip
	\begin{subfigure}{1.0\textwidth}
		\centering
		\includegraphics[width=0.85\textwidth]{openmp_gflops_vs_nodes_80_percent_memory.pdf}
		\caption{Hybrid OpenMPI/OpenMP}
		\label{fig:subim2}
	\end{subfigure}
\caption{$R_{max}$ vs Nodes using 80\% memory.}
\label{fig:image2}
\end{figure}


%
% SUB SECTION
%
\subsection{Observations}

Best NB...

PxQ discussion... 1x8 vs 2x4... ethernet comment...

Iperf...

htop...

top...

perf...

cache misses...

software interrupts...

Suggests... improve network efficiency?



%
% SECTION
%
\section{HPCC Baseline}

The HPCC baseline benchmarks were run using all 8 nodes of the Aerin Cluster. The results for each benchmark are presented below.


%
% SUB SECTION
%
\subsection{HPL}

\begin{table}[H]
\begin{center}
\begin{tabular}{ |l|c|c|c|c|c|c| } 
\hline
                & Nodes & N & NB & P & Q & $R_{max}$ (Gflops) \\ 
\hline
OpenBLAS Serial & 8 & 52320 & 120 & 1 & 32 & \\ 
                & 8 & 52320 & 120 & 2 & 16 & \\ 
                & 8 & 52320 & 120 & 4 & 8  & \\ 
\hline
OpenBLAS OpenMP & 8 & 52360 &  88 & 1 & 8  & 8.685e+01 \\ 
                & 8 & 52360 &  88 & 2 & 4  & 9.497e+01 \\ 
\hline
BLIS Serial     & 8 & 52416 & 168 & 1 & 32 & 8.866e+01 \\ 
                & 8 & 52416 & 168 & 2 & 16 & 1.014e+02 \\ 
                & 8 & 52416 & 168 & 4 & 8  & 1.005e+02 \\ 
\hline
BLIS OpenMP     & 8 & 52400 & 200 & 1 & 8  & 8.163e+01 \\ 
                & 8 & 52400 & 200 & 2 & 4  & 8.841e+01 \\ 
\hline
\end{tabular}
\end{center}
\caption{\label{tab:table-name}HPCC HPL.}
\end{table}


%
% SUB SECTION
%
\subsection{DGEMM}

\begin{table}[H]
\begin{center}
\begin{tabular}{ |l|l| } 
\hline
                & Results \\ 
\hline
OpenBLAS Serial & DGEMM\_N=5339 \\
                & StarDGEMM\_Gflops=3.59743 \\
                & SingleDGEMM\_Gflops=4.91086 \\
\hline
OpenBLAS OpenMP & DGEMM\_N=10687 \\
                & StarDGEMM\_Gflops=14.4261 \\
                & SingleDGEMM\_Gflops=14.426 \\
\hline
BLIS Serial     & DGEMM\_N=5349 \\
                & StarDGEMM\_Gflops=3.02439 \\
                & SingleDGEMM\_Gflops=4.95418 \\
\hline
BLIS OpenMP     & DGEMM\_N=10695 \\
                & StarDGEMM\_Gflops=16.3355 \\
                & SingleDGEMM\_Gflops=15.2042 \\
\hline
\end{tabular}
\end{center}
\caption{\label{tab:table-name}HPCC DGEMM.}
\end{table}

Table ?? requires some interpretation. For the single-threaded serial versions of the OpenBLAS and BLIS libraries, the cluster consists of 32 processing cores, so the \verb|SingleDGEMM_Gflops| are per core. For the multi-threaded OpenMP versions of the libraries, the cluster consists of 8 processing nodes, so the \verb|SingleDGEMM_Gflops| are per node.

The results are consistent with the HPL benchmarks, which spend 96\%+ of the benchmark time in the BLAS \verb|dgemm| subroutine. 


%
% SUB SECTION
%
\subsection{STREAM}

\begin{table}[H]
\begin{center}
\begin{tabular}{ |l|l| } 
\hline
                & Results \\ 
\hline
OpenBLAS Serial & STREAM\_VectorSize=28514400 \\
                & STREAM\_Threads=1 \\
                & StarSTREAM\_Copy=0.92926 \\
                & StarSTREAM\_Scale=0.979969 \\
                & StarSTREAM\_Add=0.902324 \\
                & StarSTREAM\_Triad=0.899619 \\
                & SingleSTREAM\_Copy=5.36868 \\
                & SingleSTREAM\_Scale=5.41684 \\
                & SingleSTREAM\_Add=4.75638 \\
                & SingleSTREAM\_Triad=4.75692 \\
\hline
OpenBLAS OpenMP & STREAM\_VectorSize=114232066 \\
                & STREAM\_Threads=1 \\
                & StarSTREAM\_Copy=4.76068 \\
                & StarSTREAM\_Scale=5.44287 \\
                & StarSTREAM\_Add=4.51713 \\
                & StarSTREAM\_Triad=4.53621 \\
                & SingleSTREAM\_Copy=5.47035 \\
                & SingleSTREAM\_Scale=5.46963 \\
                & SingleSTREAM\_Add=4.87128 \\
                & SingleSTREAM\_Triad=4.89569 \\
\hline
BLIS Serial     & STREAM\_VectorSize=28619136 \\
                & STREAM\_Threads=1 \\
                & StarSTREAM\_Copy=0.943137 \\
                & StarSTREAM\_Scale=0.989024 \\
                & StarSTREAM\_Add=0.910843 \\
                & StarSTREAM\_Triad=0.909211 \\
                & SingleSTREAM\_Copy=4.72341 \\
                & SingleSTREAM\_Scale=4.21768 \\
                & SingleSTREAM\_Add=3.90016 \\
                & SingleSTREAM\_Triad=3.94385 \\
\hline
BLIS OpenMP     & STREAM\_VectorSize=114406666 \\
                & STREAM\_Threads=1 \\
                & StarSTREAM\_Copy=5.05861 \\
                & StarSTREAM\_Scale=5.39591 \\
                & StarSTREAM\_Add=4.66044 \\
                & StarSTREAM\_Triad=4.6751 \\
                & SingleSTREAM\_Copy=5.41884 \\
                & SingleSTREAM\_Scale=5.45544 \\
                & SingleSTREAM\_Add=4.80613 \\
                & SingleSTREAM\_Triad=4.81397 \\
\hline
\end{tabular}
\end{center}
\caption{\label{tab:table-name}HPCC STREAM.}
\end{table}


%
% SUB SECTION
%
\subsection{PTRANS}

\begin{table}[H]
\begin{center}
\begin{tabular}{ |l|l| } 
\hline
                & Results \\
\hline
OpenBLAS Serial & PTRANS\_GBs=0.465891 \\
                & PTRANS\_n=26160 \\ 
                & PTRANS\_nb=120 \\
                & PTRANS\_nprow=1 \\
                & PTRANS\_npcol=32 \\
\hline
OpenBLAS OpenMP & PTRANS\_GBs=0.616885 \\
                & PTRANS\_n=26180 \\
                & PTRANS\_nb=88 \\
                & PTRANS\_nprow=2 \\
                & PTRANS\_npcol=4 \\
\hline
BLIS Serial     & PTRANS\_GBs=0.483766 \\
                & PTRANS\_n=26208 \\
                & PTRANS\_nb=168 \\
                & PTRANS\_nprow=1 \\
                & PTRANS\_npcol=32 \\
\hline
BLIS OpenMP     & PTRANS\_GBs=0.637484 \\
                & PTRANS\_n=26200 \\
                & PTRANS\_nb=200 \\
                & PTRANS\_nprow=2 \\
                & PTRANS\_npcol=4 \\
\hline
\end{tabular}
\end{center}
\caption{\label{tab:table-name}HPCC PTRANS.}
\end{table}


%
% SUB SECTION
%
\subsection{Random Access}

\begin{table}[H]
\begin{center}
\begin{tabular}{ |l|l| } 
\hline
                & Results \\ 
\hline
OpenBLAS Serial & MPIRandomAccess\_LCG\_N=2147483648 \\
                & MPIRandomAccess\_LCG\_GUPs=0.000642364 \\
                \cline{2-2} 
                & MPIRandomAccess\_N=2147483648 \\
                & MPIRandomAccess\_GUPs=0.000645338 \\
                \cline{2-2} 
                & RandomAccess\_LCG\_N=67108864 \\
                & StarRandomAccess\_LCG\_GUPs=0.00373175 \\
                & SingleRandomAccess\_LCG\_GUPs=0.00815537 \\
                \cline{2-2} 
                & RandomAccess\_N=67108864 \\
                & StarRandomAccess\_GUPs=0.00373372 \\
                & SingleRandomAccess\_GUPs=0.00837312 \\ 
\hline
OpenBLAS OpenMP & MPIRandomAccess\_LCG\_N=2147483648 \\
                & MPIRandomAccess\_LCG\_GUPs=0.000473649 \\
                \cline{2-2} 
                & MPIRandomAccess\_N=2147483648 \\
                & MPIRandomAccess\_GUPs=0.000477404 \\
                \cline{2-2} 
                & RandomAccess\_LCG\_N=268435456 \\
                & StarRandomAccess\_LCG\_GUPs=0.00580416 \\
                & SingleRandomAccess\_LCG\_GUPs=0.00582959 \\
                \cline{2-2} 
                & RandomAccess\_N=268435456 \\
                & StarRandomAccess\_GUPs=0.00614337 \\
                & SingleRandomAccess\_GUPs=0.00613214 \\
\hline
BLIS Serial     & MPIRandomAccess\_LCG\_N=2147483648 \\
                & MPIRandomAccess\_LCG\_GUPs=0.000644523 \\
                \cline{2-2} 
                & MPIRandomAccess\_N=2147483648 \\
                & MPIRandomAccess\_GUPs=0.00064675 \\
                \cline{2-2} 
                & RandomAccess\_LCG\_N=67108864 \\
                & StarRandomAccess\_LCG\_GUPs=0.00374527 \\
                & SingleRandomAccess\_LCG\_GUPs=0.00835127 \\
                \cline{2-2} 
                & RandomAccess\_N=67108864 \\
                & StarRandomAccess\_GUPs=0.00374741 \\
                & SingleRandomAccess\_GUPs=0.00820883 \\
\hline
BLIS OpenMP     & MPIRandomAccess\_LCG\_N=2147483648 \\
                & MPIRandomAccess\_LCG\_GUPs=0.000475485 \\
                \cline{2-2} 
                & MPIRandomAccess\_N=2147483648 \\
                & MPIRandomAccess\_GUPs=0.000476047 \\
                \cline{2-2} 
                & RandomAccess\_LCG\_N=268435456 \\
                & StarRandomAccess\_LCG\_GUPs=0.00580705 \\
                & SingleRandomAccess\_LCG\_GUPs=0.00578222 \\
                \cline{2-2} 
                & RandomAccess\_N=268435456 \\
                & StarRandomAccess\_GUPs=0.00614596 \\
                & SingleRandomAccess\_GUPs=0.00613275 \\
\hline
\end{tabular}
\end{center}
\caption{\label{tab:table-name}HPCC Random Access.}
\end{table}


%
% SUB SECTION
%
\subsection{FFT}


%
% SUB SECTION
%
\subsection{Network Bandwidth and Latency}



%
% SECTION
%
\section{HPCG Baseline}

The June 2020 HPCG List ranks 169 computer in order of conjugate gradient performance. Ranking number 1 in the list is the Fugaku supercomputer. And ranking 169 is the Spaceborne Computer, which is onboard the International Space Station. The Spaceborne Computer is a 32 core system based on the Intel Xeon E5-2620 v4 8 Core CPU clocked at 2.1GHz, with an Infiniband interconnect. The HPCG List performance results of these two computers are in Table ??.

\begin{table}[H]
\begin{center}
\begin{tabular}{ |c|c|c|c|c|c|c| } 
\hline
HPCG & Name & Cores & HPL $R_{max}$ & TOP500 & HPCG   & Fraction \\
Rank &      &       & Pflops        & Rank   & Pflops & of Peak  \\
\hline
1 & Fugaku & 6,635,520 & 415.530 & 1 & 13.366 & 2.6\% \\
\hline
169 & Spaceborne Computer & 32 & 0.001 & - & 0.000034 & 2.9\% \\
\hline
\end{tabular}
\end{center}
\caption{\label{tab:table-name}Extract from June 2020 HPCG List.}
\end{table}

For computers to be officially ranked in the HPCG List, the HPCG benchmark must be run for in excess of 30 minutes using at least 25\% of available memory.

The following results were obtained for the Aerin Cluster running the HPCG benchmark for 60 minutes using 75\% of available memory and using all 8 nodes.

\begin{table}[H]
\begin{center}
\begin{tabular}{ |c|c|c|c|c|c|c| }
\hline
HPCG & Name & Cores & HPL $R_{max}$ & TOP500 & HPCG   & Fraction \\
Rank &      &       & Pflops        & Rank   & Pflops & of Peak  \\
\hline
- & OpenBLAS Serial & 32 & & - & & \% \\
\hline
- & OpenBLAS OpenMP & 32 & & - & & \% \\
\hline
- & BLIS Serial & 32 & & - & & \% \\
\hline
- & BLIS OpenMP & 32 & & - & & \% \\
\hline
\end{tabular}
\end{center}
\caption{\label{tab:table-name}The Aerin Cluster HPCG Benchmark.}
\end{table}


%
% SECTION
%
\section{Optimisations}


%
% SUB SECTION
%
\subsection{Kernel Preemption Model}

The Linux kernel has 3 Preemption Models, as discussed in Chapter 2:

\begin{itemize}
\item Preemptive
\item Voluntary Preemption
\item No Forced Preemption
\end{itemize}

For scientific computing workloads the No Forced Preemption model should be used, as suggested by the help accompanying the Linux kernel configuration utility:

\say{This is the traditional Linux preemption model, geared towards throughput. It will still provide good latencies most of the time, but there are no guarantees and occasional longer delays are possible. Select this option if you are building a kernel for a server or scientific/computation system, or if you want to maximise the raw processing power of the kernel, irrespective of scheduling latencies.}

The kernel installed by Ubuntu 20.04 LTS 64-bit uses the Voluntary Preemption model. To use the No Forced Preemption model the Linux kernel must be recompiled. Detailed instructions on how to do this are included in the \emph{Kernel Build With No Forced Preemption} project repository wiki page.

HPL was run using all 8 nodes with a No Forced Preemption model kernel. The results are tabulated and compared with the default kernel in Table ??.

\begin{table}[H]
\begin{center}
\begin{tabular}{ |c|c|c|c|c|c|c| } 
\hline
\multicolumn{5}{|c}{} & \multicolumn{2}{|c|}{$\mathbf{R_{max}}$} \\
\hline
\textbf{BLAS} & \textbf{N} & \textbf{NB} & \textbf{P} & \textbf{Q} & \textbf{Voluntary}  & \textbf{No Forced}  \\
              &            &             &            &            & \textbf{Preemption} & \textbf{Preemption} \\
\hline
OpenBLAS & 52320 & 120 & 1 & 32 & 8.6787e+01 & 8.5741e+01 \\
Serial   & 52320 & 120 & 2 & 16 & 9.5517e+01 & 9.2858e+01 \\
         & 52320 & 120 & 4 &  8 & 9.5525e+01 & 9.3784e+01 \\
\hline
OpenBLAS & 52360 &  88 & 1 &  8 & 8.8626e+01 & 8.6306e+01 \\
OpenMP   & 52360 &  88 & 2 &  4 & 9.6630e+01 & 9.4254e+01 \\
\hline
BLIS     & 52416 & 168 & 1 & 32 & 9.1148e+01 & 8.9747e+01 \\
Serial   & 52416 & 168 & 2 & 16 & 1.0341e+02 & 1.0114e+02 \\
         & 52416 & 168 & 4 &  8 & 1.0190e+02 & 1.0183e+02 \\
\hline
BLIS     & 52400 & 200 & 1 &  8 & 9.1180e+01 & 7.4336e+01 \\
OpenMP   & 52400 & 200 & 2 &  4 & 1.0041e+02 & 7.8256e+01 \\
\hline
\end{tabular}
\end{center}
\caption{\label{tab:table-name}Comparison of HPL $R_{max}$ with the a No Forced Preemption kernel and the default Voluntary Preemption kernel.}
\end{table}

Observations and discussion...

BLIS OpenMP...!

\begin{table}[H]
\begin{center}
\begin{tabular}{ |c|c|c|c|c|c|c| } 
\hline
\multicolumn{5}{|c}{} & \multicolumn{2}{|c|}{$\mathbf{R_{max}}$} \\
\hline
\textbf{BLAS} & \textbf{N} & \textbf{NB} & \textbf{P} & \textbf{Q} & \textbf{Voluntary}  & \textbf{No Forced}  \\
              &            &             &            &            & \textbf{Preemption} & \textbf{Preemption} \\
\hline
BLIS     & 52416 & 168 & 1 &  8 & 9.1180e+01 & 9.4440e+01 \\
OpenMP   & 52416 & 168 & 2 &  4 & 1.0041e+02 & 1.0728e+02 \\
\hline
\end{tabular}
\end{center}
\caption{\label{tab:table-name}A repeat of the previous results for BLIS OpenMP but with NB reduced to 168.}
\end{table}

SIGNIFICANCE OF THIS...


%
% SUB SECTION
%
\subsection{Interrupt Coalecsing}

HPL was run using all 8 nodes with Adaptive RX interrupt coalescing enabled. The results are tabulated and compared with the default no interrupt coalescing in Table ??.

\begin{table}
\begin{center}
\begin{tabular}{ |c|c|c|c|c|c|c| } 
\hline
\multicolumn{5}{|c}{} & \multicolumn{2}{|c|}{$\mathbf{R_{max}}$} \\
\hline
\textbf{BLAS} & \textbf{N} & \textbf{NB} & \textbf{P} & \textbf{Q} & \textbf{No Interrupt} & \textbf{RX Interrupt} \\
              &            &             &            &            & \textbf{Coalescing}   & \textbf{Coalescing} \\
\hline
OpenBLAS & 52320 & 120 & 1 & 32 & 8.6787e+01 & \\
Serial   & 52320 & 120 & 2 & 16 & 9.5517e+01 & \\
         & 52320 & 120 & 4 &  8 & 9.5525e+01 & \\
\hline
OpenBLAS & 52360 &  88 & 1 &  8 & 8.8626e+01 & \\
OpenMP   & 52360 &  88 & 2 &  4 & 9.6630e+01 & \\
\hline
BLIS     & 52416 & 168 & 1 & 32 & 9.1148e+01 & \\
Serial   & 52416 & 168 & 2 & 16 & 1.0341e+02 & \\
         & 52416 & 168 & 4 &  8 & 1.0190e+02 & \\
\hline
BLIS     & 52400 & 200 & 1 &  8 & 9.1180e+01 & \\
OpenMP   & 52400 & 200 & 2 &  4 & 1.0041e+02 & \\
\hline
\end{tabular}
\end{center}
\caption{\label{tab:table-name}Comparison of HPL $R_{max}$ with Adaptive RX interrupt coalescing enabled and the default no interrupt coalescing.}
\end{table}

Observations and discussion...


%
% SUB SECTION
%
\subsection{Receive Packet Steering and Receive Flow Steering}

HPL was run using all 8 nodes with Receive Packet Steering (RPS) and Receive Flow Steering (RFS) enabled. The results are tabulated and compared with the default RPS and RFS disabled in Table ??.

\begin{table}
\begin{center}
\begin{tabular}{ |c|c|c|c|c|c|c| } 
\hline
\multicolumn{5}{|c}{} & \multicolumn{2}{|c|}{$\mathbf{R_{max}}$} \\
\hline
\textbf{BLAS} & \textbf{N} & \textbf{NB} & \textbf{P} & \textbf{Q} & \textbf{RPS \& RFS} & \textbf{RPS \& RFS} \\
              &            &             &            &            & \textbf{Disabled}   & \textbf{Enabled} \\
\hline
OpenBLAS & 52320 & 120 & 1 & 32 & 8.6787e+01 & \\
Serial   & 52320 & 120 & 2 & 16 & 9.5517e+01 & \\
         & 52320 & 120 & 4 &  8 & 9.5525e+01 & \\
\hline
OpenBLAS & 52360 &  88 & 1 &  8 & 8.8626e+01 & \\
OpenMP   & 52360 &  88 & 2 &  4 & 9.6630e+01 & \\
\hline
BLIS     & 52416 & 168 & 1 & 32 & 9.1148e+01 & \\
Serial   & 52416 & 168 & 2 & 16 & 1.0341e+02 & \\
         & 52416 & 168 & 4 &  8 & 1.0190e+02 & \\
\hline
BLIS     & 52400 & 200 & 1 &  8 & 9.1180e+01 & \\
OpenMP   & 52400 & 200 & 2 &  4 & 1.0041e+02 & \\
\hline
\end{tabular}
\end{center}
\caption{\label{tab:table-name}Comparison of HPL $R_{max}$ with Receive Packet Steering (RPS) and Receive Flow Steering (RFS) enabled and the default RPS and RFS disabled.}
\end{table}

Observations and discussion...



%
% SUB SECTION
%
\subsubsection{?}

\lstset{style=type}
\begin{lstlisting}
$ sudo perf record mpirun -allow-run-as-root -np 4 xhpl
\end{lstlisting}



Running xhpl on 8 nodes using OpenBLAS...

\lstset{style=type}
\begin{lstlisting}
$ mpirun -host node1:4 ... node8:4 -np 32 xhpl
\end{lstlisting}


SHORTLY AFTER PROGRAM START...

On node1,... where we initiated...

top...

\lstset{style=type}
\begin{lstlisting}
top - 20:33:15 up 8 days,  6:02,  1 user,  load average: 4.02, 4.03, 4.00
Tasks: 140 total,   5 running, 135 sleeping,   0 stopped,   0 zombie
%Cpu(s): 72.5 us, 21.7 sy,  0.0 ni,  0.0 id,  0.0 wa,  0.0 hi,  5.8 si,  0.0 st
MiB Mem :   3793.3 total,    330.1 free,   3034.9 used,    428.3 buff/cache
MiB Swap:      0.0 total,      0.0 free,      0.0 used.    698.7 avail Mem 

    PID USER      PR  NI    VIRT    RES    SHR S  %CPU  %MEM     TIME+ COMMAND                                                   
  34884 john      20   0  932964 732156   7980 R 100.3  18.8 106:40.29 xhpl                                                      
  34881 john      20   0  933692 732272   7916 R 100.0  18.9 107:29.75 xhpl                                                      
  34883 john      20   0  932932 731720   8136 R  99.3  18.8 107:33.25 xhpl                                                      
  34882 john      20   0  932932 731784   8208 R  97.7  18.8 107:33.64 xhpl                                                      
\end{lstlisting}

SOFTIRQS...


NODE 2 - 2 NODES ONLY TO SEE EFFECT...

IPERF!!!

On node8, running the top command...

\lstset{style=type}
\begin{lstlisting}
$ top
\end{lstlisting}

We can see...

\lstset{style=type}
\begin{lstlisting}
top - 18:58:44 up 8 days,  4:29,  1 user,  load average: 4.00, 3.75, 2.35
Tasks: 133 total,   5 running, 128 sleeping,   0 stopped,   0 zombie
%Cpu(s): 50.7 us, 47.8 sy,  0.0 ni,  0.0 id,  0.0 wa,  0.0 hi,  1.4 si,  0.0 st
MiB Mem :   3793.3 total,    392.7 free,   2832.6 used,    568.0 buff/cache
MiB Swap:      0.0 total,      0.0 free,      0.0 used.    901.1 avail Mem 

    PID USER      PR  NI    VIRT    RES    SHR S  %CPU  %MEM     TIME+ COMMAND                                                   
  23928 john      20   0  883880 682456   8200 R 100.0  17.6  13:14.17 xhpl                                                      
  23927 john      20   0  883988 682432   7932 R  99.7  17.6  13:12.58 xhpl                                                      
  23930 john      20   0  883912 682664   7832 R  99.7  17.6  13:17.01 xhpl                                                      
  23929 john      20   0  883880 682640   8376 R  99.3  17.6  13:16.25 xhpl  
\end{lstlisting}

Indicates that only 50.7\% of CPU time is being utilised by user programs (us), Linpack/OpenMPI...

I hypothesise that the 1.4\% of software interrupts (si) is responsible 47.8\% of CPU time in the kernel (sy) servicing these interupts...

Lets have a look at the software interrupts on the system...

\lstset{style=type}
\begin{lstlisting}
$ watch -n 1 cat /proc/softirqs
\end{lstlisting}


\lstset{style=type}
\begin{lstlisting}
Every 1.0s: cat /proc/softirqs

                    CPU0       CPU1       CPU2       CPU3
          HI:          0          1          0          1
       TIMER:  122234556   86872295   85904119   85646345
      NET_TX:  222717797     228381     147690     144396
      NET_RX: 1505715680       1132       1294       1048
       BLOCK:      63160      11906      13148      11223
    IRQ_POLL:          0          0          0          0
     TASKLET:   58902273         33          2          6
       SCHED:    3239933    3988327    2243001    2084571
     HRTIMER:       8116         55         53         50
         RCU:    6277982    4069531    4080009    3994395
\end{lstlisting}

As can be seen...

1. the majority of software interrupts are being generated by network receive (NET\_RX) activity, followed by network transmit activity (NET\_TX)...

2. these interrupts are being almost exclusively handled by CPU0...

What is there to be done?...

1. Reduce the numbers of interrupts...

1.1 Each packet produces an interrupt - interrupt coalesing...

1.2 Reduce the number of packets - increase MTU...

2.1 Share the interrupt servicing activity evenly across the CPUs...


%
% Network Optimisation
%
\subsection{Network Optimisation}

On node2 start the Iperf server...

\lstset{style=type}
\begin{lstlisting}
$ iperf -s
\end{lstlisting}

On node1 start the Iperf client...

\lstset{style=type}
\begin{lstlisting}
$ iperf -c
\end{lstlisting}

ping tests of MTU...




iperf network speed...





\subsubsection{Jumbo Frames}

Requires a network switch capable of Jumbo frames...


\lstset{style=type}
\begin{lstlisting}
$ ip link show eth0
\end{lstlisting}


\lstset{style=type}
\begin{lstlisting}
2: eth0: <BROADCAST,MULTICAST,UP,LOWER_UP> mtu 1500 qdisc mq state UP mode DEFAULT group default qlen 1000
    link/ether dc:a6:32:60:7b:cd brd ff:ff:ff:ff:ff:ff
\end{lstlisting}



\lstset{style=type}
\begin{lstlisting}
$ ping -c 1 -s 1500 -M do node2
\end{lstlisting}

\lstset{style=type}
\begin{lstlisting}
PING node2 (192.168.0.2) 1500(1528) bytes of data.
ping: local error: message too long, mtu=1500
\end{lstlisting}


\lstset{style=type}
\begin{lstlisting}
$ ping -c 1 -s 1472 -M do node2
\end{lstlisting}


\lstset{style=type}
\begin{lstlisting}
PING node2 (192.168.0.2) 1472(1500) bytes of data.
1480 bytes from node2 (192.168.0.2): icmp_seq=1 ttl=64 time=0.392 ms
\end{lstlisting}


Trying to set the MTU to 9000 bytes...

\lstset{style=type}
\begin{lstlisting}
$ sudo ip link set eth0 mtu 9000 
\end{lstlisting}

... results with...

\lstset{style=type}
\begin{lstlisting}
Error: mtu greater than device maximum.
\end{lstlisting}

In fact, attempting to set the MTU to anything greater than 1500 bytes...

\lstset{style=type}
\begin{lstlisting}
$ sudo ip link set eth0 mtu 1501 
\end{lstlisting}

... results with...

\lstset{style=type}
\begin{lstlisting}
Error: mtu greater than device maximum.
\end{lstlisting}


Need to build a kernel with Jumbo frame support...

See Appendix ?...

\lstset{style=type}
\begin{lstlisting}
$ ip link show eth0
\end{lstlisting}

\lstset{style=type}
\begin{lstlisting}
2: eth0: <BROADCAST,MULTICAST,UP,LOWER_UP> mtu 9000 qdisc mq state UP mode DEFAULT group default qlen 1000
    link/ether dc:a6:32:60:7b:cd brd ff:ff:ff:ff:ff:ff
\end{lstlisting}

\lstset{style=type}
\begin{lstlisting}
$ ping -c 1 -s 9000 -M do node2
\end{lstlisting}

\lstset{style=type}
\begin{lstlisting}
PING node2 (192.168.0.2) 9000(9028) bytes of data.
ping: local error: message too long, mtu=9000
\end{lstlisting}

\lstset{style=type}
\begin{lstlisting}
$ ping -c 1 -s 8972 -M do node2
\end{lstlisting}

\lstset{style=type}
\begin{lstlisting}
PING node2 (192.168.0.2) 8972(9000) bytes of data.
8980 bytes from node2 (192.168.0.2): icmp_seq=1 ttl=64 time=0.847 ms
\end{lstlisting}


On \verb|node2| create the \verb|Iperf| server...

\lstset{style=type}
\begin{lstlisting}
$ iperf -s
\end{lstlisting}

On \verb|node1| create and run the \verb|Iperf| client...

\lstset{style=type}
\begin{lstlisting}
$ iperf -i 1 -c node2
\end{lstlisting}

\lstset{style=type}
\begin{lstlisting}
------------------------------------------------------------
Client connecting to node2, TCP port 5001
TCP window size:  682 KByte (default)
------------------------------------------------------------
[  3] local 192.168.0.1 port 46216 connected with 192.168.0.2 port 5001
[ ID] Interval       Transfer     Bandwidth
[  3]  0.0-10.0 sec  1.15 GBytes   991 Mbits/sec
\end{lstlisting}


\begin{figure}
	\centering	
	\includegraphics[width=1.0\textwidth]{bandwidth_vs_mtu.pdf}
	\caption{Network Node to Node Bandwidth vs MTU.}
\end{figure}


\subsection{Kernel TCP Parameters Tuning}

REFERENCE...

https://www.open-mpi.org/faq/?category=tcp

\lstset{style=listing}
\begin{lstlisting}[caption=/etc/sysctl.d/picluster.conf]
net.core.rmem_max = 16777216
net.core.wmem_max = 16777216
net.ipv4.tcp_rmem = 4096 87380 16777216
net.ipv4.tcp_wmem = 4096 65536 16777216
net.core.netdev_max_backlog = 30000
net.core.rmem_default = 16777216
net.core.wmem_default = 16777216
net.ipv4.tcp_mem= 16777216 16777216 16777216
net.ipv4.route.flush = 1
\end{lstlisting}


\lstset{style=type}
\begin{lstlisting}
sudo sysctl --system
\end{lstlisting}

or

\lstset{style=type}
\begin{lstlisting}
sudo shutdown -r now
\end{lstlisting}


\lstset{style=type}
\begin{lstlisting}
Aug 11 03:35:40 node5 kernel: [19256.425779] bcmgenet fd580000.ethernet eth0: bcmgenet_xmit: tx ring 1 full when queue 2 awake
\end{lstlisting}





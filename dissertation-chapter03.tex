This chapter initially discusses the \emph{matrix-matrix multiplication} operation, which is of fundamental importance in High Performance Computing, before discussing the mathematical background to the High Performance Linpack (HPL) and HP Conjugate Gradients (HPCG) benchmarks.


%
% SECTION
%
\section{Matrix-Matrix Multiplication}

The \emph{matrix-matrix multiplication} operation is of fundamental importance in HPC applications and benchmarks. As indicated in Figure 3.1, when running HPL on all 8 nodes of the Aerin Cluster, 87.26\% of the run time is spent in the \verb|HPL_dgemm| function which calls the BLAS \verb|dgemm| function. The \verb|dgemm| function name is derived from \emph{\textbf{d}ouble precision \textbf{ge}neral \textbf{m}atrix \textbf{m}ultiplication}. Running on single node, without any networking overhead, this increases to 96+\%.

\begin{figure}[h]
	\centering	
	\includegraphics[width=1.0\textwidth]{screenshot-perf.png}
	\caption{Output of \texttt{sudo perf report} generated from recording the events of an \texttt{xhpl} process using \texttt{sudo perf record -p 9020 -g -- sleep 30}, where \texttt{9020} is the \texttt{xhpl} process identifier.}
\end{figure}

The naive computational complexity of $\mathbf{C} = \mathbf{AB}$, where $\mathbf{A}$, $\mathbf{B}$ and $\mathbf{C}$ $\in \mathbf{R}^{n \times n}$, is $\mathcal{O}(n^3)$. Each \emph{row-column} dot product is $\mathcal{O}(n)$, requiring $n$ multiplications and $n - 1$ additions. There are $n$ of these dot products per row of $\mathbf{A}$ and each column of $\mathbf{B}$, and $\mathbf{A}$ has $n$ rows. In the more general case of $\mathbf{A} \in \mathbf{R}^{m \times k}$ and $\mathbf{B} \in \mathbf{R}^{k \times n}$, the complexity is $\mathcal{O}(mkn).$ 

In 1969, Volker Strassen proved the computational complexity can be reduced to $\mathcal{O}(N^{\log _{2}7+o(1)})\approx \mathcal{O}(N^{2.8074})$. This is achieved by subdividing each matrix into sub-matrices of dimension $2 \times 2$. By doing do, and introducing intermediate matrices, it is possible to reduce the number of multiplications required from 8 to 7 for each sub-matrix multiplication. This is called the \emph{Strassen Algorithm}.

In 1990, Don Coppersmith and Shmuel Winograd further reduced the complexity to $\mathcal{O}(n^{2.375477})$. The \emph{Coppersmith-Winograd Algorithm} was further improved by Andrew Stothers and Virginia Vassilevska Williams, and most recently in 2014 by François Le Gallf, which reduced the complexity to $\mathcal {O}(n^{2.3728639})$.

However, the reductions in complexity beyond the \emph{Strassen Algorithm} also add implementation complexity. The benefits may not outweigh the computational overhead unless the matrices are very large. For this reason the naive and \emph{Strassen Algorithm} are usually implemented in practice.    

The $\mathcal{O}(n^3)$ complexity of \emph{matrix-matrix multiplication} plays a central role in HPC benchmarks because, compared to other matrix operations, it has the highest $\frac{\text{computation}}{\text{memory movement}}$ ratio. This tests data movement from main memory, through the L1/L2/L3 caches, to the processor registers for computational operations. In a computer cluster it also tests network connectivity. It is for this reason that BLAS libraries usually indicate performance, and comparison with alternative libraries, with plots of \verb|dgemm| Gflops versus matrix size. Figure 3.2 is an example performance plot comparing OpenBLAS with Intel MKL.

\begin{figure}[h]
	\centering	
	\includegraphics[width=0.9\textwidth]{openblas-performance.png}
	\caption{OpenBLAS \texttt{dgemm} performance.}
\end{figure}






%
% SECTION
%
\section{High Performance Linpack (HPL)}


%
% SECTION
%
\section{HP Conjugate Gradients (HPCG)}




\documentclass{report}
\usepackage[utf8]{inputenc}
\setlength{\parindent}{0em}
\setlength{\parskip}{1em}
\usepackage[svgnames]{xcolor}
\usepackage{listings}
\usepackage{pdfpages}
\usepackage{float}
\usepackage[]{caption}
\usepackage[]{subcaption}
\usepackage{gensymb}
\usepackage{amsmath}
\usepackage{hyperref}
\usepackage{dirtytalk}

%\usepackage[a4paper, total={5.5in, 9in}]{geometry}

%\definecolor{termcolor}{RGB}{1.0, 0.97, 0.86}
%\definecolor{listingcolor}{RGB}{0.61, 0.87, 1.0}

\lstdefinestyle{type}{
frame=single,
backgroundcolor=\color{Cornsilk},   
basicstyle=\verbatim@font\small,
numbers=none,
tabsize=2,
breaklines=true,
postbreak=\mbox{\textcolor{red}{$\hookrightarrow$}\space}
}

\lstdefinestyle{term}{
frame=single,
backgroundcolor=\color{Ivory},   
basicstyle=\verbatim@font\small,
numbers=none,
tabsize=2,
breaklines=false,
%breaklines=true,
%postbreak=\mbox{\textcolor{red}{$\hookrightarrow$}\space}
}

\lstdefinestyle{hack}{
frame=single,
backgroundcolor=\color{MistyRose},   
basicstyle=\verbatim@font\small,
numbers=none,
tabsize=2,
breaklines=true,
breakindent=0pt,
breakautoindent=true,
postbreak={}
}

\lstdefinestyle{listing}{
frame=single,
backgroundcolor=\color{AliceBlue},   
basicstyle=\verbatim@font\small,
numbers=left,
numberstyle=\tiny,                    
tabsize=2,
breaklines=false
}


\title{MSc Scientific Computing Dissertation\\Benchmarking a Raspberry Pi 4 Cluster}
\author{John Duffy}
\date{September 2020}

\begin{document}

\maketitle
%\begin{titlepage}
   \begin{center}
       \vspace*{1cm}

       \textbf{Benchmarking a Raspberry Pi 4 Cluster}

       \vspace{0.5cm}
        Thesis Subtitle
            
       \vspace{1.5cm}

       \textbf{John Duffy}

       \vfill
            
       A thesis presented for the degree of\\
       Master of Science
            
       \vspace{0.8cm}
     
       %\includegraphics[width=0.4\textwidth]{university}
            
       Department of Physics and Astronomy\\
       UCL\\
       Country\\
       September 2020
            
   \end{center}
\end{titlepage}



%
% ABSTRACT
%
\chapter*{Abstract}
%\input{dissertation-abstract}


%
% DEDICATION
%
\chapter*{Dedication}



%
% DECLARATION
% 
\chapter*{Declaration}
I declare that..


%
% ACKNOWLEDGEMENTS
%
\chapter*{Acknowledgements}
I want to thank...


%
% TOC
%
\tableofcontents


%
% PART I
%
\part{Project Report}


%
% CHAPTER 1
%
\chapter{Introduction}
%
% SECTION
%
\section{Arm}

Since the release of the Acorn Computers Arm1 in 1985, as a second coprocessor for the BBC Micro, through to powering today's fastest supercomputer, the 7,630,848 core \emph{Fugaku} supercomputer \cite{fujitsu-fugaku}, Arm has steadily grown to become a dominant force in the microprocessor industry, with more than 170+ billion Arm-based microprocessors shipped to date \cite{arm-fugaku}.

Famed for power efficiency, which directly equates to battery life, Arm-based microprocessors dominate the mobile device market for phones and tablets. And market segments which have almost exclusively been based upon x86 microprocessors from Intel or AMD are also increasingly turning to Arm. Microsoft's current flagship laptop, the Surface Pro X, released in October 2019, is based on a Microsoft designed Arm-based microprocessor. And Apple announced in June 2020 a roadmap to transition all Apple devices to Apple designed Arm-based microprocessors within 2 years.

When Acorn engineers designed the Arm1, and subsequently the Arm2 for the Acorn Archimedes personal computer, low power consumptions was not the primary design criteria. Their focus was on simplicity of design. Influenced by research projects \cite{risc} at Stanford University and the University of California, Berkeley, their focus was on producing a RISC (Reduced Instruction Set Computer) design. In comparison to contemporary CISC (Complicated Instruction Set Computer) designs, the simplicity of RISC  required fewer transistors, which directly translated to lower power consumption. The RISC design permitted the Arm2 to outperform the Intel 80286, a contemporary CISC microprocessor, whilst using less power. 


%
% SECTION
%
\section{Raspberry Pi}

The Raspberry Pi Foundation, founded in 2009, is a UK based charity whose aim is to "promote the study of computer science and related topics, especially at school level, and to put the fun back into learning computing". Through it's subsidiary, Raspberry Pi (Trading) Ltd, it provides low-cost, high-performance single-board computers called Raspberry Pi's, and free software.

At the heart of every Raspberry Pi is a Broadcom ``System on a Chip'' (SoC). The SoC integrates Arm microprocessor cores with video, audio and Input/Output (IO). The IO includes USB, Ethernet, and General Purpose IO (GPIO) pins for interfacing with devices such as sensors and motors. The SoC is mounted on small form factor circuit board which hosts the memory chip, and video, audio, and IO connectors. A MicroSD card is used to boot the operating system and for permanent storage.

\begin{figure}
	\centering	
	\includegraphics[width=0.9\textwidth]{images/raspberry-pi-4-model-b.jpeg}
	\caption{\textbf{The Raspberry Pi 4 Model B}.}
\end{figure}

Initially released in 2012 as the Raspberry Pi 1, each subsequent model has seen improvements in SoC microprocessor core count or performance, clock speed, connectivity and available memory.

The Raspberry Pi 1 has a single-core 32-bit ARM1176JZF-S based SoC clocked at 700 MHz and 256 MB of RAM. The RAM was increased to 512 MB in 2016.

The Raspberry Pi 2, released in 2015, introduced a quad-core 32-bit Arm Cortex-A7 based SoC clocked at 900 MHz and 1 GB of RAM.

In 2016, the Raspberry Pi 3 was released with a quad-core 64-bit Arm Cortex-A53 based SoC clocked at 1.2 GHz, together with 1 GB of RAM.

The most recent addition to the range, in 2019, is the Raspberry Pi 4, sporting a quad-core 64-bit Cortex-A-72 based SoC clocked at 1.5 GHz. This model is available with 1, 2, 4 and 8 GB of RAM. This model with 4 GB of RAM was used for this project.

\begin{figure}
	\centering	
	\includegraphics[width=0.9\textwidth]{images/raspberry-pi-zero.jpeg}
	\caption{\textbf{The Raspberry Pi Zero}.}
\end{figure}

Since 2012 the official operating system for all Raspberry Pi models has been Raspbian, a Linux operating system based on Debian. Raspbian has recently been renamed Raspberry Pi OS. To support the aims of the Foundation, a number of educational software packages are bundled with Raspberry Pi OS. These include \emph{Wolfram Mathematica}, and a graphical programming environment aimed at young children called \emph{Scratch}.

Python is the official programming language, due to its popularity and ease of use, and the inclusion of an easy to use Python IDE has been a Foundation priority. This is currently \emph{Thonny}. 

Even though the Raspberry Pi 3 introduced a 64-bit microprocessor, Raspberry Pi OS has remained a 32-bit operating system. However, to complement the introduction of the Raspberry Pi 4 with 8 GB of RAM, a 64-bit version is currently in public beta testing.

Raspberry Pi OS is not the only operating system available for the Raspberry Pi. The Raspberry Pi website provides downloads for Raspberry Pi OS, and also NOOBS (New Out of the Box Software), together with a MicroSD card OS image writing tool called Raspberry Pi Imager. NOOBS and Raspberry Pi Imager make it easy to install operating systems such as Ubuntu, RISC OS (the original Acorn Archimedes OS), Windows 10 IoT Core, and more. Ubuntu 20.04 LTS 64-bit, the operating system used for this project, is available for download from the Ubuntu website, and is also available as an install option within Raspberry Pi Imager.

\begin{figure}
	\centering	
	\includegraphics[width=0.9\textwidth]{images/raspberry-pi-compute-module-3.jpeg}
	\caption{\textbf{The Raspberry Pi Compute Module 3+}.}
\end{figure}

Since the release of the Raspberry Pi 1, the Raspberry Pi has been available in a number of model variants and circuit board formats. The Model B of each release is the most powerful variant, and is intended for desktop use. The Model A is a simpler and cheaper variant intended for embedded projects. The models B+ and A+ designate an improvement to the current release hardware. The Raspberry Pi Zero is a tiny, inexpensive variant without most of the external connectors, designed for low power, possibly battery powered, embedded projects. The Raspberry Pi Compute Module is a stripped down version of the Raspberry Pi without any external connectors. This model is aimed at industrial applications and fits into a standard DDR2 SODIMM connector.


%
% SECTION
%
\section{Aims}


%
% SUB SECTION
%
\subsection{Benchmark Performance}

The main aim of this project is to benchmark the performance of an 8 node Raspberry Pi 4 Model B cluster using standard HPC benchmarks. These benchmarks include High Performance Linpack (HPL), HPC Challenge (HPCC) and High Performance Conjugate Gradients (HPCG).

A pure OpenMPI topology was benchmarked, together with a hybrid OpenMPI/OpenMP topology.


%
% SECTION
%
\subsection{Performance Optimisations}

Having determined a Baseline performance benchmark, opportunities for performance optimisations were investigated for a single core, single node and the whole cluster. Network optimisation was also investigated, and proved to be significant factor in overall cluster performance. 


%
% SECTION
%
\subsection{Investigate Gflops/Watt}

The Green500 List ranks computer systems by energy efficiency, Gflops/Watt. In June 2020, ranking Number 1, the most energy-efficient system was the MN-3 by Preferred Networks in Japan, which achieved a record 21.1 Gigaflops/Watt \cite{green500}. Ranking 200 was Archer at the University of Edinburgh, which achieved 0.497 Gflops/Watt \cite{green500}.

The final aim of this project was to investigate where the Aerin cluster might fare in relation to the Green500 List. 


%
% SECTION
%
\section{Project GitHub Repositories}

The project code and benchmark results are hosted in the following GitHub project repository:

\begin{verbatim}
https://github.com/johnduffymsc/picluster
\end{verbatim}

Detailed instructions for building the Aerin Cluster and running the benchmarks are included in the project repository wiki:

\begin{verbatim}
https://github.com/johnduffymsc/picluster/wiki
\end{verbatim}

This dissertation \LaTeX{} and PDF files, and the Jupyter Notebook used to generate the plots, are hosted in the following GitHub repository:

\begin{verbatim}
https://github.com/johnduffymsc/dissertation
\end{verbatim}
 





%
% CHAPTER 2
%
\chapter{Computer Architecture and HPC Benchmarks}
\section{Introduction}

In his 1937 seminal paper "On Computable Numbers, with an Application to the Entscheidungsproblem" Alan Turing imagined a \emph{univeral computing machine} capable of performing any conceivable mathematical operation. Turing proved that by formulating a mathematical problem as an algorithm, consisting of a sequence of numbers and operations on these numbers, on an infinitely long tape, and with operations to move the tape left and right, it was possible to mechanise the computation of any problem. These machines became known as Turing Machines. 

Today's computers are Turing Machines. Turing's original sequence of numbers and operations are now referred to as the data and  instructions contained within a computer program. The infinitely long tape is now referred to as a computer's memory. And the set of instructions which manipulate program data, and which also permit access to the full range of available memory (move the tape left and right), are referred to as a computer's \emph{instruction set}.

High Performance Computing (HPC) is the solving of numerical problems which are beyond the capabilities of desktop and laptop computers in terms of the amount of data to be processed and the speed of computation required. For example, numerical weather forecasting (NWF) uses a grid of 3D positions to model a section of the Earth's atmosphere, and then solves partial differential equations at each of these points to produce forecasts. The processing performance and memory required to model such systems far exceeds that of even a high-end desktop. 

The UK Met Office use a number of grids to model global and and UK weather. The finest UK grid being a 1.5 km spaced 622 x 810 point inner grid, with a 4 km spaced 950 x 1025 outer grid, both with 70 vertical levels. To model the atmosphere on these grids the UK Met Office currently uses three Cray XC40 supercomputers, capable of 14 Petaflops ($10^{15}$ Floating Point Operations per Second), and which contain 460,000 computer cores, 2 Petabytes of memory and 24 Petabytes of storage.

Clearly a single Cray XC40 used for NWF is a somewhat different beast than a single imaginary Turing Machine. Some of the differences obviously relate to the imaginary nature of the Turing Machine, with its infinitely long tape, and some to what it is possible to build within the limits of today's technology. The Cray XC40's 2 Petabytes of memory is large, but not infinite. But possibly the most important differences are architectural. Each Cray XC40 is a massively parallel supercomputer, made up of a large number of individual processing nodes. Each node has a large but finite amount of processing capacity and memory. The problem data and program instructions must be divided up and distributed amongst the nodes. The nodes must be able to communicate in an efficient manner. And opportunities for \emph{parallel} and \emph{concurrent} processing should be exploited to minimise processing time. Each of these differences is a requirement to map HPC workloads onto a real-world machine. And each of these difference introduces a degree of complexity.   

Since the birth of electronic computing, there has always been a need to know long it will take for a computer to perform a particular task. This may be solely related to allocating computer time efficiently, or simply just wanting to know how long a program will take to run. Or, it may be commercially related; even a moderately sized single computer can be a large investment requiring the maximum performance possible for the purchase price. And more recently, the need to know how much processing power per unit of electricity a computer can achieve has become an important metric. This need for information is addressed by using a benchmark.

A benchmark is a standardised measure of performance. In computing terms this is a piece of software which performs a known task, and which tests a particular aspect(s) of computer performance. One aspect may be raw processing performance. High Performance Linpack (HPL) is one such benchmark, which produces a single measure of Floating Point Operations per Second (Flops) for a single, or more commonly, a cluster of computers. To address the complexity of design of modern supercomputers, as discussed above, a number of complementary benchmarks have been introduced, namely HPC Challenge (HPCC) and HP Conjugate Gradient (HPCG). HPC Challenge is a suite of benchmarks which measure processing performance, memory bandwidth, and network latency and bandwidth, to give a broad view of likely real-world application performance. HP Conjugate Gradient is intended to measure the performance of modern HPC workloads.

To put benchmark results into context, and to extrapolate from the results where performance gains might be realised, it is necessary to have an understanding of the main components of a computer and the network connecting a cluster of computers. The following sections of this chapter describe these components and the network in more detail.

Matrix-matrix multiplication plays an important role in computer benchmarking because the multiplication of large matrices tests processing, memory, and network performance. A more detailed discussion of this topic is also included in this chapter. 


%
% SECTION
%
\section{CPU Architecture}

\subsection{Threads}

\subsection{Processes}

\subsection{Context Switch}

\subsection{Concurrency}

\subsection{Parallel Computation}

\subsection{Scheduling}

\subsection{Interrrupts}

\subsection{Kernel Preemption Model}

\subsection{ARM Architecture }

Fugatku...

Numer of Arm-based in Top/Green 500...

48 core workstation...

RISC paper...

RISC/CISC

Load/Store architecture

Simplicity of design...

Transistor count...

Electrical power...

armv7...

armv8... 64-bit

armv8.1...

armv8.2...

SVE...

Fugaku chip...

Fujitsu chip... 


%
% SECTION
%
\section{Main Memory}

Main memory is the largest component of the memory system of a computer. On desktop, laptop and larger computers, the memory chips usually reside on small circuit boards that fit into sockets on the computer mainboard. These can be upgraded in size by the user. On some smaller computers, such as the Raspberry Pi 4, the memory chip is soldered onto the computer circuit board and is not upgradable.

Each memory location contains a byte of data, where a byte is 8 binary bits. Bytes are stored sequentially at an \emph{addresses}, which is a binary number in the range 0 up to the maximum address supported by the system. The maximum address typically aligns with the register size. For example, 64-bit computer has 64-bit registers which can hold an address in the range 0 to $64^2$. This requires a 64-bit physical \emph{address bus} to address each byte of memory. Practical considerations sometimes limit the size of the address bus. The Raspberry Pi 4 is a 64-bit computer but has a 48-bit physical address bus.

Computers systems without an operating system, such as embedded systems, permit direct access to main memory from software. In this case there is a direct mapping between the memory address within a computer program and the physical address in main memory. Most operating systems present an abstracted view of main memory to each program running on the system. This is called \emph{virtual memory}.
 
    
\begin{figure}
	\centering	
	\includegraphics[width=0.9\textwidth]{virtual-memory.pdf}
	\caption{\textbf{Virtual Memory}. Each process has a \emph{virtual address} space mapped to main memory in \emph{pages} by a \emph{page table} which resides in main memory. A smaller page table called the \emph{Translation Lookaside Buffer} (TLB) is a \emph{cache} in close proximity to each core. The TLB enables fast lookup of physical page addresses without resorting to a slower lookup in the main memory page table.}
\end{figure}

\subsection{Virtual Memory}

Virtual memory is the abstracted view of main memory presented to a running program by the operating system. Virtual memory requires both hardware support, through the Memory Management Unit (MMU), and software support by the operating system. Contiguous regions of virtual memory are organised into \emph{pages}, typically 4 KB in size. Each page of virtual memory maps to a page of physical memory through a \emph{page table} which resides in main memory. A smaller page table called the \emph{Translation Lookaside Buffer} (TLB), which is a \emph{cache} in close proximity to each processing core, is discussed later.

There are a number of benefits of implementing virtual memory. One is to permit the use of a smaller amount of physical memory than is actually addressable. In this case, pages currently in use reside in main memory, and pages no longer required are \emph{swapped} to permanent storage to make space for new pages. This illusion of a full amount of addressable main memory is transparent to the user. But the \emph{paging} between main memory and permanent storage is slow, and is therefore not used in HPC applications.

Possibly the most important benefit of using virtual memory is to implement a protection mechanism called \emph{process isolation}. Each running program, or \emph{process}, executes in its own private, virtual address space. This means that it is not possible for a process to overwrite memory in the address space of another process, possibly due a bug in a program. This process isolation in managed by the operating system using virtual memory. It is possible for multiple processes to communicate through \emph{shared memory}, where each process can read and write to the same block of memory, but this requires programs to be specifically written to make use of this mechanism. 

  
\subsection{UMA versus NUMA}

\subsection{Shared Memory}

\subsection{Distributed Memory}


%
% SECTION
%
\section{Caches}

If we imagine Turing's infinitely long tape and the inertia that must be overcome to move such a tape left and right, it would not be too much of a leap of the imagination to propose copying some sequential part of the tape onto a finite, lighter tape which could be moved left and right faster. Then if the data required for the current part of our computation was contained within this faster tape, the computation would be conducted faster. The contents of the finite tape would be refreshed with data from the infinite tape as required, which may be expensive in terms of time. And if the speed at which we can perform operations on the lighter tape began to outpace the speed of movement of the tape, we might propose copying some of the data onto an even shorter, even faster tape.

If we replace speed of tape movement with speed of memory access, then this imaginary situation is analogous to the layering of memory within a real computer system. Main memory access is slow compared to processor computing speed, so main memory is copied into smaller, faster \emph{caches} colocated on the same silicon die as the processing cores. Each level of cache closer to a processing core is smaller but faster than the previous, with the cache closest to the processing core being referred to as Level 1 (L1) cache. A processor may have L1, L2 and L3 caches, the outer cache possibly being shared between a number of processing cores. As we shall discuss later in this chapter, the speed at which program data flows from main memory through the caches to the processing cores is critical for application performance, and considerable care is taken to minimise \emph{cache misses} which require a \emph{cache refresh} from main memory.

Caches... lines

Cache coherency...

TLB...

The MMU has the following features:

48-entry fully-associative L1 instruction TLB.
32-entry fully-associative L1 data TLB for data load and store pipelines.
4-way set-associative 1024-entry L2 TLB in each processor.

The L1 instruction memory system has the following features:

48KB 3-way set-associative instruction cache.
Fixed line length of 64 bytes.

The L1 data memory system has the following features:

32KB 2-way set-associative data cache.
Fixed line length of 64 bytes.

The features of the L2 memory system include:

Configurable L2 cache size of 512KB, 1MB, 2MB and 4MB.
Fixed line length of 64 bytes.
Physically indexed and tagged cache.
16-way set-associative cache structure.

The SCU uses hybrid Modified Exclusive Shared Invalid (MESI) and Modified Owned Exclusive Shared Invalid (MOESI) protocols to maintain coherency between the individual L1 data caches and the L2 cache.

The L2 memory system requires support for inclusion between the L1 data caches and the L2 cache. A line that resides in any of the L1 data caches must also reside in the L2 cache. However, the data can differ between the two caches when the L1 cache line is in a dirty state. If another agent, a core in the cluster or another cluster, accesses this line in the L2 then it knows the line is present in the L1 of a processor and then it queries that core for the most recent data.


%
% SECTION
%
\section{Networking}

\subsection{Ethernet}

\subsection{MTU}

\subsection{Interrupt Coalescing}

\subsection{Receive Side Scaling}

\subsection{Receive Packet Steering}

\subsection{Receive Flow Steering}


%
% SECTION
%
\section{Matrix Multipication}


%
% SECTION
%
\section{Benchmarks}

\subsection{High Performance Linpack}

\subsection{HPC Challenge}

\subsection{HP Conjugate Gradients}


%
% OLD STUFF FROM HERE TO WEED...
% 


%
% SECTION
%
\section{Landscape}

High Performance Linpack (HPL) is the industry standard HPC benchmark and has been for since 1993. It is used by the Top500 and Green500 lists to rank supercomputers in terms of raw performance and performance per Watt, respectively. However, it has been criticised for producing a single number, and not being a true measure of real-world application performance. This has led to the creation of complementary benchmarks, namely HPC Challenge (HPCC) and High Performance Conjugate Gradients (HPCG). These benchmarks measure whole system performance, including processing power, memory bandwidth, and network speed and latency, in relation to standard HPC algorithms such as FFT and CG.

HPL has been the main focus of this project, mainly because it is the industry standard HPC benchmark, but also because tuning performance for HPL will also produce optimum results for HPCC (HPCC includes HPL) and HPCG.

Because BLAS (Basic Linear Algebra Subroutine) library performance and cluster topology, pure OpenMPI and hybrid OpenMPI/OpenMP, have a direct impact on benchmark performance, a discussion of these topics is also included in this chapter. 


A detailed description of each benchmark follows.

%
% SECTION
%
\section{High Performance Linpack (HPL)}

HPL did not begin life as a supercomputer benchmark. LINPACK is a software package for solving Linear Algebra problems. And in 1979 the ``LINPACK Report'' appeared as an appendix to the LINPACK User Manual. It listed the performance of 23 commonly used computers of the time when solving a matrix problem of size 100. The intention was that users could use this data to extrapolate the execution time of their matrix problems.

As technology progressed, LINPACK evolved through LINPACK 100, LINPACK 1000 to HPLinpack, developed for use on parallel computers. High Performance Linpack (HPL) is an implementation of HPLinpack.

In 1993 the Top500 List was created to rank the performance of supercomputers and HPL was used, and still is used, to measure performance and create the rankings.

HPL solves a dense system of equations of the form:

\[A\mathbf{x}=\mathbf{b}\]

HPL generates random data for a problem size N. It then solves the problem using LU decomposition and partial row pivoting.

HPL requires an implementation of MPI (Message Passing Interface) and a BLAS (Basic Linear Algebra Subroutines) library to be installed. For this project, OpenMPI was the MPI implementation used, and OpenBLAS and BLIS were the BLAS libraries used. Both BLAS libraries were used in the single-threaded serial version and also the multi-threaded OpenMP version.

In HPL terminology, $R_{peak}$ is the theoretical maximum performance. And $R_{max}$ is the maximum achieved performance, which will normally be observed using the maximum problem size $N_{max}$.


%
% SUB SECTION
%
\subsection{Determining Input Parameters}

The main parameters which affect benchmark results are the block size NB, the problem size N, and the processor grid dimensions P and Q.

The block size NB is used for two purposes. Firstly, to ``block'' the problem size matrix of dimension N x N into sub-matrices of dimension NB x NB. This is described in more detail in the Section ??. And secondly, as the message size (or multiples of) for distributing data between cluster nodes.

The optimum size for NB is related to the BLAS library \verb|dgemm| \emph{kernel} block size, which is related to CPU register and L1, L2, and L3 (when available) cache sizes. But this is not easily determined as a simple multiple of the \emph{kernel} block size. Some experimentation is required to determine the optimum size for NB.

\href{https://www.netlib.org/benchmark/hpl/faqs.html}{HPL Frequently Asked Questions} suggests NB should be in the range 32 to 256. A smaller size is better for data distribution latency, but may result in data not being available in sufficiently large chunks to be processed efficiently. Too high a value may result in data starvation while nodes wait for data due to network latency.

For this project, HPL benchmarks were run with NB in the range 32 to 256 in order to determine the optimum size for the Aerin Cluster, and for each BLAS library in serial and OpenMP versions. 

For maximum data processing efficiency, and therefore optimum benchmark performance, the problem size N should be as large as possible. This optimises the cluster processing/communications ratio. Optimum efficiency is achieved when the problem size utilises 100\% of memory. But this is never actually achievable, since the operating system and benchmark software require memory to run. \href{https://www.netlib.org/benchmark/hpl/faqs.html}{HPL Frequently Asked Questions} suggests 80\% of total available memory as a good starting point, and this value was used for this project.

For optimum benchmark performance the problem size N needs to be an integer multiple of the block size NB. This ensures every NB x NB sub-matrix is a full sub-matrix of the N x N problem size, i.e. there are no partially full NB x NB sub-matrices at N x N matrix boundaries.

For each value of NB, the following formula is used to determine the problem size N, taking into account 80\% memory usage:

\[N = \left[\left(0.8 \sqrt{\frac{\text{Memory in GB} \times 1024^3}{8}}\right) \div NB\right] \times NB\]

The division by 8 in the inner parenthesis is the size in bytes of a double precision floating point number.


The online tool \href{http://hpl-calculator.sourceforge.net}{HPL Calculator} by Mohammad Sindi automates the process of calculating the problem size N for block sizes NB in the range 96 to 256, and for memory usage 80\% to 100\%.

The values of N determined using HPL Calculator were cross-checked with the formula above.

The processor grid dimensions P and Q represent a P x Q grid of processor cores. For example, the Aerin cluster has 8 nodes, each with 4 cores, giving a total of 32 processor cores. These core can be organised in compute grids of 1 x 32, 2 x 16 and 4 x 8.

The HPL algorithm favours P x Q grids as square as possible, i.e. with P almost equal to Q, but with P smaller than Q. So, for a single node with 4 cores, a processor grid of 1 x 4 gives better benchmark performance than 2 x 2.

If the Aerin Cluster used a high speed interconnect between nodes, such as InfiniBand, as used on large HPC clusters, maximum performance would be expected to be achieved using a processor grid of 4 x 8. This is the ``squarest'' possible P x Q grid using 32 cores whilst maintaining P less than Q. However, as noted in \href{https://www.netlib.org/benchmark/hpl/faqs.html}{HPL Frequently Asked Questions}, Ethernet is not a high speed interconnect. An Ethernet network is simplistically a single wire connecting the nodes, with each node competing (using random transmission times) for access to the wire to transmit data. This physical limitation reduces potential maximum cluster performance, and the maximum achievable performance is seen using a flatter P x Q grid. This proved to be the case, and maximum cluster performance was observed using a processor grid of 2 x 16 for all 8 nodes. This phenomena was also observed using when using less than 8 nodes.


%
% SUB SECTION
%
\subsection{Running HPL}

HPL uses the input file \verb|HPL.dat| to specify the input parameters for each benchmark run.

Each of these has a preceding count parameter (\verb|#|) which specifies the number of each parameter (for example, there may be more than 1 processor grid shape).

The problem size N and block size NB are set as follows, in this case there being a single problem size and a single block size:

\lstset{style=listing}
\begin{lstlisting}[numbers=none, caption=HPL.dat]
1            # of problems sizes (N)
52360        Ns
1            # of NBs
88           NBs
\end{lstlisting}

The processor grid shapre parameters P and Q are set as follows, in this case there being 32 cores/slots with 3 possible grid shapes, 1 x 32, 2 x 16 and 4 x 8:

\lstset{style=listing}
\begin{lstlisting}[numbers=none, caption=HPL.dat]
3            # of process grids (P x Q)
1 2 4        Ps
32 16 8      Qs
\end{lstlisting}

For each benchmark run, the above parameters are set accordingly.

For this project the remaining parameters, which have a lesser effect on benchmark results, were set in accordance with the advice in \href{https://www.netlib.org/benchmark/hpl/tuning.html}{HPL Tuning}, with \verb|swapping threshold| being set to match NB, as follows:

\lstset{style=listing}
\begin{lstlisting}[numbers=none, caption=HPL.dat]
16.0         threshold
1            # of panel fact
1            PFACTs (0=left, 1=Crout, 2=Right)
2            # of recursive stopping criterium
4 8          NBMINs (>= 1)
1            # of panels in recursion
2            NDIVs
1            # of recursive panel fact.
2            RFACTs (0=left, 1=Crout, 2=Right)
2            # of broadcast
1 3          BCASTs (0=1rg,1=1rM,2=2rg,3=2rM,4=Lng,5=LnM)
2            # of lookahead depth
0 1          DEPTHs (>=0)
2            SWAP (0=bin-exch,1=long,2=mix)
88           swapping threshold
0            L1 in (0=transposed,1=no-transposed) form
0            U  in (0=transposed,1=no-transposed) form
1            Equilibration (0=no,1=yes)
8            memory alignment in double (> 0)
\end{lstlisting}

HPL is run using a serial BLAS library as follows, in this case using 2 nodes with 4 cores/slots:

\lstset{style=type}
\begin{lstlisting}
$ mpirun --bind-to core -host node1:4,node2:4 -np 8 xhpl
\end{lstlisting}

HPL is run using a multi-threaded BLAS library as follows, again, in this case using 2 nodes with 4 cores/slots:

\lstset{style=type}
\begin{lstlisting}
$ mpirun --bind-to socket -host node1:1,node2:1 -np 2 -x OMP_NUM_THREADS=4 xhpl
\end{lstlisting}

The results generated by each benchmark run are either printed on \verb|stdout|, \verb|stderr|, or placed in a file, depending on the \verb|HPL.dat| parameter \verb|device out|.

For this project, all benchmark results were placed in a file called \verb|HPL.out| by specifying the filename and setting \verb|device out| to zero, as follows:

\lstset{style=listing}
\begin{lstlisting}[numbers=none, caption=HPL.dat]
HPL.out      output file name (if any)
0            device out (6=stdout,7=stderr,file)
\end{lstlisting}


The \verb|HPL.out| file from each benchmark run was renamed to reflect the N, NB, P and Q parameters used, and also the BLAS library used. For example:

\lstset{style=type}
\begin{lstlisting}
$ mv HPL.out HPL.out.6_node_45320_88_1_6_2_3.openblas_openmp
\end{lstlisting}

Each results file was then stored appropriately in the \verb|picluster/results| directory structure. 


%
% SECTION
%
\section{HPC Challenge (HPCC)}

HPCC is a suite of benchmarks which test different aspects of cluster performance. These benchmarks include tests for processing performance, memory bandwidth, and network bandwidth and latency. HPCC is intended to give a broader view of cluster performance than HPL alone, which should reflect real-world application performance more closely. HPCC includes HPL as one of the suite of benchmarks.

The HPCC suite consists of the following 7 benchmarks, where \emph{single} indicates the benchmark is run a single randomly selected node, \emph{star} indicates the benchmark in run independently on all nodes, and \emph{global} indicates the benchmark is run using all nodes in a coordinated manner. 


%
% SUB SECTION
%
\subsection{HPL}

HPL is a \emph{global} benchmark which solves a dense system of linear equations.


%
% SUB SECTION
%
\subsection{DGEMM}

The DGEMM benchmark tests double precision matrix-matrix multiplication performance in both \emph{single} and \emph{star} modes.

 
%
% SUB SECTION
%
\subsection{STREAM}

The STREAM benchmark tests memory bandwidth, to and from memory, in both \emph{single} and \emph{star} modes.


%
% SUB SECTION
%
\subsection{PTRANS}

PTRANS, Parallel Matrix Transpose, is a \emph{global} benchmark which tests system performance in transposing a large matrix.


%
% SUB SECTION
%
\subsection{RandomAccess}

The RandomAccess benchmark tests the performance of random updates to a large table in memory, in \emph{single}, \emph{star}, and \emph{global} modes.


%
% SUB SECTION
%
\subsection{FFT}

FFT tests the Fast Fourier Transform performance of a large vector, in \emph{single}, \emph{star}, and \emph{global} modes.


%
% SUB SECTION
%
\subsection{Network Bandwidth and Latency}

This benchmark measures network/communications bandwidth and latency in \emph{global} mode.


%
% SUB SECTION
%
\section{Running HPCC}

HPCC is run in the same manner as HPL. An input file \verb|hpccinf.txt| is created, which is of same format as \verb|HPL.dat|, but may contain additional problem sizes and block sizes for the PTRANS benchmark.

HPCC is run using a serial BLAS library as follows, in this case using 2 nodes with 4 cores/slots:

\lstset{style=type}
\begin{lstlisting}
$ mpirun --bind-to core -host node1:4,node2:4 -np 8 hpcc
\end{lstlisting}

HPCC is run using a multi-threaded BLAS library as follows, again, in this case using 2 nodes with 4 cores/slots:

\lstset{style=type}
\begin{lstlisting}
$ mpirun --bind-to socket -host node1:1,node2:1 -np 2 -x OMP_NUM_THREADS=4 hpcc
\end{lstlisting}

The benchmark results are placed in an output file \verb|hpccoutf.txt|.  

For this project each benchmark \verb|hpccoutf.txt| file was renamed to reflect the N, NB, P and Q parameters used, and also the BLAS library used. For example:

\lstset{style=type}
\begin{lstlisting}
$ mv hpccoutf.txt hpccoutf.txt.8_node_52400_200_1_8_2_4.blis_openmp
\end{lstlisting}

Each results file was then stored appropriately in the \verb|picluster/results| directory structure. 


%
% SECTION
%
\section{High Performance Conjugate Gradients (HPCG)}

HPCG is intended to be complementary to HPL, and to incentivise hardware manufacturers to improve computer architectures for modern HPC workloads. 

Quoting the Super Computing 2019 HPCG Handout:

\say{The HPC Conjugate Gradient (HPCG) benchmark uses a preconditioned conjugate gradient (PCG) algorithm to measure the performance of HPC platforms with respect to frequently observed, yet challenging, patterns of execution, memory access, and global communication.}

\say{The PCG implementation uses a regular 27-point stencil discretixation in 3 dimensions of an elliptic partial differential equation (PDE) with zero Dirichlet boundary condition. The 3-D domain is scaled to fill a 3-D virtual process grid of all available MPI process ranks. The CG iteration includes a local and symmetric Gauss-Seidel preconditioner, which computes a forward and a back solve with a triangular matrix. All of these features combined allow HPCG to deliver a more accurate performance metric for modern HPC architectures.}


%
% SUB SECTION
%
\subsection{Running HPCG}

In a similar manner to HPL and HPCC, HPCG uses an input file \verb|hpcg.dat| for benchmark run configuration.

The format of \verb|hpcg.dat| with default parameter values is:

\lstset{style=type}
\begin{lstlisting}
HPCG benchmark input file
Sandia National Laboratories; University of Tennessee, Knoxville
104 104 104
60
\end{lstlisting}

Lines 1 and 2 are comments, line 3 specifies the 3 dimensions of the problem size, and line 4 specifies the run time in seconds. The problem size is per OpenMPI process. For benchmark runs to be submitted for official performance ranking the problem size memory usage must exceed 25\% of available memory and the run time must exceed 30 minutes (1800 seconds).

HPCG can be built in serial and OpenMP versions. See Part II Chapter 10 for details.

For the serial version each node runs 4 \verb|xhpcg| processes, one on each core. Foe the OpenMP version a single \verb|xhpcg| process is run on each node.

The serial version of HPCG is run as follows, in this case on 2 nodes each with 4 cores:

\lstset{style=type}
\begin{lstlisting}
mpirun --bind-by core -host node1:4,node2:4 -np 8 xhpcg
\end{lstlisting}

The OpenMP version of HPCG is run as follows, again, in this case on 2 nodes each with 4 cores:

\lstset{style=type}
\begin{lstlisting}
mpirun --bind-by socket -host node1:1,node2:1 -np 2 -x OMP_NUM_THREADS=4 xhpcg
\end{lstlisting}

The output from each benchmark run is placed is an output file with a name including a timestamp, for example:

\verb|HPCG-Benchmark_3.1_2020-08-30_15-00-29.txt|


%
% SECTION
%
\section{BLAS Libraries}

If we use the Linux \verb|perf| command to sample and record CPU stack traces (via frame pointers) for an \verb|xhpl| process (with Process ID 6595) for 30 seconds:

\lstset{style=type}
\begin{lstlisting}
$ sudo perf record -p 6595 -g -- sleep 30
\end{lstlisting}

And then look at the stack trace report: 

\lstset{style=type}
\begin{lstlisting}
$ sudo perf report
\end{lstlisting}

\lstset{style=term}
\begin{lstlisting}
+ 100.00% 0.00% xhpl xhpl             [.] _start                                                                                        
+ 100.00% 0.00% xhpl libc-2.31.so     [.] __libc_start_main                                                                             
+ 100.00% 0.00% xhpl xhpl             [.] main                                                                                          
+ 100.00% 0.00% xhpl xhpl             [.] HPL_pdtest                                                                                    
+ 100.00% 0.00% xhpl xhpl             [.] HPL_pdgesv                                                                                    
+ 100.00% 0.00% xhpl xhpl             [.] HPL_pdgesv0                                                                                   
+  98.03% 0.00% xhpl xhpl             [.] HPL_pdupdateTT                                                                                
+  97.71% 0.00% xhpl libblas.so.3     [.] 0x0000ffffaa839ff0                                                                            
+  97.71% 0.00% xhpl libgomp.so.1.0.0 [.] GOMP_parallel                                                                                 
+  97.70% 0.00% xhpl libblas.so.3     [.] 0x0000ffffaa839e80                                                                            
-  96.56% 0.00% xhpl xhpl             [.] HPL_dgemm                                                                                     
    HPL_dgemm                                                                                                                                           
    dgemm_            
\end{lstlisting}

It can be seen that 96.56\% of the time within the \verb|xhpl| process is spent in the \verb|HPL_dgemm| function, which subsequently calls the BLAS \verb|dgemm_| function (the \verb|_| appended to \verb|dgemm| function name is the Fortran function name decoration).
 
It is for this reason that the efficiency of the BLAS library is critical for both benchmark and real-world application performance. The efficiency of the \verb|dgemm| (\textbf{d}ouble precision \textbf{ge}neral \textbf{m}atrix \textbf{m}ultiplication) function is particularly important for the dense matrix HPL benchmark.

The mathematical operation implemented by \verb|dgemm| is:

\[C := \alpha \times A \times B + \beta \times C\]

where $A$ is a $M \times K$ matrix, $B$ is a $K \times N$ matrix, $C$ is a $M \times N$ matrix, and $\alpha$ and $\beta$ are scalars.

In the case of the HPL benchmark, $M = N = K$.

Efficient BLAS libraries ``block'' matrix multiplication into smaller sub-matrix multiplications. The ``block'' sizes of these sub-matrices multiplications are carefully chosen to make optimum use of CPU registers, and L1, L2, and L3 (when available) cache sizes. These sub-matrix multiplications are referred to as \emph{kernels}, or sometimes \emph{micro-kernels}.

Maximum performance is achieved when the matrix multiplication problem size in an integer multiple of the \verb|dgemm| ``block'' size. In the case of the HPL benchmark, maximum performance is achieved when the the problem size N is an integer multiple of the HPL NB block size, which in turn is an integer multiple of the BLAS ``block'' size.

The above is depicted in Figure ??.

\begin{figure}
	\centering
	\includegraphics[width=1.0\textwidth]{dgemm.pdf}
	\caption{\texttt{dgemm} \emph{kernel} Matrix-Matrix Multiplication.}
	\label{fig:image2}
\end{figure}


%
% SECTION
%
\subsection{GotoBLAS}

GotoBLAS is a high performance BLAS library developed by Kazushige Goto at the Texas Advanced Computing Center (TACC), a department of the University of Texas at Austin.

GotoBLAS achieves high performance through the use of hand-crafted assembly language \emph{kernels}. Higher level BLAS routines are decomposed in \emph{kernels}, which stream data from the L1 and L2 CPU caches. These kernels typically reflect the size of the CPU registers, and L1 and L2 caches. For example, a CPU architecture may have a 4 x 4 \emph{dgemm kernel} and a 4 x 8 \emph{dgemm kernel} which conduct a double precision matrix-matrix multiplication on 4 x 4 and 4 x 8 matrices, respectively, and which have been sized for a specific architecture.

The source code for GotoBLAS and GotoBLAS2 is still available as Open Source software, but the library is no longer in active development.


%
% SECTION
%
\subsection{OpenBLAS}

OpenBLAS is an Open Source fork of the original GotoBLAS2 library, and is in active development by volunteers led by Zhang Xianyi at the Lab of Parallel Software and Computational Science, Institute of Software, Chinese Academy of Sciences (ISCAS).

OpenBLAS is used by many of the Top500 supercomputers, including the Fugaku supercomputer which tops the June 2020 TOP500 List.

For the Arm64 architecture, OpenBLAS implements the following \verb|dgemm| \emph{kernels}, where \verb|.S| indicates an assembly language file:

\begin{itemize}
  \item dgemm\_kernel\_4x4.S
  \item dgemm\_kernel\_4x8.S
  \item dgemm\_kernel\_8x4.S 
\end{itemize}


%
% SECTION
%
\subsection{BLIS}

The ``BLAS-like Library Instantiation Software'' (BLIS) is a BLAS library implementation for many CPU architectures, and also a framework for implementing new BLAS libraries for new architectures. Using the BLIS framework, by solely implementing an optimised \verb|dgemm| \emph{kernel} in assembly language or compiler intrinsics, BLAS library functionality can be realised which achieves 60\% - 90\% of theoretical maximum performance.

BLIS is developed by the Science of High-Performance Computing (SHPC) group of the Oden Institute for Computational Engineering and Sciences, at The University of Texas at Austin.

For the Arm64 architecture, BLIS implements the following \verb|dgemm| assembly language \emph{kernel}:

\begin{itemize}
  \item gemm\_armv8a\_asm\_6x8
\end{itemize}


%
% SECTION
%
\section{Pure OpenMPI Topology}

In a pure OpenMPI topology, work is distributed across the cluster nodes, and the processor cores on each node, by OpenMPI. Processor cores are referred to as \emph{slots}. The number of nodes in the cluster, and the number of slots on each node, are specified using either the \verb|-host| or \verb|-hostfile| parameter of the \verb|mpirun| command. Each processor slot is a target for a work process. The total number of work processes to be run is specified by the \verb|-np| parameter.

The \verb|-host| parameter is used to specify nodes and slots on the command line.

The \verb|-hostfile| parameter is used to specify a file which contains the nodes and slots information.

In the following example, the \verb|-host| parameter is used to specify 2 nodes, each with 4 slots, on which to run 8 \verb|xhpl| work processes:

\lstset{style=type}
\begin{lstlisting}
$ mpirun --bind-to core -host node1:4,node2:4 -np 8 xhpl
\end{lstlisting}

The same number of nodes, slots and work processes is specified using the \verb|-hostfile| parameter as follows:

\lstset{style=type}
\begin{lstlisting}
$ mpirun --bind-to core -hostfile nodes -np 8 xhpl
\end{lstlisting}

Where \verb|nodes| is a file containing the following:

\lstset{style=listing}
\begin{lstlisting}[numbers=none, caption=nodes]
node1 slots=4
node2 slots=4
\end{lstlisting}

The \verb|--bind-to core| parameter instructs \verb|mpirun| to not migrate a work processes from the core on which it was started. Once started on a specific core, a work process will remain \emph{bound} to that core. This is an optimisation which reduces cache refreshes when a work process is interrupted, by a kernel system call for example, and is then restarted.

A pure OpenMPI distribution of \verb|xhpl| work processes on a single node with 4 cores/slots is depicted in Figure ?? (a). Each \verb|xhpl| process calls the functions of a single-threaded BLAS serial library.


%
% SECTION
%
\section{Hybrid OpenMPI/OpenMP Topology}
In a hybrid OpenMPI/OpenMP topology, OpenMPI is used to distribute work between the nodes. Each node runs a single work process. OpenMP is then used to distribute the work of this single process between the node cores using a multi-threaded BLAS library.

The \verb|-host|, \verb|-hostfile| and \verb|-np| parameters of the \verb|mpirun| command are used in a same manner as the pure OpenMPI case, noting that each node now has 1 slot on which to run a work process.

The additional parameter \verb|-x| is required now required. This parameter distributes and sets the \emph{environmental variable} \verb|OMP_NUM_THREADS| on each node prior to a work process being started on each node. This variable is queried by the multi-threaded BLAS library, and the appropriate number of threads are utilized.

Using 2 nodes, and 4 cores per node, as per the pure OpenMPI example, 2 \verb|xhpl| processes are run, one on each node, with the multi-threaded BLAS library utilising 4 cores, as follows:

\lstset{style=type}
\begin{lstlisting}
$ mpirun --bind-to socket -host node1:1,node2:1 -np 2 -x OMP_NUM_THREADS=4 xhpl
\end{lstlisting}

The \verb|--bind-to socket| parameter indicates to \verb|mpirun| that the \verb|xhpl| process is not associated with a particular core, it is not to be \emph{bound} to a specific core. The OpenMP runtime will determine which core(s) are used to run the \verb|xhpl| process.

Note, without the \verb|--bind-to socket| parameter only a single thread will be utilised for a multi-threaded BLAS library, even if the \verb|OMP_NUM_THREADS| environmental variable is set correctly.

A hybrid OpenMPI/OpenMP distribution of work is depicted in Figure ?? (b). A single \verb|xhpl| process calls functions from a multi-threaded BLAS library which run as threads on the processor cores.

\begin{figure}
	\begin{subfigure}{1.0\textwidth}
		\centering
		\includegraphics[width=0.71\textwidth]{core-pure-openmpi.pdf}
		\caption{Pure OpenMPI}
		\label{fig:subim1}
	\end{subfigure}
	\par\bigskip
	\begin{subfigure}{1.0\textwidth}
		\centering
		\includegraphics[width=0.71\textwidth]{core-hybrid-openmpi-openmp.pdf}
		\caption{Hybrid OpenMPI/OpenMP}
		\label{fig:subim2}
	\end{subfigure}
\caption{Single Node Toplologies.}
\label{fig:image2}
\end{figure}






%
% CHAPTER 3
%
\chapter{Mathematical Background of HPC Benchmarks}


%
% CHAPTER 4
%
\chapter{The Aerin Cluster}
This chapter describes the hardware and software components of the Aerin Cluster. A description of the BLAS (Basic Linear Algebra Subroutines) libraries installed on the cluster is also included, together with a description of pure OpenMPI and hybrid OpenMPI/OpenMP cluster topologies. Finally, a description of Pi Cluster Tools, a suite of scripts which make command line management of the cluster easier and less prone to error is also included. 

Detailed build instructions for the Aerin Cluster, and the implementation details of Pi Cluster Tools, are included in the \verb|picluster| repository wiki. 

\begin{figure}[h]
	\centering	
	\includegraphics[width=0.95\textwidth]{aerin.jpg}
	\caption{\textbf{The Aerin Cluster}.}
\end{figure}



%
% SECTION
%
\section{Hardware}

The Aerin Cluster consists of the following hardware components:

\begin{itemize}
  \item 8 x Raspberry Pi 4 Model B compute nodes, \verb|node1| to \verb|node8|
  \item 1 x Raspberry Pi 4 Model B build node, \verb|node9|
  \item 9 x Official Raspberry Pi 4 power supplies
  \item 9 x Class 10 A1 MicroSD cards
  \item 9 x Heatsinks with integrated fans
  \item 1 x Netgear FVS318G 8 Port Gigabit Router/Firewall
  \item 1 x Netgear GS316 16 Port Gigabit Switch (with Jumbo Frame Support)
  \item Cat 7 cabling
\end{itemize}


%
% SUB SECTION
%
\subsection{Raspberry Pi 4 Model B}
The 9 x Raspberry Pi 4 Model B's used in the cluster are the 4GB RAM version. Recently, an 8GB RAM version became available. This which would be the preferred version for a future cluster.

The cluster compute nodes are \verb|node1| to \verb|node8|. These nodes are used to run benchmarks. Some benchmarks take a substantial amount of time to run, so it is convenient to have a dedicated build node for compiling software, etc, while benchmarks are running. This build node is \verb|node9|.

It is also convenient to have one of the compute nodes designated the ``master'' compute node. This is \verb|node1|. Software which needs to be compiled locally to the compute nodes, and not on the build node, is compiled on the ``master'' node. This node is also used to mirror the GitHub \verb|picluster| repository and to run the tools from the Pi Cluster Tools suite. 


%
% SUB SECTION
%
\subsection{Power Supplies}
The Raspberry Pi 4 is sensitive to voltage drops, especially whilst booting. So it was decided to use 9 \emph{Official Raspberry Pi 4} power supplies, rather than a USB hub with multiple power outlets, which may not have been able to maintain output voltage whilst booting 9 nodes. The 9 power supplies do occupy some space, so a future development would be to investigate a suitably rated USB power hub.


%
% SUB SECTION
%
\subsection{MicroSD Cards}
MicroSD cards are available in a number of speed classes and \emph{use} categories. The recommended minimum specification for the Raspberry Pi 4 is Class 10 A1. The ``10'' refers to a 10 MB/s write speed. And the ``A1'' refers to the ``Application'' category, which supports at least 1500 read operations and 500 write operations per second.


%
% SUB SECTION
%
\subsection{Heatsinks}
Cooling is a major consideration when building any cluster. The Raspberry Pi 4 Model B throttles back the clock speed at approximately 85\degree C, which would not only have had a negative impact on benchmark results, but also on repeatability. So, it was very important to select suitable cooling. After some investigation, it was decided to use heatsinks with integrated fans. These proved to be very successful, with no greater than 65\degree C observed at any time, even with 100\% CPU utilisation for many hours. 


%
% SUB SECTION
%
\subsection{Network Considerations}

The Raspberry Pi 4 Model B has a single Gigabit Ethernet interface. The theoretical maximum bandwidth of this interface is \emph{1 Gigabits per second}. As observed during benchmarking, the Raspberry Pi 4 is capable of utilising almost all of this bandwidth. It is therefore important that all networking equipment and cabling supports Gigabit Ethernet, otherwise the network performance of the cluster would be unnecessarily degraded.


%
% SUB SECTION
%
\subsection{Router/Firewall}
The router/firewall acts as the Aerin Cluster interface to the outside world. One side of the firewall is the cluster LAN (Local Area Network), on which the compute nodes and \verb|node9| are connected. The other side of the firewall is the WAN (Wide Area Network). The firewall only permits specifically configured network packets from the WAN through the firewall to the LAN. The Aerin Cluster is configured to only permit \verb|ssh| packets through the firewall, which are then routed to \verb|node1|.

The router exposes a single IP address to the WAN. Access to the cluster is through this single IP address. In my home environment the WAN is connected to my ADSL router via an Ethernet cable. This permits the compute nodes to connect to the internet and download updates. When relocated to UCL the WAN would be connected to the internal UCL network.

The router also acts as DHCP (Dynamic Host Configuration Protocol) server for the compute node LAN. Compute node hostnames, such as \verb|node1| etc, are configured by a boot script which determines the node hostname from the last octet of the node IP address, served by the DHCP server based on the MAC address. This ensures that each compute node is always assigned the same LAN IP address and hostname across reboots.

The router/firewall is easily configured through a web-based setup. Details on how to do this are included in the \verb|picluster| repository wiki.


%
% SUB SECTION
%
\subsection{Network Switch}

The network switch acts as an extension to the number of Ethernet ports on the compute node LAN. And because it supports Jumbo Frames it can accommodate an MTU increase to 9000 bytes localised to the compute nodes.


%
% SUB SECTION
%
\subsection{Cabling}
Cat 5 network cabling only support 100 Mbit/s. Cat 5e and Cat 6 supports 1 Gbit/s, but not necessarily with electrical shielding. Cat 6a and Cat 7 support 10 Gbit/s with electrical shielding. Therefore, to ensure maximum use of the network capabilities of the Raspberry Pi 4, a minimum of Cat 5e cabling must be used.

The Aerin Cluster uses Cat 7 cabling for optimum network performance.


%
% SECTION
%
\section{Software}

The Aerin Cluster consists of the following software components.


%
% SUB SECTION
%
\subsection{Operating System}
The operating system used for the Aerin Cluster is Ubuntu 20.04 LTS 64-bit Pre-Installed Server for the Raspberry Pi 4. Detailed instructions for installing the operating system are included in the \verb|picluster| repository wiki.


%
% SUB SECTION
%
\subsection{\texttt{cloud-init}}

The \verb|cloud-init| system was originally developed by Ubuntu to simplify the instantiation of operating system images in cloud computing environments, such as Amazon's AWS and Microsoft's Azure. It is now an industry standard. It can also be used for automating the installation of the same operating system on a cluster of computers using a single installation image.

The idea is that a \verb|user-data| file is added to the \verb|boot| directory of an installation image. When a node boots using the image, this file is read and the configuration/actions specified in this file are automatically applied/run as the operating system is installed.

For the Aerin Cluster the following configuration/actions were applied to each node:

\begin{itemize}
\item Add the user \verb|john| to the system and set the initial password 
\item Add \verb|john's| public key
\item Update the \verb|apt| data cache
\item Upgrade the system
\item Install specified software packages
\item Create a \verb|/etc/hosts| file
\item Set the hostname based on the IP address
\end{itemize}

All of the above is done from a single installation image and \verb|user-data| file.

The main software packages installed by \verb|cloud-init| are:

\begin{itemize}
\item build-essential
\item openmpi-bin
\item libopenblas0-serial
\item libopenblas0-openmp
\item libblis3-serial
\item libblis3-openmp
\end{itemize}

The \verb|build-essential| package installs essential software build tools, such as C/C++ compilers and \verb|make|. The \verb|openmpi-bin| package installs the OpenMPI binary and development files. And the OpenBLAS and BLIS libraries install both the serial and OpenMP versions of the respective libraries. 


%
% SUB SECTION
%
\subsection{Benchmark Software}

The HPL, HPCC and HPCG benchmark software is compiled locally from source. The instructions for how to do this are included in the \verb|picluster| repository wiki.


%
% SECTION
%
\section{BLAS Libraries}

%
% SECTION
%
\subsection{GotoBLAS}

GotoBLAS is a high performance BLAS library developed by Kazushige Goto at the Texas Advanced Computing Center (TACC), a department of the University of Texas at Austin.

GotoBLAS achieves high performance through the use of hand-crafted assembly language \emph{kernels}. Higher level BLAS routines are decomposed in \emph{kernels}, which stream data from the L1/L2/L3 CPU caches. These kernels typically reflect the size of the CPU registers, and L1/L2/L3 caches. For example, a CPU architecture may have a 4 x 4 \emph{dgemm kernel} and a 4 x 8 \emph{dgemm kernel} which conduct a double precision matrix-matrix multiplication on 4 x 4 and 4 x 8 matrices, respectively, and which have been sized for a specific architecture.

The source code for GotoBLAS and GotoBLAS2 is still available as Open Source software, but the library is no longer in active development.


%
% SECTION
%
\subsection{OpenBLAS}

OpenBLAS is an Open Source fork of the original GotoBLAS2 library, and is in active development by volunteers led by Zhang Xianyi at the Lab of Parallel Software and Computational Science, Institute of Software, Chinese Academy of Sciences (ISCAS).

OpenBLAS is used by many of the Top500 supercomputers, including the Fugaku supercomputer which tops the June 2020 TOP500 List.

For the Arm64 architecture, OpenBLAS implements the following \verb|dgemm| \emph{kernels}, where \verb|.S| indicates an assembly language file:

\begin{itemize}
  \item dgemm\_kernel\_4x4.S
  \item dgemm\_kernel\_4x8.S
  \item dgemm\_kernel\_8x4.S 
\end{itemize}


%
% SECTION
%
\subsection{BLIS}

The ``BLAS-like Library Instantiation Software'' (BLIS) is a BLAS library implementation for many CPU architectures, and also a framework for implementing new BLAS libraries for new architectures. Using the BLIS framework, by solely implementing an optimised \verb|dgemm| \emph{kernel} in assembly language or compiler intrinsics, BLAS library functionality can be realised which achieves 60\% - 90\% of theoretical maximum performance.

BLIS is developed by the Science of High-Performance Computing (SHPC) group of the Oden Institute for Computational Engineering and Sciences, at The University of Texas at Austin.

For the Arm64 architecture, BLIS implements the following \verb|dgemm| assembly language \emph{kernel}:

\begin{itemize}
  \item gemm\_armv8a\_asm\_6x8
\end{itemize}


%
% SUB SECTION
%
\subsection{Aerin Cluster BLAS Libraries}

To enable comparison between BLAS library implementations, OpenBLAS and BLIS, in both serial and OpenMP versions, are installed on the Aerin Cluster. For benchmark consistency, and repeatability, it is essential that the same BLAS library in use on each node at the same time. Two tools from the Pi Cluster Tools suite, \verb|libblas-query| and \verb|libblas-set|, simplify BLAS library management. 

%
% SECTION
%
\section{Cluster Topologies}

\begin{figure}
	\begin{subfigure}{1.0\textwidth}
		\centering
		\includegraphics[width=0.71\textwidth]{core-pure-openmpi.pdf}
		\caption{Pure OpenMPI}
		\label{fig:subim1}
	\end{subfigure}
	\par\bigskip
	\begin{subfigure}{1.0\textwidth}
		\centering
		\includegraphics[width=0.71\textwidth]{core-hybrid-openmpi-openmp.pdf}
		\caption{Hybrid OpenMPI/OpenMP}
		\label{fig:subim2}
	\end{subfigure}
\caption{\textbf{Single Node Toplologies.}}
\label{fig:image2}
\end{figure}


%
% SUB SECTION
%
\subsection{Pure OpenMPI}

In a pure OpenMPI topology, work is distributed across the processor cores of each node by OpenMPI. Each core runs a single MPI work process. This is depicted in Figure 4.2(a).

A work process can be \emph{bound} to a specific core. This is an optimisation which reduces \emph{cache refreshes} when a work process is interrupted and is then subsequently re-scheduled.

Each core of the Aerin Cluster supports a single thread of execution. Therefore, in this topology, work processes are linked against the serial, single-threaded, versions of the BLAS libraries.


%
% SUB SECTION
%
\subsection{Hybrid OpenMPI/OpenMP}
In a hybrid OpenMPI/OpenMP topology, OpenMPI is used to distribute work between cluster nodes. Each node runs a single MPI work process. OpenMP is then used to distribute the work of this single process between node cores. This is depicted in Figure 4.2(b).

Each node of the Aerin Cluster supports a multiple threads of execution, one on each core. Therefore, in this topology, work processes are linked against the OpenMP, multi-threaded, versions of the BLAS libraries.


%
% SECTION
%
\section{Pi Cluster Tools}

Command line administration of a cluster of computers is repetitive and prone to error. To work around this problem, and to make administration of the Aerin Cluster easier, a selection of \verb|bash| scripts were written and called \emph{Pi Cluster Tools}. Each tool loops over a list of nodes and uses \verb|ssh| to invoke a remote action on each node in turn.

The Pi Cluster Tools scripts should be invoked from \verb|node1|.

Pi Cluster Tools consists of the following tools:

\begin{itemize}
\item upgrade
\item reboot
\item shutdown
\item do
\item libblas-query
\item libblas-set
\item linpack-profiler
\item interrupt-coalescing
\item receive-packet-steering
\item receive-flow-steering
\end{itemize}

Script listings are sample usage are included in the \verb|picluster| repository wiki.



%
% CHAPTER
%
\chapter{Benchmark Results and Optimisations}
%
% SECTION
%
\section{Theoretical Maximum Performance}

The Raspberry Pi 4 Model B is based on the Broadcom BCM2711 System on a Chip (SoC). The BCM2711 includes four Arm Cortex-A72 cores clocked at 1.5 GHz.

Each core of the Arm Cortex-A72 implements the 64-bit Armv8-A ISA (Instruction Set Architecture). This instruction set includes Advanced SIMD instructions which operate on a single 128-bit SIMD pipeline. This pipeline can conduct two 64-bit double precision \emph{floating point operations} per clock cycle.  

A \emph{fused multiply-add} (FMA) instruction implements a \emph{multiplication} followed by an \emph{add} in a single instruction. The main purpose of FMA instructions is to improve result accuracy by conducting a single rounding operation on completion of both the \emph{multiplication} and \emph{add} operations. A single FMA instruction conducts two \emph{floating point operations} per clock cycle. 

The theoretical maximum performance of a single Aerin Cluster node, $R_{peak}$, is therefore:

\begin{align}
R_{peak} &= 4 \textrm{ cores} \times 1.5 \textrm{ GHz} \times 2 \textrm{ doubles} \times 2 \textrm{ FMA}\\
&= 24 \textrm{ Gflops}
\end{align}

This $R_{peak}$ of 24 Gflops is only achievable, continuously, if every instruction in a program is an FMA instruction, and the program data is aligned in memory appropriately for efficient access. This obviously cannot be the case, since a program will consist of at least some instructions to  load data from memory and store results back in memory, and these are not FMA instructions. Therefore, $R_{peak}$ is very much a theoretical maximum performance. 

The theoretical maximum performance of the Aerin Cluster as a whole is therefore:

\begin{align}
R_{peak} &= 8 \textrm{ nodes} \times 24 \textrm{ Gflops}\\
&= 192 \textrm{ Gflops}
\end{align}

For maximum performance, the HPL benchmark requires a problem size which utilises 100\% of memory. But, because the operating system requires memory, this is never fully achievable. 

\begin{figure}[h]
	\centering	
	\includegraphics[width=1.0\textwidth]{screenshot-memory.png}
	\caption{\textbf{Available Memory}. Output from \texttt{dmesg | grep Memory} indicates the memory usage by the Linux kernel, and the memory available to applications and benchmarks}
\end{figure}

As can be seen in Figure 5.1, 3.6 GB of memory (3,783,868 KB) is available to applications and benchmarks per node. This equates to 90\% of the total 4 GB (4,194,304 KB) per node. Any transient use of more than 90\% of memory will result in memory pages being swapped to permanent storage, which will negatively impact benchmark performance.

Therefore, for the HPL baseline benchmarks, 80\% of available memory was chosen for the problem size. This is also the amount suggested as an initial \emph{good guess} in the HPL Frequently Asked Questions \cite{hpl-faq}.

The above necessarily results in the baseline performance only being able to achieve 80\% of $R_{peak}$, at best. This is 4.8 Gflops for a single core, 19.2 Gflops for a single node, and 153.6 Gflops for the 8 node cluster. These values are indicated on the HPL baseline performance plots.


%
% SECTION
%
\section{HPL Baseline}

Detailed instructions on how to install the HPL benchmark software, and how to run the HPL benchmark are included in the project repository wiki.

To establish \emph{baseline} performance, the HPL benchmark was run using the default Ubuntu 20.04 LTS 64-bit packages, and without any system or network tuning.

Baseline performance was investigated for the single core, single node, two node and whole cluster configurations.


%
% SUB SECTION
%
\subsection{HPL 1 Core Baseline}

The purpose of this investigation is to determine the performance of a single core running a single \verb|xhpl| process, with the single core having exclusive access to the shared L2 cache. 

As discussed in the previous section, the HPL problem size is restricted to 80\% of available memory. In the case, 80\% of a single node's 4 GB. Using values of block size NB from 32 to 256, as suggested by HPL Frequently Asked Questions \cite{hpl-faq}, and using equation ?? to ensure the problem size N is an integer multiple of NB, results in Table 5.1 of NB and N combinations.

\begin{table}[H]
\begin{center}
	\begin{tabular}{ |c|c|c|c|c|c|c|c|c|c| } 
		\hline
		\textbf{NB} & \textbf{N} & \textbf{NB} & \textbf{N} & \textbf{NB} & \textbf{N} & \textbf{NB} & \textbf{N} & \textbf{NB} & \textbf{N} \\ 
		\hline
		32 & 18528 &  80 & 18480 & 128 & 18432 & 176 & 18480 & 224 & 18368 \\ 
		40 & 18520 &  88 & 18480 & 136 & 18496 & 184 & 18400 & 232 & 18328 \\ 
 		48 & 18528 &  96 & 18528 & 144 & 18432 & 192 & 18432 & 240 & 18480 \\
		56 & 18536 & 104 & 18512 & 152 & 18392 & 200 & 18400 & 248 & 18352 \\ 
 		64 & 18496 & 112 & 18480 & 160 & 18400 & 208 & 18512 & 256 & 18432 \\
		72 & 18504 & 120 & 18480 & 168 & 18480 & 216 & 18360 &   - &     - \\ 
 		\hline
	\end{tabular}
\end{center}
\caption{\label{tab:table-name}\textbf{1 Core NB and N Combinations.} Block size NB and problem size N combinations for NB between 32 and 256 using 80\% of 4 GB of memory.}
\end{table}

The benchmark results are plotted in Figure 5.2.

\begin{figure}[H]
	\begin{subfigure}{1.0\textwidth}
		\centering
		\includegraphics[width=0.85\textwidth]{openmpi_gflops_vs_nb_1_core_80_percent_memory.pdf}
		\caption{Pure OpenMPI}
		\label{fig:subim1}
	\end{subfigure}
	\par\bigskip
	\begin{subfigure}{1.0\textwidth}
		\centering
		\includegraphics[width=0.85\textwidth]{openmp_gflops_vs_nb_1_core_80_percent_memory.pdf}
		\caption{Hybrid OpenMPI/OpenMP}
		\label{fig:subim2}
	\end{subfigure}
\caption{\textbf{HPL 1 Core $\mathbf{R_{max}}$ versus NB.}}
\label{fig:image2}
\end{figure}

Note: There is no benefit in using a hybrid OpenMPI/OpenMP topology for a single core running a single \verb|xhpl| process, as only a single thread is used. However, to ensure similar results were achieved, both a pure OpenMPI and hybrid OpenMPI/OpenMP were benchmarked.

%
% SUB SUB SECTION
%
\subsubsection{Observations}

As expected, there is no significant performance difference between the two topologies for both OpenBLAS and BLIS.

OpenBLAS and BLIS both attain 80\% $R_{peak}$. Without competition from additional cores for access to the L2 cache, both libraries are able to efficiently stream data from main memory, through the L1 and L2 caches, to the core registers for computation. 


%
% SUB SECTION
%
\subsection{HPL 1 Node Baseline}

The purpose of this investigation is to determine the performance of the 4 cores of a single node. In this case each core shares the L2 cache with the other cores, so less L2 data will be available per core. This should result in more L2 \emph{cache misses} requiring a \emph{cache load} from main memory. It is therefore anticipated that this will result in a performance reduction, per core, compared to the single core case.

As per the single core benchmark, the HPL problem size is restricted to 80\% of available memory. Again, this is 80\% of a single node's 4 GB. This results in the same NB and N combinations as the single core benchmark of Table 5.1.

The benchmark results are plotted in Figure 5.3. 

%
% SUB SUB SECTION
%
\subsubsection{Observations}

As anticipated, there is indeed a reduction in performance per core, 80\% $R_{peak}$ in no longer attained.

Pure OpenMPI topology attains a $R_{max}$ of ?? with an NB of ??.

The hybrid OpenMPI/OpenMP topology attains a $R_{max}$ of ?? with an NB of ??.

\begin{figure}[H]
	\begin{subfigure}{1.0\textwidth}
		\centering
		\includegraphics[width=0.85\textwidth]{openmpi_gflops_vs_nb_1_node_80_percent_memory.pdf}
		\caption{Pure OpenMPI}
		\label{fig:subim1}
	\end{subfigure}
	\par\bigskip
	\begin{subfigure}{1.0\textwidth}
		\centering
		\includegraphics[width=0.85\textwidth]{openmp_gflops_vs_nb_1_node_80_percent_memory.pdf}
		\caption{Hybrid OpenMPI/OpenMP}
		\label{fig:subim2}
	\end{subfigure}
\caption{\textbf{HPL 1 Node $\mathbf{R_{max}}$ versus NB.}}
\label{fig:image2}
\end{figure}


%
% SUB SECTION
%
\subsection{HPL 2 Node Baseline}

The purpose of this baseline is to determine the performance of 2 nodes. Now, each core not only has to share access to the L2 cache, but the cache will be loaded with data less frequently due to network delays and competition between the nodes for access to network. It is therefore anticipated that this will result in a performance reduction, per node, compared to the single node case.

For this baseline the HPL problem size is restricted to 80\% of 2 nodes combined memory, 80\% of 8 GB. This results in the NB and N combinations in Table 5.2.

\begin{table}[H]
\begin{center}
	\begin{tabular}{ |c|c|c|c|c|c|c|c|c|c| } 
		\hline
		\textbf{NB} & \textbf{N} & \textbf{NB} & \textbf{N} & \textbf{NB} & \textbf{N} & \textbf{NB} & \textbf{N} & \textbf{NB} & \textbf{N} \\ 
		\hline
		32 & 26208 &  80 & 26160 & 128 & 26112 & 176 & 26048 & 224 & 26208 \\ 
		40 & 26200 &  88 & 26136 & 136 & 26112 & 184 & 26128 & 232 & 25984 \\ 
 		48 & 26208 &  96 & 26208 & 144 & 26208 & 192 & 26112 & 240 & 26160 \\
		56 & 26208 & 104 & 26208 & 152 & 26144 & 200 & 26200 & 248 & 26040 \\ 
 		64 & 26176 & 112 & 26208 & 160 & 26080 & 208 & 26208 & 256 & 26112 \\
		72 & 26208 & 120 & 26160 & 168 & 26208 & 216 & 26136 &   - &     - \\ 
 		\hline
	\end{tabular}
\end{center}
\caption{\label{tab:table-name}\textbf{2 Node NB and N Combinations.}  Block size NB and problem size N combinations for NB between 32 and 256 using 80\% of 8 GB of memory}
\end{table}

The results are plotted in Figure 5.4

\begin{figure}[H]
	\begin{subfigure}{1.0\textwidth}
		\centering
		\includegraphics[width=0.85\textwidth]{openmpi_gflops_vs_nb_2_node_80_percent_memory.pdf}
		\caption{Pure OpenMPI}
		\label{fig:subim1}
	\end{subfigure}
	\par\bigskip
	\begin{subfigure}{1.0\textwidth}
		\centering
		\includegraphics[width=0.85\textwidth]{openmp_gflops_vs_nb_2_node_80_percent_memory.pdf}
		\caption{Hybrid OpenMPI/OpenMP}
		\label{fig:subim2}
	\end{subfigure}
\caption{\textbf{HPL 2 Node $\mathbf{R_{max}}$ versus NB.}}
\label{fig:image2}
\end{figure}

%
% SUB SUB SECTION
%
\subsubsection{Observations}

...


%
% SUB SECTION
%
\subsection{HPL Whole Cluster Baseline}

This whole cluster baseline uses the optimum values of NB from the 2 node baseline.

\begin{table}[H]
\begin{center}
\begin{tabular}{ |l|c|c|c|c|c|c| } 
\hline
\textbf{BLAS Library} & \textbf{Nodes} & \textbf{N} & \textbf{NB} & \textbf{P} & \textbf{Q} & \textbf{$\mathbf{R_{max}}$} \textbf{(Gflops)} \\ 
\hline
OpenBLAS Serial & 3 & 32040 & 120 & 1 & 12 & 3.3720e+01 \\ 
                & 3 & 32040 & 120 & 2 &  6 & 3.1946e+01 \\
                & 3 & 32040 & 120 & 3 &  4 & 3.3844e+01 \\
                \cline{2-7} 
                & 4 & 36960 & 120 & 1 & 16 & 4.7742e+01 \\ 
                & 4 & 36960 & 120 & 2 &  8 & 4.9390e+01 \\ 
                \cline{2-7} 
                & 5 & 41400 & 120 & 1 & 20 & 5.6513e+01 \\ 
                & 5 & 41400 & 120 & 2 & 10 & 5.6038e+01 \\ 
                & 5 & 41400 & 120 & 4 &  5 & 5.5649e+01 \\ 
                \cline{2-7} 
                & 6 & 45360 & 120 & 1 & 24 & 6.8392e+01 \\ 
                & 6 & 45360 & 120 & 2 & 12 & 7.3856e+01 \\ 
                & 6 & 45360 & 120 & 3 &  8 & 6.9952e+01 \\ 
                \cline{2-7} 
                & 7 & 48960 & 120 & 1 & 28 & 7.8248e+01 \\ 
                & 7 & 48960 & 120 & 2 & 14 & 8.1017e+01 \\ 
                & 7 & 48960 & 120 & 4 &  7 & 8.1433e+01 \\ 
                \cline{2-7} 
                & 8 & 52320 & 120 & 1 & 32 & 8.6787e+01 \\ 
                & 8 & 52320 & 120 & 2 & 16 & 9.5517e+01 \\ 
                & 8 & 52320 & 120 & 4 &  8 & 9.5525e+01 \\ 
\hline
OpenBLAS OpenMP & 3 & 32032 & 88 & 1 & 3 & 3.7842e+01 \\ 
                \cline{2-7} 
                & 4 & 37048 & 88 & 1 & 4 & 4.8657e+01 \\ 
                \cline{2-7} 
                & 5 & 41448 & 88 & 1 & 5 & 6.0428e+01 \\ 
                \cline{2-7} 
                & 6 & 45320 & 88 & 1 & 6 & 6.8713e+01 \\ 
                & 6 & 45320 & 88 & 2 & 3 & 7.3722e+01 \\ 
                \cline{2-7} 
                & 7 & 49016 & 88 & 1 & 7 & 7.8712e+01 \\ 
                \cline{2-7} 
                & 8 & 52360 & 88 & 1 & 8 & 9.4245e+01 \\ 
                & 8 & 52360 & 88 & 2 & 4 & 9.6630e+01 \\ 
\hline
\end{tabular}
\end{center}
\caption{\label{tab:table-name}\textbf{HPL Whole Cluster OpenBLAS Baseline.}}
\end{table}



\begin{table}
\begin{center}
\begin{tabular}{ |l|c|c|c|c|c|c| } 
\hline
\textbf{BLAS Library} & \textbf{Nodes} & \textbf{N} & \textbf{NB} & \textbf{P} & \textbf{Q} & \textbf{$\mathbf{R_{max}}$ (Gflops)} \\ 
\hline
BLIS Serial     & 3 & 32088 & 168 & 1 & 12 & 3.9005e+01 \\ 
                & 3 & 32088 & 168 & 2 &  6 & 3.9050e+01 \\ 
                & 3 & 32088 & 168 & 3 &  4 & 3.8958e+01 \\ 
                \cline{2-7} 
                & 4 & 36960 & 168 & 1 & 16 & 4.9694e+01 \\ 
                & 4 & 36960 & 168 & 2 &  8 & 5.4268e+01\\ 
                \cline{2-7} 
                & 5 & 41328 & 168 & 1 & 20 & 5.5398e+01 \\ 
                & 5 & 41328 & 168 & 2 & 10 & 6.5226e+01 \\ 
                & 5 & 41328 & 168 & 4 &  5 & 6.2356e+01 \\ 
                \cline{2-7} 
                & 6 & 45360 & 168 & 1 & 24 & 7.0278e+01 \\ 
                & 6 & 45360 & 168 & 2 & 12 & 7.9685e+01 \\ 
                & 6 & 45360 & 168 & 3 &  8 & 7.5475e+01 \\ 
                \cline{2-7} 
                & 7 & 48888 & 168 & 1 & 28 & 8.0168e+01 \\ 
                & 7 & 48888 & 168 & 2 & 14 & 8.7571e+01 \\ 
                & 7 & 48888 & 168 & 4 &  7 & 8.6035e+01 \\ 
                \cline{2-7} 
                & 8 & 52416 & 168 & 1 & 32 & 9.1148e+01 \\ 
                & 8 & 52416 & 168 & 2 & 16 & 1.0341e+02 \\ 
                & 8 & 52416 & 168 & 4 &  8 & 1.0190e+02 \\ 
\hline
BLIS OpenMP     & 3 & 32000 & 200 & 1 & 3 & 3.5132e+01 \\ 
                \cline{2-7} 
                & 4 & 37000 & 200 & 1 & 4 & 4.6953e+01 \\ 
                \cline{2-7} 
                & 5 & 41400 & 200 & 1 & 5 & 6.2550e+01 \\ 
                \cline{2-7} 
                & 6 & 45400 & 200 & 1 & 6 & 6.7204e+01 \\ 
                & 6 & 45400 & 200 & 2 & 3 & 7.2585e+01 \\ 
                \cline{2-7} 
                & 7 & 49000 & 200 & 1 & 7 & 8.1255e+01 \\ 
                \cline{2-7} 
                & 8 & 52400 & 200 & 1 & 8 & 9.1180e+01 \\ 
                & 8 & 52400 & 200 & 2 & 4 & 1.0041e+02 \\ 
\hline
\end{tabular}
\end{center}
\caption{\label{tab:table-name}\textbf{HPL Whole Cluster BLIS Baseline.}}
\end{table}



The high value of $R_{max}$ for each node count is plotting in Figure 5.5.

\begin{figure}[H]
	\begin{subfigure}{1.0\textwidth}
		\centering
		\includegraphics[width=0.85\textwidth]{openmpi_gflops_vs_nodes_80_percent_memory.pdf}
		\caption{Pure OpenMPI}
		\label{fig:subim1}
	\end{subfigure}
	\par\bigskip
	\begin{subfigure}{1.0\textwidth}
		\centering
		\includegraphics[width=0.85\textwidth]{openmp_gflops_vs_nodes_80_percent_memory.pdf}
		\caption{Hybrid OpenMPI/OpenMP}
		\label{fig:subim2}
	\end{subfigure}
\caption{\textbf{HPL $\mathbf{R_{max}}$ versus Node Count.}}
\label{fig:image2}
\end{figure}


%
% SUB SECTION
%
\subsection{Observations}

Best NB...

PxQ discussion... 1x8 vs 2x4... ethernet comment...

Iperf...

htop...

top...

perf...

cache misses...

software interrupts...

Suggests... improve network efficiency?



%
% SECTION
%
\section{HPCC Baseline}

The HPCC benchmark suite was run using all 8 nodes of the Aerin Cluster to obtain a whole cluster baseline.

Recall from Chapter 2, \emph{single} indicates the benchmark is run on a single randomly selected node, \emph{star} indicates the benchmark in run independently on all nodes, and \emph{global} indicates the benchmark is run using all nodes in a coordinated manner.

The results for each benchmark are presented below.


%
% SUB SECTION
%
\subsection{HPL}

HPL is included in HPCC. The performance results when running HPL as part of HPCC were similar to those when running HPL as a standalone benchmark.


%
% SUB SECTION
%
\subsection{DGEMM}

\begin{table}[H]
\begin{center}
\begin{tabular}{ |l|l| } 
\hline
\textbf{BLAS Library} & \textbf{Results} \\ 
\hline
OpenBLAS Serial & DGEMM\_N=5339 \\
                & StarDGEMM\_Gflops=3.59743 \\
                & SingleDGEMM\_Gflops=4.91086 \\
\hline
OpenBLAS OpenMP & DGEMM\_N=10687 \\
                & StarDGEMM\_Gflops=14.4261 \\
                & SingleDGEMM\_Gflops=14.426 \\
\hline
BLIS Serial     & DGEMM\_N=5349 \\
                & StarDGEMM\_Gflops=3.02439 \\
                & SingleDGEMM\_Gflops=4.95418 \\
\hline
BLIS OpenMP     & DGEMM\_N=10695 \\
                & StarDGEMM\_Gflops=16.3355 \\
                & SingleDGEMM\_Gflops=15.2042 \\
\hline
\end{tabular}
\end{center}
\caption{\label{tab:table-name}\textbf{HPCC DGEMM.}}
\end{table}

The DGEMM benchmark measures the performance of double precision real \emph{matrix-matrix multiplication}. The benchmark results are presented in Table 5.5.

For the single-threaded serial versions of the OpenBLAS and BLIS libraries the cluster consists of 32 processing cores. The \verb|SingleDGEMM_Gflops| results are per core.

For the multi-threaded OpenMP versions of the libraries, the cluster consists of 8 processing nodes. The \verb|SingleDGEMM_Gflops| results are per node.

\subsubsection{Observations}

The results are consistent with the HPL benchmarks, which spends approximately 87\% of the benchmark run time in the BLAS \verb|dgemm| subroutine. 

Of note  is the \emph{jitter} between the \emph{single} and \emph{star} results, particularly the OpenBLAS Serial and BLIS Serial results. This is explained by the fact that a \emph{single} randomly selected core running the benchmark has exclusive access to the L2 cache. The \emph{single} result is consistent with 1 Core Baseline results. In the \emph{star} case, all cores on each node are running the benchmark and have to share access to the L2 cache. The \emph{star} result is consistent with the 1 Node Baseline results.


%
% SUB SECTION
%
\subsection{STREAM}

The STREAM benchmark measures sustained memory bandwidth by performing four vector operations, \emph{Copy}, \emph{Scale}, \emph{Sum} and \emph{Triad}, on vectors which are at least 4 times the size of the L2 cache. This ensures the benchmark is measuring main memory access performance.

For the purposes of measuring pure memory bandwidth the \emph{Copy} operation is the most appropriate. This measures the copying of a vector from one memory location to another, without any computation on the vector data.   

The STREAM benchmark results are presented in table 5.6.

\begin{table}[]
\begin{center}
\begin{tabular}{ |l|l| } 
\hline
\textbf{BLAS Library} & \textbf{Results} \\ 
\hline
OpenBLAS Serial & STREAM\_VectorSize=28514400 \\
                & STREAM\_Threads=1 \\
                & StarSTREAM\_Copy=0.92926 \\
                & StarSTREAM\_Scale=0.979969 \\
                & StarSTREAM\_Add=0.902324 \\
                & StarSTREAM\_Triad=0.899619 \\
                & SingleSTREAM\_Copy=5.36868 \\
                & SingleSTREAM\_Scale=5.41684 \\
                & SingleSTREAM\_Add=4.75638 \\
                & SingleSTREAM\_Triad=4.75692 \\
\hline
OpenBLAS OpenMP & STREAM\_VectorSize=114232066 \\
                & STREAM\_Threads=1 \\
                & StarSTREAM\_Copy=4.76068 \\
                & StarSTREAM\_Scale=5.44287 \\
                & StarSTREAM\_Add=4.51713 \\
                & StarSTREAM\_Triad=4.53621 \\
                & SingleSTREAM\_Copy=5.47035 \\
                & SingleSTREAM\_Scale=5.46963 \\
                & SingleSTREAM\_Add=4.87128 \\
                & SingleSTREAM\_Triad=4.89569 \\
\hline
BLIS Serial     & STREAM\_VectorSize=28619136 \\
                & STREAM\_Threads=1 \\
                & StarSTREAM\_Copy=0.943137 \\
                & StarSTREAM\_Scale=0.989024 \\
                & StarSTREAM\_Add=0.910843 \\
                & StarSTREAM\_Triad=0.909211 \\
                & SingleSTREAM\_Copy=4.72341 \\
                & SingleSTREAM\_Scale=4.21768 \\
                & SingleSTREAM\_Add=3.90016 \\
                & SingleSTREAM\_Triad=3.94385 \\
\hline
BLIS OpenMP     & STREAM\_VectorSize=114406666 \\
                & STREAM\_Threads=1 \\
                & StarSTREAM\_Copy=5.05861 \\
                & StarSTREAM\_Scale=5.39591 \\
                & StarSTREAM\_Add=4.66044 \\
                & StarSTREAM\_Triad=4.6751 \\
                & SingleSTREAM\_Copy=5.41884 \\
                & SingleSTREAM\_Scale=5.45544 \\
                & SingleSTREAM\_Add=4.80613 \\
                & SingleSTREAM\_Triad=4.81397 \\
\hline
\end{tabular}
\end{center}
\caption{\label{tab:table-name}\textbf{HPCC STREAM.}}
\end{table}


\subsubsection{Observations}

As noted previously, the \emph{single} OpenBLAS Serial and BLIS Serial benchmarks relate to single core. And as indicated in the results, the maximum observed single core memory bandwidth is approximately 5.4 MB/s. The \emph{star} serial results of approximately 0.94 GB/s need to be factored by the node count to measure the whole node main memory bandwidth, which is then the same order of magnitude but reduced due to L2 cache collisions. 

For bandwidth measurement, the STREAM benchmark counts both the memory read and the memory write as a memory movement. This differs from most memory bandwidth benchmarks in which the read and write from one memory location to another count as a single memory movement. Therefore, for the \emph{Copy} operation, the benchmark results needs to be factored by 0.5 to align with other benchmarks and memory specifications.

For three out of the 4 BLAS library combinations, the observed \emph{Copy} vector operation bandwidth is approximately 5.4 GB/s.

The Raspberry Pi 4 Model B is equipped with LPDDR4–3200 SDRAM (Low-Power Double Data Rate Static DRAM), with a maximum data transfer rate of 3200 MB/s (3.2 GB/s).

The observed benchmark \emph{Copy} performance of 5.4 GB/s, when factored by 0.5 is 2.7 GB/s. This is 80\% of the maximum data transfer rate. This suggests that the Raspberry Pi 4 Model B is making good use of the available memory bandwidth. 



%
% SUB SECTION
%
\subsection{PTRANS}

PTRANS is a \emph{global} benchmark which implements the transpose of a large matrix in memory using the cluster nodes operating in parallel. The main purpose of this benchmark is to test inter-node communication performance.

The value of NB was selected to be optimum value from the 2 Node Baseline, with the corresponding value of N equating to 40\% of total memory. Since this not a computation benchmark, 40\% memory usage is sufficient to test inter-node communication without excessive run time. 



\begin{table}[]
\begin{center}
\begin{tabular}{ |l|l| } 
\hline
\textbf{BLAS Library} & \textbf{Results} \\
\hline
OpenBLAS Serial & PTRANS\_GBs=0.465891 \\
                & PTRANS\_n=26160 \\ 
                & PTRANS\_nb=120 \\
                & PTRANS\_nprow=1 \\
                & PTRANS\_npcol=32 \\
\hline
OpenBLAS OpenMP & PTRANS\_GBs=0.616885 \\
                & PTRANS\_n=26180 \\
                & PTRANS\_nb=88 \\
                & PTRANS\_nprow=2 \\
                & PTRANS\_npcol=4 \\
\hline
BLIS Serial     & PTRANS\_GBs=0.483766 \\
                & PTRANS\_n=26208 \\
                & PTRANS\_nb=168 \\
                & PTRANS\_nprow=1 \\
                & PTRANS\_npcol=32 \\
\hline
BLIS OpenMP     & PTRANS\_GBs=0.637484 \\
                & PTRANS\_n=26200 \\
                & PTRANS\_nb=200 \\
                & PTRANS\_nprow=2 \\
                & PTRANS\_npcol=4 \\
\hline
\end{tabular}
\end{center}
\caption{\label{tab:table-name}\textbf{HPCC PTRANS.}}
\end{table}


\subsubsection{Observations}

Both the OpenBLAS and BLIS serial benchmark runs produce similar results. Similarly, both OpenMP benchmark runs produce similar results. This is to be expected, since this benchmark is not testing BLAS library performance. The difference between the serial and OpenMP performance is related to the topology of the cluster, either a pure OpenMPI or hybrid OpenMPI/OpenMP topology.

With an average pure OpenMPI result of 0.475 GB/s, and a hybrid OpenMPI/OpenMP result of 0.627 GB/s, the inter-node communication performance is 32\% faster using the hybrid OpenMPI/OpenMP topology.

This result is to be expected, since the node count is 32 in the pure OpenMPI topology, but only 8 in the hybrid OpenMPI/OpenMP topology. Even though the amount of matrix data to be transposed remains the same, the inter-node communication is reduced. The inter-node messages may be larger, but there are fewer messages with less node addressing overhead, making the cluster network more efficient. 


%
% SUB SECTION
%
\subsection{Random Access}

The Random Access benchmark tests the integer update performance of a large array in memory. Random numbers generated from a normal distribution and the Linear Congruential Generator algorithm are used. The unit of the results is GUPs (\emph{GigaUpdates per Second}).

\begin{table}[]
\begin{center}
\begin{tabular}{ |l|l| } 
\hline
\textbf{BLAS Library} & \textbf{Results} \\ 
\hline
OpenBLAS Serial & MPIRandomAccess\_LCG\_N=2147483648 \\
                & MPIRandomAccess\_LCG\_GUPs=0.000642364 \\
                \cline{2-2} 
                & MPIRandomAccess\_N=2147483648 \\
                & MPIRandomAccess\_GUPs=0.000645338 \\
                \cline{2-2} 
                & RandomAccess\_LCG\_N=67108864 \\
                & StarRandomAccess\_LCG\_GUPs=0.00373175 \\
                & SingleRandomAccess\_LCG\_GUPs=0.00815537 \\
                \cline{2-2} 
                & RandomAccess\_N=67108864 \\
                & StarRandomAccess\_GUPs=0.00373372 \\
                & SingleRandomAccess\_GUPs=0.00837312 \\ 
\hline
OpenBLAS OpenMP & MPIRandomAccess\_LCG\_N=2147483648 \\
                & MPIRandomAccess\_LCG\_GUPs=0.000473649 \\
                \cline{2-2} 
                & MPIRandomAccess\_N=2147483648 \\
                & MPIRandomAccess\_GUPs=0.000477404 \\
                \cline{2-2} 
                & RandomAccess\_LCG\_N=268435456 \\
                & StarRandomAccess\_LCG\_GUPs=0.00580416 \\
                & SingleRandomAccess\_LCG\_GUPs=0.00582959 \\
                \cline{2-2} 
                & RandomAccess\_N=268435456 \\
                & StarRandomAccess\_GUPs=0.00614337 \\
                & SingleRandomAccess\_GUPs=0.00613214 \\
\hline
BLIS Serial     & MPIRandomAccess\_LCG\_N=2147483648 \\
                & MPIRandomAccess\_LCG\_GUPs=0.000644523 \\
                \cline{2-2} 
                & MPIRandomAccess\_N=2147483648 \\
                & MPIRandomAccess\_GUPs=0.00064675 \\
                \cline{2-2} 
                & RandomAccess\_LCG\_N=67108864 \\
                & StarRandomAccess\_LCG\_GUPs=0.00374527 \\
                & SingleRandomAccess\_LCG\_GUPs=0.00835127 \\
                \cline{2-2} 
                & RandomAccess\_N=67108864 \\
                & StarRandomAccess\_GUPs=0.00374741 \\
                & SingleRandomAccess\_GUPs=0.00820883 \\
\hline
BLIS OpenMP     & MPIRandomAccess\_LCG\_N=2147483648 \\
                & MPIRandomAccess\_LCG\_GUPs=0.000475485 \\
                \cline{2-2} 
                & MPIRandomAccess\_N=2147483648 \\
                & MPIRandomAccess\_GUPs=0.000476047 \\
                \cline{2-2} 
                & RandomAccess\_LCG\_N=268435456 \\
                & StarRandomAccess\_LCG\_GUPs=0.00580705 \\
                & SingleRandomAccess\_LCG\_GUPs=0.00578222 \\
                \cline{2-2} 
                & RandomAccess\_N=268435456 \\
                & StarRandomAccess\_GUPs=0.00614596 \\
                & SingleRandomAccess\_GUPs=0.00613275 \\
\hline
\end{tabular}
\end{center}
\caption{\label{tab:table-name}\textbf{HPCC Random Access.}}
\end{table}


\subsubsection{Observations}

For the \emph{global} benchmarks, the array size $N = 2,147,483,648$ represents 25\% of the cluster's total 32 GB of memory.

For the \emph{single} pure OpenMPI benchmarks, the array size $N = 67,108,864$ represents 25\% of 1 GB of memory. Each of the 4 cores of each node is allocated 1 GB of the total 4 GB of memory, and the array is selected to be 25\% of this.

And, for the \emph{single} hybrid OpenMPI/OpenMP benchmarks, the array size $N = 268,435,456$ represents 25\% of a single nodes's 4 GB of memory.

The above array sizes are determined by the benchmark and not selected by the user.

As expected, because this benchmark is not a test of computational performance, the pure OpenMPI topology performance is almost identical for both BLAS libraries. Similarly, the hybrid OpenMPI/OpenMP topology performance is almost identical for both BLAS libraries.

 


%
% SUB SECTION
%
\subsection{FFT}

The FFT benchmark...

\begin{table}[h]
\begin{center}
\begin{tabular}{ |l|c| } 
\hline
\textbf{BLAS Library} & \textbf{Results} \\ 
\hline
OpenBLAS Serial & Global Vector size: 268435456\\
                & Global Gflops:      1.569 \\
                \cline{2-2} 
                & Star Vector size:   16777216 \\
                & Star Min Gflops:    0.251138 \\
                & Star Avg Gflops:    0.314311 \\ 
                & Star Min Gflops:    0.335671 \\
                \cline{2-2}
                & Single Gflops:      0.494388 \\
\hline
OpenBLAS OpenMP & Global Vector size: 268435456 \\
                & Global Gflops:      1.347 \\
                \cline{2-2} 
                & Star Vector size:   67108864 \\
                & Star Min Gflops:    0.314714 \\
                & Star Avg Gflops:    0.344635 \\ 
                & Star Min Gflops:    0.375736 \\
                \cline{2-2}
                & Single Gflops:      0.532823 \\
\hline
BLIS Serial     & Global Vector size: 268435456 \\
                & Global Gflops:      1.563 \\
                \cline{2-2} 
                & Star Vector size:   16777216 \\
                & Star Min Gflops:    0.252763 \\
                & Star Avg Gflops:    0.312753 \\ 
                & Star Min Gflops:    0.344300 \\
                \cline{2-2}
                & Single Gflops:      0.515476\\
\hline
BLIS OpenMP     & Global Vector size: 268435456 \\
                & Global Gflops:      1.313 \\
                \cline{2-2} 
                & Star Vector size:   67108864 \\
                & Star Min Gflops:    0.315112 \\
                & Star Avg Gflops:    0.329216 \\ 
                & Star Min Gflops:    0.344110 \\
                \cline{2-2}
                & Single Gflops:      0.439117 \\
\hline
\end{tabular}
\end{center}
\caption{\label{tab:table-name}\textbf{HPCC FFT.}}
\end{table}


\subsubsection{Observations}

The...


%
% SUB SECTION
%
\subsection{Network Bandwidth and Latency}

The Network Bandwidth and Latency benchmark measures the time to send MPI messages between cluster processes. Latency is measured using 8 byte messages, and bandwidth is measured using 2,000,000 byte messages.

The major benchmark results are presented in table 5.10.

\begin{table}[h]
\begin{center}
\begin{tabular}{ |c| } 
\hline
\textbf{OpenBLAS Serial} \\ 
\hline
Max Ping Pong Latency:                 0.189604 msecs \\
Randomly Ordered Ring Latency:         0.192830 msecs \\
Min Ping Pong Bandwidth:             107.306358 MB/s \\
Naturally Ordered Ring Bandwidth:     61.385709 MB/s \\
Randomly  Ordered Ring Bandwidth:     16.907255 MB/s \\
\hline
\textbf{OpenBLAS OpenMP} \\ 
\hline
Max Ping Pong Latency:                 0.099864 msecs \\
Randomly Ordered Ring Latency:         0.075013 msecs \\
Min Ping Pong Bandwidth:             112.574381 MB/s \\
Naturally Ordered Ring Bandwidth:     70.511474 MB/s \\
Randomly  Ordered Ring Bandwidth:     73.119334 MB/s \\
\hline
\textbf{BLIS Serial} \\ 
\hline
Max Ping Pong Latency:                 0.191411 msecs \\
Randomly Ordered Ring Latency:         0.189172 msecs \\
Min Ping Pong Bandwidth:             107.075234 MB/s \\
Naturally Ordered Ring Bandwidth:     51.395031 MB/s \\
Randomly  Ordered Ring Bandwidth:     16.492559 MB/s \\
\hline
\textbf{BLIS OpenMP} \\ 
\hline
Max Ping Pong Latency:                 0.101356 msecs \\
Randomly Ordered Ring Latency:         0.070292 msecs \\
Min Ping Pong Bandwidth:             112.788813 MB/s \\
Naturally Ordered Ring Bandwidth:     99.948389 MB/s \\
Randomly  Ordered Ring Bandwidth:     72.585888 MB/s \\
\hline
\end{tabular}
\end{center}
\caption{\label{tab:table-name}\textbf{HPCC Network Bandwidth and Latency.}}
\end{table}

The BLAS libraries are not used, but the results for all four library combinations are included for consistency with other benchmarks results.

A clear distinction needs to be made between Gigabits and Gigabytes/Megabytes. Using the standard \emph{byte} of 8 bits, 1 Gigabits per second (1 Gb/s) is 125 Megabytes per second (125 MB/s). This is the theoretical maximum bandwidth of the Raspberry Pi 4's Gigabit Ethernet interface.


\subsubsection{Observations}

The benchmark results are specified in Megabytes per second (MB/s).

The average OpenMP topology bandwidth is 112.682 MB/s, which is 90\% of the maximum theoretical bandwidth. The is consistent with 92.2\% of \emph{node-to-node} maximum theoretical bandwidth observed using the Linux \verb|iperf| command in Section ??.



The average serial topology bandwidth is 107.548 MB/s, which is 86\% of the maximum theoretical bandwidth. It is expected to observe a lower bandwidth with the serial topology due to the increased MPI process count compared to the the OpenMP topology. 



%
% SECTION
%
\section{HPCG Baseline}

The June 2020 HPCG List ranks 169 computer in order of HPCG benchmark performance.

Ranking number 1 is the Fugaku supercomputer.

Ranking 169 is the Spaceborne Computer. The Spaceborne Computer, onboard the International Space Station (ISS), is a 32 core system based on the Intel Xeon E5-2620 v4 8 core CPU, clocked at 2.1GHz, with an Infiniband interconnect.

An extract from the list for these two computers is in Table ??.

\begin{table}[H]
\begin{center}
\begin{tabular}{ |c|c|c|c|c|c|c| } 
\hline
\textbf{HPCG} & \textbf{Name} & \textbf{Cores} & \textbf{HPL $\mathbf{R_{max}}$} & \textbf{HPCG}   & \textbf{Fraction} \\
\textbf{Rank} &      &       & \textbf{Pflops}        & \textbf{Pflops} & \textbf{of Peak}  \\
\hline
1 & Fugaku & 6,635,520 & 415.530  & 13.366 & 2.6\% \\
\hline
169 & Spaceborne & 32 & 0.001  & 0.000034 & 2.9\% \\
    & Computer   &    &        &          &       \\
\hline
\end{tabular}
\end{center}
\caption{\label{tab:table-name}\textbf{Extract from June 2020 HPCG List.}}
\end{table}

For computers to be officially ranked in the HPCG List, the HPCG benchmark must be run for in excess of 30 minutes, using at least 25\% of available memory.

The HPCG benchmark results for the Aerin Cluster, obtained running the benchmark for 60 minutes using 75\% of available memory, are presented in Table ??.

\begin{table}[H]
\begin{center}
\begin{tabular}{ |c|c|c|c|c|c|c| } 
\hline
\textbf{HPCG} & \textbf{Name} & \textbf{Cores} & \textbf{HPL $\mathbf{R_{max}}$} & \textbf{HPCG}   & \textbf{Fraction} \\
\textbf{Rank} &      &       & \textbf{Gflops}        & \textbf{Gflops} & \textbf{of Peak}  \\
\hline
- & OpenBLAS & 32 & 9.5525e+01 & 3.49084 & 1.8\% \\
  & Serial   &   &  &  &    \\
\hline
- & OpenBLAS & 32 & 9.6630e+01 & 2.90942 & 1.5\% \\
  & OpenMP   &   &  &  &    \\
\hline
- & BLIS     & 32 & 1.0341e+02 & 3.44246 & 1.8\% \\
  & Serial   &   &  &  &    \\
\hline
- & BLIS     & 32 & 1.0041e+02 & 2.95205 & 1.5\% \\
  & OpenMP   &   &  &  &    \\
\hline
\end{tabular}
\end{center}
\caption{\label{tab:table-name}\textbf{Aerin Cluster HPCG Benchmark results.}}
\end{table}


\subsubsection{Observations}

The Aerin Cluster...


%
% SECTION
%
\section{Optimisations}


\lstset{style=type}
\begin{lstlisting}
$ sudo perf record mpirun -allow-run-as-root -np 4 xhpl
\end{lstlisting}



Running xhpl on 8 nodes using OpenBLAS...

\lstset{style=type}
\begin{lstlisting}
$ mpirun -host node1:4 ... node8:4 -np 32 xhpl
\end{lstlisting}


SHORTLY AFTER PROGRAM START...

On node1,... where we initiated...

top...

\lstset{style=type}
\begin{lstlisting}
top - 20:33:15 up 8 days,  6:02,  1 user,  load average: 4.02, 4.03, 4.00
Tasks: 140 total,   5 running, 135 sleeping,   0 stopped,   0 zombie
%Cpu(s): 72.5 us, 21.7 sy,  0.0 ni,  0.0 id,  0.0 wa,  0.0 hi,  5.8 si,  0.0 st
MiB Mem :   3793.3 total,    330.1 free,   3034.9 used,    428.3 buff/cache
MiB Swap:      0.0 total,      0.0 free,      0.0 used.    698.7 avail Mem 

    PID USER      PR  NI    VIRT    RES    SHR S  %CPU  %MEM     TIME+ COMMAND                                                   
  34884 john      20   0  932964 732156   7980 R 100.3  18.8 106:40.29 xhpl                                                      
  34881 john      20   0  933692 732272   7916 R 100.0  18.9 107:29.75 xhpl                                                      
  34883 john      20   0  932932 731720   8136 R  99.3  18.8 107:33.25 xhpl                                                      
  34882 john      20   0  932932 731784   8208 R  97.7  18.8 107:33.64 xhpl                                                      
\end{lstlisting}

SOFTIRQS...


NODE 2 - 2 NODES ONLY TO SEE EFFECT...

IPERF!!!

On node8, running the top command...

\lstset{style=type}
\begin{lstlisting}
$ top
\end{lstlisting}

We can see...

\lstset{style=type}
\begin{lstlisting}
top - 18:58:44 up 8 days,  4:29,  1 user,  load average: 4.00, 3.75, 2.35
Tasks: 133 total,   5 running, 128 sleeping,   0 stopped,   0 zombie
%Cpu(s): 50.7 us, 47.8 sy,  0.0 ni,  0.0 id,  0.0 wa,  0.0 hi,  1.4 si,  0.0 st
MiB Mem :   3793.3 total,    392.7 free,   2832.6 used,    568.0 buff/cache
MiB Swap:      0.0 total,      0.0 free,      0.0 used.    901.1 avail Mem 

    PID USER      PR  NI    VIRT    RES    SHR S  %CPU  %MEM     TIME+ COMMAND                                                   
  23928 john      20   0  883880 682456   8200 R 100.0  17.6  13:14.17 xhpl                                                      
  23927 john      20   0  883988 682432   7932 R  99.7  17.6  13:12.58 xhpl                                                      
  23930 john      20   0  883912 682664   7832 R  99.7  17.6  13:17.01 xhpl                                                      
  23929 john      20   0  883880 682640   8376 R  99.3  17.6  13:16.25 xhpl  
\end{lstlisting}

Indicates that only 50.7\% of CPU time is being utilised by user programs (us), Linpack/OpenMPI...

I hypothesise that the 1.4\% of software interrupts (si) is responsible 47.8\% of CPU time in the kernel (sy) servicing these interupts...

Lets have a look at the software interrupts on the system...

\lstset{style=type}
\begin{lstlisting}
$ watch -n 1 cat /proc/softirqs
\end{lstlisting}


\lstset{style=type}
\begin{lstlisting}
Every 1.0s: cat /proc/softirqs

                    CPU0       CPU1       CPU2       CPU3
          HI:          0          1          0          1
       TIMER:  122234556   86872295   85904119   85646345
      NET_TX:  222717797     228381     147690     144396
      NET_RX: 1505715680       1132       1294       1048
       BLOCK:      63160      11906      13148      11223
    IRQ_POLL:          0          0          0          0
     TASKLET:   58902273         33          2          6
       SCHED:    3239933    3988327    2243001    2084571
     HRTIMER:       8116         55         53         50
         RCU:    6277982    4069531    4080009    3994395
\end{lstlisting}

As can be seen...

1. the majority of software interrupts are being generated by network receive (NET\_RX) activity, followed by network transmit activity (NET\_TX)...

2. these interrupts are being almost exclusively handled by CPU0...

What is there to be done?...

1. Reduce the numbers of interrupts...

1.1 Each packet produces an interrupt - interrupt coalesing...

1.2 Reduce the number of packets - increase MTU...

2.1 Share the interrupt servicing activity evenly across the CPUs...



%
% SUB SECTION
%
\subsection{Interrupt Coalescing}

As discussed in Chapter 2, each packet received by a network interface generates a hardware interrupt. This results in a context switch to the kernel to process the packet. \emph{Interrupt coalescing} delays the raising of a hardware interrupt until a specified number of packets have been received, or a specified period of time has elapsed, thereby reducing the number of context switches. This potentially improves throughput and benchmark performance.

The Pi Cluster Tools \verb|interrupt-coalescing| tool enables \emph{Adaptive RX} interrupt coalescing. In \emph{Adaptive RX} interrupt coalescing the kernel actively manages the number of packets received, and the time elapsed, before raising a hardware interrupt on the network interface receive queue.

The results of running the \verb|linpack-profiler| tool with \emph{Adaptive RX} interrupt coalescing enabled are plotted in Figure ??.

\begin{figure}[h]
	\centering
	\includegraphics[width=0.9\textwidth]{profile_interrupt_coalescing.pdf}
	\caption{Pi Cluster Tools \texttt{linpack-profiler} $R_{max}$ with \emph{Adaptive RX} interrupt coalescing enabled.}
	\label{fig:subim1}
\end{figure}

Using the value of the block size NB which achieved optimum performance running the \texttt{linpack-profiler} tool, the HPL benchmark results using 80\% of available memory are presented in Table ??. 

\begin{table}[H]
\begin{center}
\begin{tabular}{ |c|c|c|c|c|c|c| } 
\hline
\multicolumn{5}{|c}{} & \multicolumn{2}{|c|}{$\mathbf{R_{max}}$ \textbf{(Gflops)}} \\
\hline
\textbf{BLAS} & \textbf{N} & \textbf{NB} & \textbf{P} & \textbf{Q} & \textbf{Baseline}  & \textbf{Adaptive RX}  \\
              &            &             &            &            &                    & \textbf{Coalescing} \\
\hline
OpenBLAS Serial & 52416 & 104 & 2 & 16 &  & 9.6078e+01 \\
\hline
OpenBLAS OpenMP & 52416 & 72 & 2 & 4 &  & 9.3222e+01 \\
\hline
BLIS Serial     & 52416 & 168 & 4 & 8 &  & 1.0205e+02 \\
\hline
BLIS OpenMP     & 52416 & 144 & 2 & 4 &  & 1.0582e+02 \\
\hline
\end{tabular}
\end{center}
\caption{\label{tab:table-name}\textbf{HPL $\mathbf{R_{max}}$ with \emph{Adaptive RX} interrupt coalescing enabled.}}
\end{table}

\subsubsection{Observations}

Observations and discussion...


%
% SUB SECTION
%
\subsection{Receive Packet Steering and Receive Flow Steering}

The results of running the \verb|linpack-profiler| tool with \emph{Receive Packet Steering} and \emph{Receive Flow Steering} enabled are plotted in Figure ??.

\begin{figure}[h!]
	\centering
	\includegraphics[width=0.9\textwidth]{profile_rps_rfs.pdf}
	\caption{Pi Cluster Tools \texttt{linpack-profiler} $R_{max}$ with Receive Packet Steering and Receive Flow Sterring enabled.}
	\label{fig:subim1}
\end{figure}

Using the value of the block size NB which achieved optimum performance running the \texttt{linpack-profiler} tool, the HPL benchmark results using 80\% of available memory are presented in Table ??. 

\begin{table}[H]
\begin{center}
\begin{tabular}{ |c|c|c|c|c|c|c| } 
\hline
\multicolumn{5}{|c}{} & \multicolumn{2}{|c|}{$\mathbf{R_{max}}$ \textbf{(Gflops)}} \\
\hline
\textbf{BLAS} & \textbf{N} & \textbf{NB} & \textbf{P} & \textbf{Q} & \textbf{Baseline}  & \textbf{Adaptive RX}  \\
              &            &             &            &            &                    & \textbf{Coalescing} \\
\hline
OpenBLAS Serial &  &  &  &  &  &  \\
\hline
OpenBLAS OpenMP &  &  &  &  &  &  \\
\hline
BLIS Serial     &  &  &  &  &  &  \\
\hline
BLIS OpenMP     &  &  &  &  &  &  \\
\hline
\end{tabular}
\end{center}
\caption{\label{tab:table-name}\textbf{HPL $\mathbf{R_{max}}$ with \emph{Receive Packet Steering} and \emph{Receive Flow Steering} enabled.}}
\end{table}


\subsubsection{Observations}

Observations and discussion...



%
% SUB SECTION
%
\subsection{Kernel Preemption Model}

The Linux kernel has 3 Preemption Models, as discussed in Chapter 2:

\begin{itemize}
\item Preemptive
\item Voluntary Preemption
\item No Forced Preemption
\end{itemize}

For scientific computing workloads the \emph{No Forced Preemption} model should be used, as suggested by the \emph{help} accompanying the Linux kernel configuration utility:

\say{This is the traditional Linux preemption model, geared towards throughput. It will still provide good latencies most of the time, but there are no guarantees and occasional longer delays are possible. Select this option if you are building a kernel for a server or scientific/computation system, or if you want to maximise the raw processing power of the kernel, irrespective of scheduling latencies.}

The kernel installed by Ubuntu 20.04 LTS 64-bit uses the \emph{Voluntary Preemption} model. To use the \emph{No Forced Preemption} model the kernel needs to be recompiled. Detailed instructions on how to do this are included in the \emph{Kernel Build With No Forced Preemption} \verb|picluster| repository wiki page.

The Pi Cluster Tools \verb|linpack-profiler| tool was run with a \emph{No Forced Preemption} kernel. The results are plotted in Figure ??.

\begin{figure}[h]
	\centering
	\includegraphics[width=0.9\textwidth]{profile_nopreempt.pdf}
	\caption{Pi Cluster Tools \texttt{linpack-profiler} $R_{max}$ with a \emph{No Forced Preemption} kernel.}
	\label{fig:subim1}
\end{figure}

Using the parameters which achieved optimum performance when running the \texttt{linpack-profiler} tool, the HPL benchmark results using 80\% of available memory are presented in Table ??. 

\begin{table}[H]
\begin{center}
\begin{tabular}{ |c|c|c|c|c|c|c| } 
\hline
\multicolumn{5}{|c}{} & \multicolumn{2}{|c|}{$\mathbf{R_{max}}$ \textbf{(Gflops)}} \\
\hline
\textbf{BLAS} & \textbf{N} & \textbf{NB} & \textbf{P} & \textbf{Q} & \textbf{Baseline}  & \textbf{No Forced}  \\
              &            &             &            &            &                    & \textbf{Preemption} \\
\hline
OpenBLAS Serial & 52416 & 112 & 2 & 16 &  & 9.5723e+01 \\
\hline
OpenBLAS OpenMP & 52416 &  72 & 2 &  4 &  & 9.3790e+01 \\
\hline
BLIS Serial     & 52320 & 120 & 2 & 16 &  & 1.0021e+02 \\
\hline
BLIS OpenMP     & 52320 & 160 & 2 &  4 &  &  \\
\hline
\end{tabular}
\end{center}
\caption{\label{tab:table-name}Comparison of the Baseline HPL $R_{max}$ with a \emph{No Forced Preemption} kernel.}
\end{table}


\subsubsection{Observations}

Observations and discussion...


%
% SUB SECTION
%
\subsection{Jumbo Frames}

On node2 start the Iperf server...

\lstset{style=type}
\begin{lstlisting}
$ iperf -s
\end{lstlisting}

On node1 start the Iperf client...

\lstset{style=type}
\begin{lstlisting}
$ iperf -c
\end{lstlisting}

ping tests of MTU...




iperf network speed...





\subsubsection{Jumbo Frames}

Requires a network switch capable of Jumbo frames...


On \verb|node2| create the \verb|Iperf| server...

\lstset{style=type}
\begin{lstlisting}
$ iperf -s
\end{lstlisting}

On \verb|node1| create and run the \verb|Iperf| client...

\lstset{style=type}
\begin{lstlisting}
$ iperf -i 1 -c node2
\end{lstlisting}

\lstset{style=type}
\begin{lstlisting}
------------------------------------------------------------
Client connecting to node2, TCP port 5001
TCP window size:  682 KByte (default)
------------------------------------------------------------
[  3] local 192.168.0.1 port 46216 connected with 192.168.0.2 port 5001
[ ID] Interval       Transfer     Bandwidth
[  3]  0.0-10.0 sec  1.15 GBytes   991 Mbits/sec
\end{lstlisting}


\begin{figure}
	\centering	
	\includegraphics[width=1.0\textwidth]{bandwidth_vs_mtu.pdf}
	\caption{Network Node to Node Bandwidth vs MTU.}
\end{figure}



%
% SECTION
%
\section{Power Usage}

\begin{figure}[]
	\begin{subfigure}{1.0\textwidth}
		\centering
		\includegraphics[width=0.5\textwidth]{power-netgear.jpg}
		\caption{Network equipment only.}
		\label{fig:subim1}
	\end{subfigure}
	\par\bigskip
	\begin{subfigure}{1.0\textwidth}
		\centering
		\includegraphics[width=0.5\textwidth]{power-idle.jpg}
		\caption{Aerin Cluster at idle.}
		\label{fig:subim2}
	\end{subfigure}
	\par\bigskip
	\begin{subfigure}{1.0\textwidth}
		\centering
		\includegraphics[width=0.5\textwidth]{power-benchmark.jpg}
		\caption{Aerin Cluster running benchmarks.}
		\label{fig:subim2}
	\end{subfigure}
\caption{\textbf{Aerin Cluster Power Usage.} Similar power usage was observed running either pure OpenMPI or hybrid OpenMPI/OpenMP benchmarks. The idle and benchmark power usage includes the network equipment.}
\label{fig:image2}
\end{figure}









%
% CHAPTER
%
\chapter{Summary}



%
% PART II
%
\part{Build Instructions}

%
% CHAPTER
%
\chapter{The Aerin Cluster}


%
% SECTION
%
\section{Introduction}

This chapter is intended to be a complete and self-contained guide for building a replica of the Aerin Cluster. The items required to build the cluster are listed in Chapter 3.

Throughout this chapter, \verb|macbook| refers to my local workstation which is not part of the Aerin Cluster. This workstation could be another Raspberry Pi or a Windows PC/Laptop. For either case, the following instructions should require little change, providing the workstation has a command line similar to Linux (on a Windows PC this could be MinGW or Windows Subsystem for Linux, or similar).


%
% SECTION
%
\section{Preliminary Tasks}


%
% SUB SECTION
%
\subsection{Update Raspberry Pi EEPROMs}

The firmware for the Raspberry Pi 4 is stored in Electrically Erasable Programmable Read-Only Memory (EEPROM). Updates to the firmware, which include functionality enhancements and bug fixes, are published at regular intervals. These updates are ``flashed'' to the EEPROM.

It is important that each node of the cluster is using the same firmware. This ensures each node operates in the same manner, and performance is uniform across the cluster.

It is recommended to update the EEPROM using the \verb|rpi-eeprom-update| command included with Raspberry Pi OS. A separate MicroSD card was used to install Raspberry Pi OS for this purpose.

Having booted each Raspberry Pi with Raspberry Pi OS, to determine if any firmware updates are available, type the following:

\lstset{style=type}
\begin{lstlisting}[]
$ sudo rpi-eeprom-update
\end{lstlisting}

This will advise if a firmware update is available.

If an update is available, update the firmware using the following commands:

\lstset{style=type}
\begin{lstlisting}[]
$ sudo rpi-eeprom-update -a
$ sudo reboot
\end{lstlisting}

Following the reboot the new firmware will be installed (now is a convenient time to obtain the node MAC address, see next section). On completion, the Raspberry Pi can be shutdown and rebooted with Ubuntu 20.04 LTS 64-bit. 


%
% SUB SECTION
%
\subsection{Obtain Raspberry Pi MAC Addresses}

The MAC address of each node is required to configure the cluster LAN IP address reservations. To determine the MAC address, type the following on each node:

\lstset{style=type}
\begin{lstlisting}[]
$ ip addr show eth0 | grep link/ether 
\end{lstlisting}

This will display output similar to this:

\lstset{style=term}
\begin{lstlisting}[]
link/ether dc:a6:32:60:7b:cd brd ff:ff:ff:ff:ff:ff
\end{lstlisting}

The MAC address is the \verb|dc:a6:32:60:7b:cd| part of the command output. Make a note of this for each node.


%
% SUB SECTION
%
\subsection{Generate User Key Pair}

OpenMPI requires password-less access to each node. This is achieved with \verb|ssh| using public-key encryption. The private and public keys for user \verb|john| are generated on \verb|macbook|. The public key is distributed to each node via \verb|cloud-init| during Ubuntu installation. The private key is subsequently manually copied to \verb|node1|.

To generate the key pair on \verb|macbook| (a passphrase is not used):

\lstset{style=type}
\begin{lstlisting}[]
$ ssh-genkey -t rsa -C john
\end{lstlisting}

This creates two files, the private key \verb|id_rsa|, and the public key \verb|id_rsa.pub|. The contents of the public key file is copied to the \verb|users| section of the \verb|cloud-init| \verb|user-data| file, in this case for user \verb|john|. The private key is copied to \verb|node1| during post-installation tasks.


\subsection{Amend \texttt{macbook} \texttt{/etc/hosts}}

To enable access to a cluster node (from within the cluster LAN) without having to type the LAN IP address, add the following to \verb|macbook| \verb|/etc/hosts|:

\lstset{style=listing}
\begin{lstlisting}[numbers=none, caption=/etc/hosts]
192.168.0.1 node1
192.168.0.2 node2
192.168.0.3 node3
192.168.0.4 node4
192.168.0.5 node5
192.168.0.6 node6
192.168.0.7 node7
192.168.0.8 node8
192.168.0.9 node9
\end{lstlisting}

This enables the easier to remember (and type):

\lstset{style=type}
\begin{lstlisting}[]
$ ssh john@node1
\end{lstlisting}

And, providing the username on \verb|macbook| is the same as the Linux username created by \verb|cloud-init|,
the abbreviated:

\lstset{style=type}
\begin{lstlisting}[]
$ ssh node1
\end{lstlisting}


%
% SUB SECTION
%

\subsection{Router/Firewall Configuration}

The router/firewall configuration consists of the following actions:

\begin{itemize}
  \item Set the WAN IP address, subnet mask and DNS servers
  \item Set the LAN IP address, subnet mask, and DHCP server IP address range
  \item Configure the LAN IP address reservations (this requires the node MAC addresses)
  \item Enable router/firewall remote administration
  \item Enable WAN access to \verb|node1| via \verb|ssh|
  \item Enable router/firewall response to \verb|ping| (for network connectivity testing)
\end{itemize}

Each of these actions is carried out using the router/firewall web-based setup, and is depicted in Figure 7.1 to Figure 7.6.

The default (and unchanged) LAN IP address for the router/firewall is \verb|192.168.0.254|. The default (and unchanged) username and password for the web-based setup is \verb|admin| and \verb|password|.

\begin{figure}[H]
	\centering	
	\includegraphics[width=1.0\textwidth]{router-wan-setup}
	\caption{Router/Firewall WAN Setup. This is the outside-world facing side of the cluster router/firewall. The WAN IP address is set to a static IP address of 192.168.1.253, with an IP subnet mask of 255.255.255.0. In my home environment this is the internal home subnet of my ADSL router. The ADSL router is the internet Gateway with an IP address of 192.168.1.254, and also acts as the Primary DNS Server. The Secondary DNS Server IP address of 8.8.8.8 is Google's public DNS Server. Once re-located to UCL, these IP addresses would be changed to match the UCL's internal network.}
\end{figure}

\begin{figure}[H]
	\centering	
	\includegraphics[width=1.0\textwidth]{router-lan-setup}
	\caption{Router/Firewall LAN Setup. This is the internal cluster LAN side of the router/firewall. The router/firewall has an internal LAN address of 192.168.0.254, with an IP subnet mask of 255.255.255.0. Note, this is a different network than the WAN; 192.168.1 for the WAN, and 192.168.0 for the LAN. The router/firewall DNS Server is enabled for the LAN, and will serve addresses to connecting hosts in the range 192.168.0.101 to 192.168.0.201. These are hosts plugged into one of the LAN sockets of the router/firewall or network switch. This does not include the cluster compute nodes, which have IP addresses reserved based on their MAC addresses. See LAN IP Address Reservations in Figure ??. There should be no need to change these settings when the Aerin Cluster is relocated to UCL.}
\end{figure}

\begin{figure}[H]
	\centering	
	\includegraphics[width=1.0\textwidth]{router-lan-ip-reservations}
	\caption{Router/Firewall LAN IP Address Reservations. To guarantee each compute node is assigned the same LAN IP address across reboots, the router/firewall is configured to serve IP addresses to connecting hosts based upon the host network card MAC address. This is ``reserving" an IP address for a particular host. This router/firewall setup page permits the relationship between MAC and IP addresses to be configured. Each line represents a compute node, or \texttt{macbook}, with a host name, e.g. \texttt{node1}, an IP address to reserve for this host, and the network card MAC address of the host. The MAC address must be known in advance.}
\end{figure}

\begin{figure}[H]
	\centering	
	\includegraphics[width=1.0\textwidth]{router-remote-admin}
	\caption{Router/Firewall Remote Management. It is convenient to be able to configure the router/firewall remotely from the WAN side of the router firewall. This means that a network cable does not have to be plugged into one of the LAN sockets on the router/firewall or network switch to configure the router/firewall. For example, once re-located to UCL, the Aerin Cluster, with router/firewall and switch, can be sited in a suitable location and the router/firewall configured from anywhere on the UCL internal network.}
\end{figure}

\begin{figure}[H]
	\centering	
	\includegraphics[width=1.0\textwidth]{router-ssh-access}
	\caption{Router/Firewall \texttt{ssh} Access. This setup page is used to configure the router/firewall to pass \texttt{ssh} packets from the WAN side of the router/firewall through the firewall to \texttt{node1}. This permits access to the compute nodes from the WAN. So, in a similar manner to the remote administration of the router/firewall, the Aerin Cluster can be sited in a suitable location at UCL, and the Aerin Cluster compute nodes, via \texttt{node1}, can be accessed using \texttt{ssh} from anywhere on the internal UCL network.}
\end{figure}

\begin{figure}[H]
	\centering	
	\includegraphics[width=1.0\textwidth]{router-respond-to-ping}
	\caption{Router/Firewall Respond to \texttt{ping}. Particularly during the build stage of the Aerin Cluster, it was useful to be able to \texttt{ping} the router/firewall to test for network connectivity. For enhanced security, by default, the router/firewall is not configured to respond to \texttt{ping} requests. This setup page enables \texttt{ping} request responses.}
\end{figure}


%
% SECTION
%
\section{Ubuntu 20.04 64-bit LTS Installation}

The idea is to have a single (tweaked) Ubuntu 20.04 64-bit image which can be used to install Ubuntu 20.04 on all of the compute nodes. This is described in Chapter 3. This section details the steps required.


%
% SUB SECTION
%
\subsection{Create the Installation Image}

On \verb|macbook|:

\begin{itemize}
  \item Download the Raspberry Pi 4 Ubuntu 20.04 LTS 64-bit pre-installed server image from the Ubuntu website
  \item Double click the compressed the \verb|.xz| file to extract the \verb|.img| file
  \item Double click the \verb|.img| file to mount the image in the \verb|macbook| filesystem as \verb|/Volumes/system-boot|
  \item Amend the \verb|user-data| file which stores the \verb|cloud-init| configuration, as per Listing 7.2
  \item Eject/unmount the \verb|.img| file
  \item Use Raspberry Pi Imager to erase each node's MicroSD card and burn the Ubuntu image, as per Figure 7.7 and Figure 7.8
\end{itemize}

\lstset{style=listing, float=H}
\lstinputlisting[caption=/Volumes/system-boot/user-data]{picluster/cloud-init/user-data-dissertation}

\begin{figure}[H]
	\centering	
	\includegraphics[width=1.0\textwidth]{screenshots/imager-erase.png}
	\caption{Using Raspberry Pi Imager to erase and format a MicroSD card.}
\end{figure}

\begin{figure}[H]
	\centering	
	\includegraphics[width=1.0\textwidth]{screenshots/imager-write.png}
	\caption{Using Raspberry Pi Imager to write the server image to a MicroSD card.}
\end{figure}



%
% SECTION
%
\section{Ubuntu Installation}

Having burnt the installation image to each MicroSD card, place the card in each node and plug in the power cable. The \verb|cloud-init| configuration process will now start. Each node will acquire its IP address from the router/firewall, setup system users, update the \verb|apt| cache, upgrade the system, download software packages, set the hostname (based on the IP address), and finally reboot.


%
% SECTION
%
\section{Post-Installation Tasks}


%
% SUB SECTION
%
\subsection{Complete the OpenMPI Password-Less Process}

The user \verb|john|'s public key was installed on each node by \verb|cloud-init|.

It remains to copy the private key to \verb|node1|:

\lstset{style=type}
\begin{lstlisting}
$ scp ~/.ssh/id_rsa node1:~/.ssh
\end{lstlisting}

To complete the process the host keys from \verb|node2| to \verb|node9| need to be imported into the \verb|known_hosts| file of \verb|node1|.

From \verb|macbook|, \verb|ssh| into \verb|node1|:

\lstset{style=type}
\begin{lstlisting}[]
$ ssh node1
\end{lstlisting}

Then from \verb|node1|, \verb|ssh| into \verb|node2| to \verb|node9| in turn, for example:

\lstset{style=type}
\begin{lstlisting}[]
$ ssh node2
\end{lstlisting}


This will generate a message similar to this:

\lstset{style=type}
\begin{lstlisting}[]
The authenticity of host 'node2 (192.168.0.2)' can't be established.
ECDSA key fingerprint is SHA256:5VgsnN2nPvpfbJmALh3aJdOeT/NvDXqN8TCreQyNaFA.
Are you sure you want to continue connecting (yes/no/[fingerprint])?
\end{lstlisting}

Responding \verb|yes| to this this message, imports the \verb|node2| host key into the \verb|~/.ssh/known_hosts| file of \verb|node1|.

Then exit from the connected node:

\lstset{style=type}
\begin{lstlisting}[]
$ exit
\end{lstlisting}

This completes the process.

Subsequent \verb|ssh| access to \verb|node2| to \verb|node9| from \verb|node1| will be done using password-less public key authentication. This is the mechanism that OpenMPI will use.


%
% SUB SECTION
%

\subsection{Uninstall \texttt{unattended-upgrades}}

The \verb|unattended-upgrades| package is installed automatically. This can potentially interfere with long running benchmarks, so it is preferable to uninstall this package from each node. This can be done using Pi Cluster Tools.

From \verb|macbook|:

\lstset{style=type}
\begin{lstlisting}[]
$ ssh node1
$ ~/picluster/tools/do "sudo apt remove unattended-upgrades"
\end{lstlisting}

It is important not to forget to manually upgrade the cluster regularly using Pi Cluster Tools:

\lstset{style=type}
\begin{lstlisting}[]
$ ssh node1
$ ~/picluster/tools/upgrade
\end{lstlisting}



%
% SUB SECTION
%
\subsection{Add Ubuntu Source Repositories}

The Ubuntu source repositories are required to rebuilding the kernel and other Ubuntu packages. These repositories are added as follows:

\lstset{style=type}
\begin{lstlisting}[]
$ ssh node1
$ sudo touch /etc/apt/sources/list.d/picluster.list
\end{lstlisting}

Then add the following to the newly created \verb|picluster.list| file:

\lstset{style=listing}
\begin{lstlisting}[caption=/etc/apt/sources.list.d/picluster.list]
deb-src http://archive.ubuntu.com/ubuntu focal main universe
deb-src http://archive.ubuntu.com/ubuntu focal-updates main universe
\end{lstlisting}

And finally, update the \verb|apt| repository cache:

\lstset{style=type}
\begin{lstlisting}[]
$ sudo apt update
\end{lstlisting}



\subsection{Create a Project Repository}

A project repository on \verb|node1| is required to hold all project software and results. This repository is mirrored with the GitHub project repository.

The project repository is created as follows:

\lstset{style=type}
\begin{lstlisting}[]
$ ssh node1
$ mkdir picluster
$ cd picluster
$ git init
\end{lstlisting}



%
% CHAPTER
%
\chapter{Install High-Performance Linpack (HPL)}

These instructions are derived from the \verb|INSTALL| and \verb|README| files in the \verb|hpl-2.3| top level source directory.

Download and install the latest version of HPL on \verb|node1|:

\lstset{style=type}
\begin{lstlisting}
$ ssh node1
$ cd ~/picluster
$ mkdir hpl
$ cd hpl
$ wget https://www.netlib.org/benchmark/hpl/hpl-2.3.tar.gz
$ gunzip hpl-2.3.tar.gz
$ tar xvf hpl-2.3.tar
$ rm hpl-2.3.tar
$ cd hpl-2.3
\end{lstlisting}

Each computer system requires a specific \verb|Make.picluster| file, which specifies the operating system commands and software package locations required to build HPL. 

Create a generic \verb|Make.picluster| file:

\lstset{style=type}
\begin{lstlisting}
$ cd setup
$ bash make_generic
$ cp Make.UNKNOWN ../Make.picluster
$ cd ..
\end{lstlisting}

Amend \verb|Make.picluster| with the specifics of the Raspberry Pi 4 and Ubuntu 20.04 as follows. The changes below are changes to the generic file created above.

Set the \emph{shell} to use:

\lstset{style=listing}
\begin{lstlisting}[numbers=none]  
SHELL        = /usr/bin/bash
\end{lstlisting}

Set the commands to use (these may vary form operating system to operating system):

\lstset{style=listing}
\begin{lstlisting}[numbers=none]  
CD           = cd
CP           = cp
LN_S         = ln -s
MKDIR        = mkdir -p
RM           = rm -f
TOUCH        = touch
\end{lstlisting}

Set the platform identifier:

\lstset{style=listing}
\begin{lstlisting}[numbers=none]  
ARCH         = picluster
\end{lstlisting}

Set the top level source directory:

\lstset{style=listing}
\begin{lstlisting}[numbers=none]  
TOPdir       = $(HOME)/picluster/hpl/hpl-2.3
\end{lstlisting}

Set the location of the OpenMPI library:

\lstset{style=listing}
\begin{lstlisting}[numbers=none]  
MPdir        = /usr/lib/aarch64-linux-gnu/openmpi
MPinc        = $(MPdir)/include
MPlib        = $(MPdir)/lib/libmpi.so
\end{lstlisting}

Set the location of the BLAS library (Note, this is the location of the Debian \emph{alternatives} \verb|libblas.so.3| library. The actual library that this points to, OpenBLAS or BLIS, is set through the Debian \verb|update-alternatives| command. This can be conveniently done using Pi Cluster Tools.):

\lstset{style=listing}
\begin{lstlisting}[numbers=none]  
LAdir        = /usr/lib/aarch64-linux-gnu
LAinc        =
LAlib        = $(LAdir)/libblas.so.3
\end{lstlisting}

Set the ``Fortran to C'' defintitions, the header directories and libraries: 

\lstset{style=listing}
\begin{lstlisting}[numbers=none]  
F2CDEFS      = -DAdd_ -DF77_INTEGER=int -DStringSunStyle
...
HPL_INCLUDES = -I$(INCdir) -I$(INCdir)/$(ARCH) -I$(MPinc)
HPL_LIBS     = $(HPLlib) $(LAlib) $(MPlib)
...
HPL_DEFS     = $(F2CDEFS) $(HPL_OPTS) $(HPL_INCLUDES)
\end{lstlisting}

And finally, set the compiler, linker and optimisation flags:

\lstset{style=listing}
\begin{lstlisting}[numbers=none]  
CC           = mpicc
CCNOOPT      = $(HPL_DEFS)
CCFLAGS      = $(HPL_DEFS) -O3 -march=armv8-a -mtune=cortex-a72
...
LINKER       = $(CC)
LINKFLAGS    = $(CCFLAGS)
...
ARCHIVER     = ar
ARFLAGS      = r
RANLIB       = echo
\end{lstlisting}

Build HPL:

\lstset{style=type}
\begin{lstlisting}
$ make arch=picluster   
\end{lstlisting}

This creates the executable \verb|xhpl| and input file \verb|HPL.dat| in the \verb|bin/picluster| directory.

The \verb|xhpl| executable has to exist in the same location on each node, so copy \verb|xhpl| to \verb|node2| to \verb|node8| (only \verb|xhpl|, and not \verb|HPL.dat|):

\lstset{style=type}
\begin{lstlisting}
$ cd bin/picluster
$ ~/picluster/tools/do "mkdir -p picluster/hpl/hpl-2.3/bin/picluster"
$ scp xhpl node2:~picluster/hpl/hpl-2.3/bin/picluster
$ scp xhpl node3:~picluster/hpl/hpl-2.3/bin/picluster
$ scp xhpl node4:~picluster/hpl/hpl-2.3/bin/picluster
$ scp xhpl node5:~picluster/hpl/hpl-2.3/bin/picluster
$ scp xhpl node6:~picluster/hpl/hpl-2.3/bin/picluster
$ scp xhpl node7:~picluster/hpl/hpl-2.3/bin/picluster
$ scp xhpl node8:~picluster/hpl/hpl-2.3/bin/picluster
\end{lstlisting}


%
% CHAPTER
%
\chapter{Install HPC Challenge (HPCC)}

These instructions are derived from the README.txt file in the top level directory of the HPCC source code.

It is assumed that HPL has previously been installed, and a HPL build file \verb|Make.picluster| has already been created. This file is copied to the HPCC build directory. See Chapter 8 for the instructions on how to install HPL.

Download and install the latest version of HPCC on \verb|node1|:

\lstset{style=type}
\begin{lstlisting}
$ ssh node1
$ cd ~/picluster
$ mkdir hpcc
$ cd hpcc
$ wget http://icl.cs.utk.edu/projectsfiles/hpcc/download/hpcc-1.5.0.tar.gz
$ gunzip hpcc-1.5.0.tar.gz
$ tar xvf hpcc-1.5.0.tar
$ rm hpcc-1.5.0.tar
$ cd hpcc-1.5.0
\end{lstlisting}

Copy the HPL build script \verb|Make.picluster| to the HPCC \verb|hpl| directory:

\lstset{style=type}
\begin{lstlisting}
$ cd hpl
$ cp ~/picluster/hpl/hpl-2.3/Make.picluster .
\end{lstlisting}

Make the following changes to \verb|Make.picluster|. These differ from the HPL build instructions:

Change the \verb|TOPdir| variable:

\lstset{style=hack}
\begin{lstlisting}[caption=Make.picluster]
TOPdir = ../../..
\end{lstlisting}

Add the \verb|math| library explicitly:

\lstset{style=hack}
\begin{lstlisting}[caption=Make.picluster]
LAlib = $(LAdir)/libblas.so.3 -lm
\end{lstlisting}

Add the constant \verb|OMPI_OMIT_MPI1_COMPAT_DECLS| to \verb|CCFLAGS|, otherwise the compilation will fail:

\lstset{style=hack}
\begin{lstlisting}[caption=Make.picluster]
CCFLAGS = $(HPL_DEFS) -O3 -march=armv8-a -mtune=cortex-a72 -DOMPI_OMIT_MPI1_COMPAT_DECLS
\end{lstlisting}

Move back up into the top level directory:

\lstset{style=type}
\begin{lstlisting}
$ cd ..
\end{lstlisting}

Build HPCC:

\lstset{style=type}
\begin{lstlisting}
$ make arch=picluster
\end{lstlisting}

Copy the \verb|hpcc| executable to the compute nodes:

\lstset{style=type}
\begin{lstlisting}
$ ~/picluster/tools/do "mkdir -p ~/picluster/hpcc/hpcc-1.5.0"
$ scp hpcc node2:~/picluster/hpcc/hpcc-1.5.0
$ scp hpcc node3:~/picluster/hpcc/hpcc-1.5.0
$ scp hpcc node4:~/picluster/hpcc/hpcc-1.5.0
$ scp hpcc node5:~/picluster/hpcc/hpcc-1.5.0
$ scp hpcc node6:~/picluster/hpcc/hpcc-1.5.0
$ scp hpcc node7:~/picluster/hpcc/hpcc-1.5.0
$ scp hpcc node8:~/picluster/hpcc/hpcc-1.5.0
\end{lstlisting}

Create the input file \verb|hpccinf.txt|:

\lstset{style=type}
\begin{lstlisting}
$ cp _hpccinf.txt hpccinf.txt
\end{lstlisting}

The input file is amended as necessary for each benchmark run, as per the HPL input file.

After each benchmark run the results will be in the output file \verb|hpccoutf.txt|.



%
% CHAPTER
%
\chapter{Install High Performance Conjugate Gradients (HPCG)}

These instructions are derived from the INSTALL and QUICKSTART files in the HPCG 3.1 top-level source directory.

\lstset{style=hack}
\begin{lstlisting}
The main build difference between HPCG and HPL is that HPCG can be built as either a single-threaded serial program, or a multi-threaded OpenMP program. It is not the BLAS library which is either single or multi-threaded. In fact, HPCG does not use a BLAS library. To investigate the performance of HPCG in either single-threaded or multi-threaded versions requires building two HPCG programs.
\end{lstlisting}

Download and install the latest version of HPCG on \verb|node1|:

\lstset{style=type}
\begin{lstlisting}
$ ssh node1
$ cd ~/picluster
$ mkdir hpcg
$ cd hpcg
$ wget http://www.hpcg-benchmark.org/downloads/hpcg-3.1.tar.gz
$ gunzip hpcg-3.1.tar.gz
$ tar xvf hpcg-3.1.tar
$ rm hpcg-3.1.tar
$ cd hpcg-3.1
\end{lstlisting}


%
% SECTION
%
\section{Serial HPCG}

Create a \verb|Make.picluster_serial| file for the serial build:

\lstset{style=type}
\begin{lstlisting}
$ cp setup/Make.Linux_serial setup/Make.picluster_serial
\end{lstlisting}

Amend \verb|setup/Make.picluster_serial| as follows.

Set the shell:

\lstset{style=listing}
\begin{lstlisting}[numbers=none]
SHELL = /usr/bin/bash
\end{lstlisting}

Set the top level directory:

\lstset{style=listing}
\begin{lstlisting}[numbers=none]
TOPdir = $(HOME)/picluster/hpcg/hpcg-3.1
\end{lstlisting}

Set the OpenMPI package location:

\lstset{style=listing}
\begin{lstlisting}[numbers=none]
MPdir = /usr/lib/aarch64-linux-gnu/openmpi
MPinc = $(MPdir)/include
MPlib = $(MPdir)/lib/libmpi.so
\end{lstlisting}

Include the OpenMPI header files and library:

\lstset{style=listing}
\begin{lstlisting}[numbers=none]
HPCG_INCLUDES = -I$(INCdir) -I$(INCdir)/$(arch) -I$(MPinc)
HPCG_LIBS     = $(MPlib)
\end{lstlisting}

Ensure HPCG is built without OpenMP support:

\lstset{style=listing}
\begin{lstlisting}[numbers=none]
HPCG_OPTS = -DHPCG_NO_OPENMP
\end{lstlisting}

Set C++ compiler flags:

\lstset{style=listing}
\begin{lstlisting}[numbers=none]
CXX      = mpic++
CXXFLAGS = $(HPCG_DEFS) -O3 -march=armv8-a -mtune=cortex-a72
\end{lstlisting}

Build HPCG:

\lstset{style=type}
\begin{lstlisting}[numbers=none]
$ mkdir build_serial
$ cd build_serial
$ ../configure picluster_serial
$ make
\end{lstlisting}

This creates the serial version of the \verb|xhpcg| executable and the \verb|hpcg.dat| input file in the \verb|build_serial/bin| directory.


%
% SECTION
%
\section{OpenMP HPCG}

Create a \verb|Make.picluster_openmp| file for the OpenMP build:

\lstset{style=type}
\begin{lstlisting}
$ cp setup/Make.Linux_serial setup/Make.picluster_openmp
\end{lstlisting}

Amend \verb|setup/Make.picluster_openmp| as per \verb|setup/Make.pcluster_serial|, with the exceptions of not disabling OpenMP, i.e. leave HPCG\_OPTS blank, and adding \verb|-fopenmp| to the compiler flags:

\lstset{style=listing}
\begin{lstlisting}[numbers=none]
HPCG_OPTS = 
\end{lstlisting}

\lstset{style=listing}
\begin{lstlisting}[numbers=none]
CXXFLAGS = $(HPCG_DEFS) -O3 -march=armv8-a -mtune=cortex-a72 -fopenmp
\end{lstlisting}

This is a bug fix for \verb|src/ComputeResidual.cpp| line 56. Add the variable \verb|n| to the shared variables list of \verb|omp parallel| clause, otherwise a compiler error is generated:

\lstset{style=hack}
\begin{lstlisting}[numbers=none]
#pragma omp parallel default(none) shared(n, local_residual, v1v, v2v)
\end{lstlisting}


Build HPCG:

\lstset{style=type}
\begin{lstlisting}[numbers=none]
$ mkdir build_openmp
$ cd build_openmp
$ ../configure picluster_openmp
$ make
\end{lstlisting}

This creates the OpenMP version of the \verb|xhpcg| executable and the \verb|hpcg.dat| input file in the \verb|build_openmp/bin| directory.


%
% CHAPTER
%
\chapter{Ubuntu Kernel Build Procedure}

This procedure is derived from the Ubuntu Wiki BuildYourOwnKernel document...

Make sure you have made the source code repositories available as per...

Create a kernel build directory with the correct directory permissions to prevent source download warnings. 

\lstset{style=type}
\begin{lstlisting}
$ ssh node1
$ mkdir -p ~/picluster/build/kernel
$ sudo chown _apt:root ~/picluster/build/kernel
$ cd ~/picluster/build/kernel
\end{lstlisting}

Install the kernel build dependencies...

\lstset{style=type}
\begin{lstlisting}
$ sudo apt-get build-dep linux linux-image-$(uname -r)
\end{lstlisting}

Download the kernel source...

\lstset{style=type}
\begin{lstlisting}
$ sudo apt-get source linux-image-$(uname -r)
$ cd linux-raspi-5.4.0
\end{lstlisting}

This bit is a fix for the subsequent \verb|editconfigs| step of the build procedure...

\lstset{style=type}
\begin{lstlisting}
$ cd debian.raspi/etc
$ sudo cp kernelconfig kernelconfig.original
$ sudo vim kernelconfig
\end{lstlisting}

And make the following change...

\lstset{style=listing}
\begin{lstlisting}[caption=diff kernelconfig kernelconfig.original, numbers=none]
5c5
< 	archs="arm64"
---
> 	archs="armhf arm64"
\end{lstlisting}

Then move back up to the kernel source top level directory...

\lstset{style=type}
\begin{lstlisting}
$ cd ../..
\end{lstlisting}

Prepare the build scripts...

\lstset{style=type}
\begin{lstlisting}
$ sudo chmod a+x debian/rules
$ sudo chmod a+x debian/scripts/*
$ sudo chmod a+x debian/scripts/misc/*
\end{lstlisting}

SOURCE CHANGES AND/OR verb|editconfigs| AT THIS POINT

\lstset{style=type}
\begin{lstlisting}
$ sudo apt install libncurses-dev
$ sudo LANG=C fakeroot debian/rules clean
$ sudo LANG=C fakeroot debian/rules editconfigs
\end{lstlisting}

Tweak the kernel name for identification...

\lstset{style=type}
\begin{lstlisting}
$ cd debian.raspi
$ sudo cp changelog changelog.original
$ sudo vim changelog
\end{lstlisting}

And make the following change, where \verb|+picluster0| is our kernel identifier...

\lstset{style=listing}
\begin{lstlisting}[caption=diff changelog changelog.original, numbers=none]
1c1
< linux-raspi (5.4.0-1015.15+picluster0) focal; urgency=medium
---
> linux-raspi (5.4.0-1015.15) focal; urgency=medium
\end{lstlisting}

Move up to the top level kernel source directory...

\lstset{style=type}
\begin{lstlisting}
$ cd ..
\end{lstlisting}

And build the kernel...

\lstset{style=type}
\begin{lstlisting}
$ sudo LANG=C fakeroot debian/rules clean
$ sudo LANG=C fakeroot debian/rules binary-arch
cd ..
\end{lstlisting}

Install the new kernel...

\lstset{style=type}
\begin{lstlisting}
$ sudo dpkg -i linux*picluster0*.deb
$ sudo shutdown -r now
\end{lstlisting}

Another build procedure fix...

After each kernel build delete the \verb|linux-libc-dev| directory...

\lstset{style=type}
\begin{lstlisting}
$ cd ~/picluster/build/kernel/linux-raspi-5.4.0/debian
$ rm -rf linux-libc-dev
$ cd ..
\end{lstlisting}


%
% CHAPTER
%
\chapter{Build Kernel with No Pre-Emption Scheduler}


%
% CHAPTER
%
\chapter{Build Kernel with Jumbo Frames Support}

Standard MTU is 1500 bytes...

Maximum payload size is 1472 bytes...

NB of 184 (x 8 bytes for Double Precision) = 1472 bytes...

NB $>$ 184 $=>$ packet fragmentation $=>$ reduced network efficiency...

This causes drop of in performance???...

Max MTU on Raspberry Pi 4 Model B is set at build time to 1500...

Not configurable above 1500...

TODO: EXAMPLE OF ERROR MSG...

Need to build the kernel with higher MTU...


Make the required changes to the source... as per REFERENCE

\begin{verbatim}
    cd linux-raspi-5.4.0 

    sudo vim include/linux/if_vlan.h...
        #define VLAN_ETH_DATA_LEN   9000
        #define VLAN_ETH_FRAME_LEN  9018
    
    sudo vim include/uapi/linux/if_ether.h...
        #define ETH_DATA_LEN        9000
        #define ETH_FRAME_LEN       9014
    
    sudo vim drivers/net/ethernet/broadcom/genet/bcmgenet.c...
        #define RX_BUF_LENGTH       10240
\end{verbatim}

Add a Jumbo Frames identifier, "+jf", to the new kernel name...

\begin{verbatim}
    sudo vim debian.raspi/changelog...
        linux (5.4.0-1013.13+jf) focal; urgency=medium
        
\end{verbatim}


%
% CHAPTER
%
\chapter{Rebuild OpenBLAS}

\lstset{style=type}
\begin{lstlisting}
$ ssh node1
$ mkdir -p build/openblas
$ chown -R _apt:root build
$ cd build/openblas
$ sudo apt-get source openblas
$ sudo apt-get build-dep openblas
$ cd openblas-0.3.8+ds
\end{lstlisting}


Edit cpuid\_arm64.c...

\lstset{style=type}
\begin{lstlisting}
$ sudo cp cpuid_arm64.c cpuid_arm64.c.original
$ sudo vim cpuid_arm64.c
\end{lstlisting}


\lstset{style=type}
\begin{lstlisting}
$ diff cpuid_arm64.c cpuid_arm64.c.original
\end{lstlisting}

\lstset{style=type}
\begin{lstlisting}
275c275
<       printf("#define L2_SIZE 1048576\n");
---
>       printf("#define L2_SIZE 524288\n");
278c278
<       printf("#define DTB_DEFAULT_ENTRIES 32\n");
---
>       printf("#define DTB_DEFAULT_ENTRIES 64\n");
\end{lstlisting}


And, then following the instructions in debian/README.Debian

\lstset{style=type}
\begin{lstlisting}
$ DEB_BUILD_OPTIONS=custom dpkg-buildpackage -uc -b
\end{lstlisting}

Once the build is complete..

\lstset{style=type}
\begin{lstlisting}
cd ..
$ sudo apt remove libopenblas0-serial
$ sudo dpkg -i libopenblas0-serial\_0.3.8+ds-1\_arm64.deb
\end{lstlisting}

Ensure the correct BLAS library is being used...

\lstset{style=type}
\begin{lstlisting}
$ sudo update-alternatives --config libblas.so.3-aarch64-linux-gnu
\end{lstlisting}

copy to other nodes
remove old...
install new...

If more than one BLAS library is installed, check update-alternatives!!!

ssh node2 .. node8
\lstset{style=type}
\begin{lstlisting}
$ ssh node2 sudo apt remove libblas0-serial
$ scp libopenblas0-serial\_0.3.8+ds-1\_arm64.deb node2:~
$ ssh sudo dpkg -i libopenblas0-serial\_0.3.8+ds-1\_arm64.deb
$ ssh sudo update-alternatives --config libblas.so.3-aarch64-linux-gnu
\end{lstlisting}


%
% CHAPTER
%
\chapter{Rebuild BLIS}

\lstset{style=type}
\begin{lstlisting}
$ ssh node1
$ mkdir -p picluster/build/blis
$ cd picluster/build/blis
$ apt-get source blis
$ sudo apt-get build-dep blis
$ cd blis-0.6.1
\end{lstlisting}


%
% CHAPTER
%

\chapter{Build OpenMPI from Source}

Do all of this on node1...

\lstset{style=type}
\begin{lstlisting}
$ ssh node1
\end{lstlisting}

We want to avoid collisions with multiple OpenMPI installations, so remove original installed version...

\lstset{style=type}
\begin{lstlisting}
$ sudo apt remove openmpi-common
$ sudo apt remove openmpi-bin
$ sudo apt autoremove 
\end{lstlisting}

OpenMPI requires the libevent-dev package...

\lstset{style=type}
\begin{lstlisting}
$ sudo apt install libevent-dev
\end{lstlisting}

Create a build directory, and download and, and and following BLAH, BLAH build OpenMPI...

\lstset{style=type}
\begin{lstlisting}
$ mkdir -p ~/picluster/build/openmpi
$ cd ~/picluster/build/openmpi
$ wget https://download.open-mpi.org/release/open-mpi/v4.0/openmpi-4.0.4.tar.gz
$ gunzip openmpi-4.0.4.tar.gz
$ tar xvf openmpi-4.0.4.tar
$ rm openmpi-4.0.4.tar
$ cd openmpi-4.0.4
$ mkdir build
$ cd build
$ ../configure CFLAGS="-O3 -march=armv8-a -mtune=cortex=a72"
$ make all
$ sudo make install
$ sudo ldconfig
\end{lstlisting}

OpenMPI will installed to /usr/local

EXTRACT FROM HPL.dat


TODO: HOW TO COPY TO ALL NODES!


%
% CHAPTER
%
\chapter{Aerin Cluster Tools}

\lstinputlisting[caption=picluster/tools/upgrade, numbers=left, backgroundcolor=\color{LightSkyBlue}]{picluster/tools/upgrade}
\lstinputlisting[caption=picluster/tools/reboot, numbers=left, backgroundcolor=\color{LightSkyBlue}]{picluster/tools/reboot}
\lstinputlisting[caption=picluster/tools/shutdown, numbers=left, backgroundcolor=\color{LightSkyBlue}]{picluster/tools/shutdown}
\lstinputlisting[caption=picluster/tools/libblas-query, numbers=left, backgroundcolor=\color{LightSkyBlue}]{picluster/tools/libblas-query}
\lstinputlisting[caption=picluster/tools/libblas-set, numbers=left, backgroundcolor=\color{LightSkyBlue}]{picluster/tools/libblas-set}


%
% CHAPTER
%

\chapter{Arm Performance Libraries}

\textcolor{red}{This does not work, yet! HPL will compile and link to Arm Performance Libraries, but raises an illegal instruction error at runtime.}

\textcolor{red}{At the time of writing, Arm Performance Libraries release 20.2.0 requires a minimum Instruction Set Architecture (ISA) of armv8.1-a. Unfortunately, the Raspberry Pi's Cortex-A72 ISA is armv8.0-a. An Arm representative has indicated on the Arm HPC Forum that the next release of the libraries will support the armv8.0-a ISA.}

\textcolor{red}{This Chapter is included for future reference.}

The Arm Performance Libraries website states:

"Arm Performance Libraries provides optimised standard core math libraries for high-performance computing applications on Arm processors. This free version of the libraries provides optimised libraries for Arm® Neoverse™ N1-based Armv8 AArch64 implementations that are compatible with various versions of GCC. You do not require a license for this version of the libraries."

To install Arm Performance Libraries, firstly downloaded Arm Performance Libraries 20.2.0 with GCC 9.3 for Ubuntu 16.04+ from the Arm website.

Then follow these instructions.

\lstset{style=type}
\begin{lstlisting}
$ ssh node1
\end{lstlisting}

Install the required \verb|environment_modules| package.

\lstset{style=type}
\begin{lstlisting}
$ sudo apt install environment-modules
\end{lstlisting}

Then extract and install Arm Performance Libraries.

The default installation directory is /opt/arm.

\lstset{style=type}
\begin{lstlisting}
$ mkdir ~/picluster/armpl
$ cd ~/picluster/armpl
$ tar xvf arm-performance-libraries_20.2_Ubuntu-16.04_gcc-9.3.tar
$ rm arm-performance-libraries_20.2_Ubuntu-16.04_gcc-9.3.tar
$ sudo ./arm-performance-libraries_20.2_Ubuntu-16.04.sh
\end{lstlisting}

Copy the \verb|Make.picluster| configuration file.

\lstset{style=type}
\begin{lstlisting}
$ cd ~/picluster/hpl/hpl-2.3
$ cp Make.picluster Make.picluster-armpl
\end{lstlisting}

Make the following changes to \verb|Make.picluster-armpl|.

\lstset{style=listing}
\begin{lstlisting}[caption=Make.picluster-armpl, numbers=none]
LAdir        = /opt/arm/armpl_20.2_gcc-9.3
LAinc        =
LAlib        = -L$(LAdir)/lib -larmpl -lgfortran -lamath -lm
\end{lstlisting}

Build HPL.

\lstset{style=type}
\begin{lstlisting}
$ make arch=picluster-armpl
\end{lstlisting}


%
% THE END
%
\end{document}
